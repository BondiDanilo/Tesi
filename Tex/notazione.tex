\chapter{Notazione}
%
\subsubsection{Coordinate polari:}
 $\theta \in[0,\pi]$ Angolo polare\\
 $\varphi \in [0,2\pi]$ Angolo azimutale\\
 $r \in (0,\infty)$ Raggio

\subsubsection{Versori:}
\begin{table}[h]
   \begin{tabular}{llll}
      $\vec u_x = (1,0,0)$ Versore asse x & &
         $\vec u _\theta = \frac{1}{r\sqrt{r^2-z^2}}(xz,yz,-z^2-r^2)$ &
         Versore angolo polare \\
      $\vec u_y = (0,1,0)$ Versore asse y & &
         $\vec u _\varphi = \frac{1}{\sqrt{r^2-z^2}}(-x,y,0)$ &
         Versore angolo azimutale \\
      $\vec u_z = (0,0,1)$ Versore asse z & &
         $\vec u _r = \frac{1}{r}(x,y,z) = \frac{\vec r}{r}$ &
         Versore radiale
   \end{tabular}
\end{table}

\begin{equation}\label{eq:versors}
   \begin{cases}
       \vec u_r = \sin\theta \cos\varphi \vec u _x
                + \sin\theta \sin\varphi \vec u _y + \cos\theta \vec u _z\\
       \vec u_\theta = \cos\theta \cos\varphi \vec u _x +
                       \cos\theta \sin\varphi \vec u _y - \sin\theta \vec u _z\\
       \vec u_\varphi = -\sin\varphi \vec u _x + \cos\varphi \vec u _y
   \end{cases}
\end{equation}

\subsubsection{Gruppi di matrici}
\begin{itemize}
   \item $GL(n,\mathbb{F}) = \{M \in Mat(n,\mathbb{F}) \tc \mathrm{det} M \neq 0)\}$\\
      \ttab Gruppo Lineare Generale: Matrici $n \times n$ sul campo $\mathbb{F}$,
      invertibili.

   \item $SL(n,\mathbb{F}) = \{ M \in GL(n,\mathbb{F}) \tc \mathrm{det} M = 1)\}$\\
      \ttab Gruppo Lineare Speciale: Matrici a determinante 1.

   \item $O(n) = \{ M \in GL(n,\R) \tc  M^T M = M M^T = 1 \}$\\
      \ttab Gruppo Ortogonale: Matrici ortogonali.

   \item $SO(n) = \{ M \in O(n)\tc \mathrm{det} M  = 1  \}$\\
      \ttab Gruppo Ortogonale Speciale: Matrici ortogonali a determinante 1.

   \item $U(n) = \{ M \in GL(n,\C)\tc M^\dagger M = M M^\dagger = 1\}$\\
      \ttab Gruppo Unitario: Matrici unitarie.

   \item $SU(n) = \{ M \in U(n)\tc \mathrm{det}M = 1  \}$\\
      \ttab Gruppo Unitario Speciale: Matrici unitarie a determinante 1.
\end{itemize}

Le matrici di $SU(2)$ hanno la forma
\begin{equation}\label{eq:matSU2}
   U =
        \begin{pmatrix}
          z^* & w^* \\
          -w  & z   \\
        \end{pmatrix}
     =
       \begin{pmatrix}
          \cos \frac{\theta}{2} e^{-i\alpha} & \sin \frac{\theta}{2} e^{-i(\varphi + \alpha)} \\
         -\sin \frac{\theta}{2} e^{i(\varphi + \alpha)} & \cos \frac{\theta}{2} e^{i\alpha}  \\
       \end{pmatrix}
\end{equation}
Dove $z,w \in \C$ tali che $zz^* + ww^* = 1$
(affinchè $U^\dagger U = UU^\dagger = 1$), $\theta,\varphi$ sono gli angoli delle
coordinate polari e $\alpha$ è il parametro ciclico del gruppo $U(1)$, cioè
$g \in U(1) \to g = e^{i\alpha}$

\subsubsection{Matrici di Pauli}
\begin{equation}\label{eq:paulimatrix}
   \begin{aligned}
      \sigma_1 &= \begin{pmatrix} 0 & 1  \\ 1 & 0  \end{pmatrix}&
      \sigma_2 &= \begin{pmatrix} 0 & -i \\ i & 0  \end{pmatrix}&
      \sigma_3 &= \begin{pmatrix} 1 & 0  \\ 0 & -1 \end{pmatrix}&
   \end{aligned}
\end{equation}
\subsubsection{(pseudo)Tensore di Levi-Civita}
$$
   \varepsilon_{ijk} := \begin{cases}
      1  & \mathrm{se \:} i,j,k \mathrm{\: permutazione \:   pari \: di\:} 1,2,3 \\
      -1 & \mathrm{se \:} i,j,k \mathrm{\: permutazione \: dispari\: di\:} 1,2,3 \\
      0  & \mathrm{altrimenti}
   \end{cases}
$$

\subsubsection{Equazioni di Maxwell nel vuoto}
\begin{equation}
   \begin{aligned}
      \partial _\nu F^{\mu\nu} & = 0 ,&
      \partial _\nu \tilde{F}^{\mu\nu}
         & =  \varepsilon^{\mu\nu\alpha\beta} \partial _\nu F_{\alpha\beta}
         = 0
   \end{aligned}
\end{equation}
