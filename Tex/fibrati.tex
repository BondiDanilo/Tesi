\section{Fibrati}\label{}
Si vuole iniziare con un esempio per rendere più chiaro l'argomento.
%------------------------------------------------------------------------------%
%------------------------------------------------------------------------------%
\subsection{Fibrato Tangente}\label{sec:fibratotangente}
Sia $M$ una varietà differenziale di dimensione $n$. Si definisce il \emph{fibrato tangente}
su $M$ (detta \emph{spazio base}) l'unione di tutti gli spazi tangenti alla varietà, indicato con $TM$.
   $$ TM := \bigcup\limits_{p \in M} T_p(M) $$
Si consideri una carta $(U,\phi)$, $U$ intorno di $p \in M$ e $x^i = \phi^i(p)$
coordinate. Gli elementi dello spazio $TU = \cup_{p \in U} T_p(M)$ sono
individuati da un punto $p \in U$ e un vettore tangente $V = \sum_i V^i
\left. \frac{\partial}{\partial x^i} \right|_p \in T_p(M)$. Per costruzione
$U$ è omeomorfo all'aperto $\phi(U) \subset \mathbb{R}^n$ e $T_p(M)$ è omeomorfo
a $\mathbb{R}^n$ stesso tramite l'identificazione tra derivata direzionale e
vettore (si veda la sezione \ref{sec:vardiff}).\\
Allora ogni punto $P \in TU$ può essere identificato con il punto
$(x^1,\dots,x^n,V^1,\dots,V^) \in \mathbb{R}^n \times \mathbb{R}^n$ in maniera
univoca. $TU$ è quindi una varietà differenziale di dimensione $2n$.\\
Inoltre $TU$ è identificato con $\mathbb{R}^n \times \mathbb{R}^n$ ed è
esattamente decomposto nel prodotto diretto $U \times \mathbb{R}^n$, cioè
ogni $P \in TU$ può essere scritto come $(p,V),\: p\in U V \in T_p(M)$.\\
Si può quindi definire la \emph{proiezione} $\pi : TU \to U \:,\: P=(p,V) \mapsto p$.
Lo spazio $T_p(M) = \pi^{-1}(p)$ viene detto \emph{fibra} in $p$.\\
Se $M = \mathbb{R}^n$ si ha che $TM = \mathbb{R}^n \times \mathbb{R}^n$ e si dice
che il fibrato ha struttura \emph{banale}. In generale però si ha una struttura
non banale ed occorre tener conto di tutte le carte possibili.\\
Siano $(U,\phi)$ e $(V,\psi)$ due carte tali che $U \cap V \neq \varnothing
\:,\: p\in U \cap V$. Siano $x^i = \phi^i(p)$ e $y^j = \psi^j(p)$, e sia $V \in T_p(M)$.
$V$ in coordinate è espresso come
$$
   V = \sum_i V^i  \left. \frac{\partial}{\partial x^i} \right |_p
     = \sum_j V'^j \left. \frac{\partial}{\partial y^j} \right |_p
     = \sum_{j,k}  V^k \left. \frac{\partial y^j}{\partial x^k} \right |_p
        \left. \frac{\partial}{\partial y^j} \right |_p
$$
dove $V'^j = \sum_k \left. \frac{\partial y^j}{\partial x^k}\right |_pV^k $.
L'applicazione $\psi \circ \phi$ deve essere invertibile, quindi la matrice Jacobiana
deve essere non singolare, cioè $J^j_i = \frac{\partial y^j}{\partial x^i} \in GL(n,\mathbb{R})$.
Il gruppo $GL(n,\mathbb{R})$ viene chiamato \emph{gruppo di struttura} di $TM$.
Le coordinate delle fibre, in seguito a un cambio di coordinate, risultano ruotate
per un elemento del gruppo di struttura.\\
Infine si definisce \emph{sezione} di $TM$ una mappa liscia $s : M \to TM$ tale
che $\pi \circ s = id_M$, ossia che a $p \in M$ associa un elemento di $TM$ $(p,V)
\: ,\: V \in T_p(M)$. Se $s$ è definita solo in un intorno $U$ viene detta sezione locale.\\

Avendo definito una struttura differenziale su $TM$, un campo vettoriale su $M$
può essere visto come una mappa liscia $M \to TM$ che a $p \in M$ associa $V_p \in T_p(M)$\\

%------------------------------------------------------------------------------%
%------------------------------------------------------------------------------%
\subsection{Fibrato}
Siano $M$ (detta \emph{spazio base}) e $F$ (detta \emph{fibra}) varietà diferenziali
(si pensi all'analogia con il fibrato tangente in cui $F=T_p(M)$) e sia
$\mathcal{U} = \{U_i\}$ un ricoprimento aperto di $M$.\\

Intuitivamente, un \emph{fibrato} $E$ su $M$ con fibra $F$ è una varietà diferenziale
(detta anche \emph{spazio totale}) che è localmente un prodotto diretto di $M$ e $F$,
ossia il fibrato $E$ è descritto topologicamente in ogni intorno $U_i$
dalla varietà prodotto $U_i \times F$.\\

Si definisce una funzione $\pi : E \to M$ continua e suriettiva (la
\emph{proiezione di fibra}) che mappa ogni fibra $F_p = \{ (p,f) | f \in F\}
\subset E$ nel punto $ p\in M$, e che rispetti la seguente condizione.\\

Si definisce un'operazione duale alla proiezione, la \emph{sezione} del fibrato
come una mappa tra lo spazio base $M$ ed il fibrato $E$
$$
s : M \to E \quad \mathrm{t.c.}\quad \pi \circ s (p) = p \quad \forall p \in M
$$
Se la sezione è definita solo in un intorno $U$ di un punto $p$ è detta \emph{sezione locale.}


\begin{axiom}\textbf{Condizione di non trivialità:} \label{nontriv}
      Per ogni punto $p \in M$ esiste un intorno $U_i \in \mathcal{U}$ di $p$ e un isomorfismo
      $\phi_i : U_i \times F \to \pi^{-1}(U_i) \subset E$ tale che per ogni
      $(p,f) \in U_i \times F$ si ha $\pi \circ \phi_i(p,f) = p$
\end{axiom}

Occorre specificare inoltre un insieme di \emph{funzioni di transizione} $\{\Phi_{ij}\}$ che
descrivono come si trasformano le coordinate delle fibre nella regione di
sovrapposizione tra due intorni $U_i \cap U_j$. Per ogni $x \in U_i \cap U_j$ fissato si
cosidera $\phi_{i,x}$ come una mappa di $F$ in $F_x$.
Allora si definiscono le mappe
\begin{equation}\label{eq:transfunctions}
   \Phi_{ij} : F \to F \quad , \quad \Phi_{ij} = \phi_i^{-1} \circ\phi_j
\end{equation}
che rispettano le condizioni:
\begin{equation}
   \Phi_{ii} = id \quad , \quad \Phi_{ij} \circ \Phi_{jk} = \Phi_{ik}
   \quad \forall x \in U_i \cap U_j \cap U_k
\end{equation}
Grazie a queste proprietà, le funzioni di transizione formano un gruppo $G$ detto
\emph{gruppo di struttura} del fibrato, che agisce su $F$ a sinistra.\\
Gli elementi di $G$, le funzioni di transizione, sono anche detti
\emph{local trivialization}.\\

Sebbene la topologia locale del fibrato sia banale, le funzioni di transizione
possono essere fortemente influenzate dalla topologia globale a causa di torsioni
relative tra fibre adiacenti (si veda l'esempio del nastro di Möbius). Un fibrato
è completamente determinato dalle sue funzioni di transizione.\\

\begin{definition}
   Un \emph{fibrato} $E$ con fibra $F$ sullo spazio base $M$ è uno spazio topologico
   $E$ dotato di una proiezione $\pi : E \to M$ che soddisfa la condizione di
   non trivialità \ref{nontriv}.
\end{definition}

\begin{example}\emph{(Cilindro): }\label{ex:cilindro}
   Il cilindro è il primo esempio di fibrato banale, ossia la cui topologia globale
   è quella prodotto diretto $E = M\times F$. Sia lo spazio base $M=S^1$ il cerchio
   unitario, parametrizzato dall'angolo $\theta \in [0,2\pi]$ e sia $F$ il segmento
   parametrizzato da $t \in [-1,1]$. Sia $\mathcal{U} = U_+ \cup U_-$ un ricoprimento
   formato dai due intorni semicircolari:
   \begin{equation*}
      \begin{aligned}
         U_+ &= \{\theta : \epsilon < \theta < \pi + \epsilon \} \quad ,&
         U_- &= \{\theta : \pi - \epsilon < \theta < 2\pi + \epsilon = \epsilon \}
      \end{aligned}
   \end{equation*}
   Il fibrato consiste in
      $$ U_+ \times F \: , \: (\theta_+,t_+) \mathrm{\quad e \quad}
         U_- \times F \: , \: (\theta_-,t_-) $$
  e le funzioni di transizione che legano $t_+$ e $t_-$ sono definite in
  $U_+\cap U_- = A \cup B$
  $$ A = \{ -\epsilon < \theta < \epsilon \} \quad , \quad
     B = \{ \pi-\epsilon < \theta < \pi + \epsilon \} $$
  Le funzione di transizione sono:
  $$ \begin{cases}
     t_+= t_- \mathrm{\: in \:} A \\
     t_+= t_- \mathrm{\: in \:} B
  \end{cases}$$
  Si ha quindi un fibrato banale uguale al cilindro $E = S^1 \times [-1,1]$.
\end{example}

\begin{example}\emph{(Nastro di Möbius): }\label{ex:mobius}
   Con la stessa notazione dell'esempio precedente, si scelgano le funzioni di
   transizione:
   $$ \begin{cases}
      t_+= t_- \mathrm{\: in \:} A \\
      t_+= -t_- \mathrm{\: in \:} B
   \end{cases}$$
   L'identificazione di $t$ con $-t$ nella regione $B$ torce il fibrato e gli dà
   la topologia non banale del Nastro di Möbius.
\end{example}

(questa parte serve solo per far capire la notazione nella definizione \ref{def:connection1form},
si può anche togliere).\\

Un \textbf{fibrato vettore} è un fibrato la cui fibra è uno spazio vettoriale.\\

Il \textbf{fibrato cotangente} è definito analogamente al fibrato tangente.
$ TM^* := \bigcup\limits_{p \in M} T^*_p(M) $\\

Siano $E,E'$ fibrati con spazio base $M,M'$ e fibra $F,F'$ rispettivamente\\

Un \textbf{fibrato prodotto} $E \times E'$ è un fibrato con spazio base
$M \times M'$ e fibra $F \oplus F'$. Gli elementi della fibra $F \oplus F'$ sono
vettori del tipo
${f}\choose{f'}$, dove  $f \in F$ e $f' \in F'$.\\

Il \textbf{fibrato prodotto tensore} $E \otimes E'$ è un fibrato vettore in cui
a ogni punto $p \in M$ è assegnato il prodotto di fibre $F_p \otimes F'_p$.
Date due basi $\{e_\alpha\}$ ed $\{f_\beta\}$ di $F$ e $F'$, la fibra prodotto
è lo spazio generato da $\{ e_\alpha \otimes f\beta \}$.

\begin{definition}\label{def:liealgebraform} \textbf{(Forma a valori in un'algebra di Lie):}
   Una k-forma differenziale su una varietà $M$ a valori in un'algebra di Lie è
   una \emph{sezione} del fibrato $(\mathfrak{g} \times M)\otimes \Lambda^k(T^*M)$
\end{definition}
%------------------------------------------------------------------------------%
