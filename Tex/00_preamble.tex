% \renewcommand{\familydefault}{\sfdefault} % Font sans serif

\documentclass[10pt,a4paper]{report}
\usepackage{graphicx}
\usepackage[utf8]{inputenc}
\usepackage[italian]{babel}
\usepackage[T1]{fontenc}

\usepackage{amsthm}             %stili teoremi
\usepackage{mathtools}

\usepackage{amssymb,amsmath}
\usepackage[colorlinks = true,
          linkcolor = blue,
          urlcolor  = blue,
          citecolor = blue,
          anchorcolor = blue]{hyperref}
\usepackage{bookmark}
\usepackage{wrapfig}

\usepackage{geometry}
\geometry{
	top=25mm,
	bottom=25mm,
	left=30mm,
	right=25mm
}
\usepackage[ampersand]{easylist}
\usepackage[nottoc]{tocbibind} %bibliografia
%------------------------------------------------------------------------------%

% Stile + numerazione teoremi, definizinoni, etc...
\theoremstyle{plain}
\newtheorem{theorem}{Teorema}[section]

\theoremstyle{definition}
\newtheorem{definition}{Definizione}[section]

\theoremstyle{plain}
\newtheorem{proposition}{Proposizione}[section]

\theoremstyle{plain}
\newtheorem{lemma}{Lemma}[section]

\theoremstyle{remark}
\newtheorem{example}{Esempio}[section]

\theoremstyle{definition}
\newtheorem{axiom}{}[section]

\renewcommand\qedsymbol{$\blacksquare$} % quadrato della dimostrazione
%------------------------------------------------------------------------------%
\renewcommand{\vec}[1]{\mathbf{#1}} % simbolo di vettore = grassetto
\newcommand{\chrsym}{\genfrac{\{}{\}}{0pt}{}} % Simboli Christoffel
%------------------------------------------------------------------------------%
\graphicspath{{/home/dan/Desktop/UNI/TESI/Images/}}	%Default path for graphics
%------------------------------------------------------------------------------%
% Remove default parindent
\newlength\tindent
\setlength{\tindent}{\parindent}
\setlength{\parindent}{0pt}
\renewcommand{\indent}{\hspace*{\tindent}}

\newcommand{\tab}[1][1cm]{\hspace*{#1}} % ridefinisce la tabulazione
\newcommand{\ttab}[1][2cm]{\hspace*{#1}} % ridefinisce la tabulazione
\newcommand{\dd}{\mathrm{d}} % dx negli integrali
\newcommand{\tc}{\mathrm{\: t.c. \:}} % tale che
\newcommand{\R}{\mathbb{R}} % insieme dei reali
\newcommand{\C}{\mathbb{C}} % insieme dei complessi
\newcommand{\Z}{\mathbb{Z}} % insieme degli interi
\newcommand{\Ll}{\mathcal{L}} % lagrangiana
\newcommand\Chapter[2]{
  \chapter*[#1: {\itshape#2}]{#1\\[2ex]\Large\itshape#2}
} %sottotitolo al capitolo

%------------------------------------------------------------------------------%
% Metadati
\hypersetup{
	pdftitle={Tesi},%
	pdfauthor={Danilo Bondì},%
	pdfsubject={Aspetti classici e quantistici dei monopoli magnetici in teorie di gauge},%
	pdfkeywords={},%
	colorlinks=true,%
	linkcolor=blue,%
	linktocpage=true,%
	pageanchor=true
}
