\chapter{Monopolo di Dirac}\label{cap:diracmonopole}
%------------------------------------------------------------------------%
\section{Potenziale di Dirac}
Si vuole dare un primo approccio classico al monopolo magnetico.
La prima idea è di sostituire la carica elettrica con un'ipotetica carica magnetica
$g$ e studiare il problema come in elettrostatica.\\
Il campo magnetico generato da una carica magnetica $g$ situata nell'origine è
%
\begin{equation}\label{eq:monopolefield}
   B = g \frac{\vec r}{r^3} = \frac{g}{r^2} \vec u _r
\end{equation}
%
dove $\vec r = (x,y,z)$ , $r = \sqrt{x^2 + y^2 + z^2}$ e $\vec u _r = \frac{\vec r}{r}$
il versore radiale.\\
Le equazioni del moto di una carica elettrica $e$ nel campo $\vec B$,
generate dalla Lagrangiana di minima interazione tra una carica elettrica e un campo
magnetico
%
$$
   L = \frac{1}{2} m \dot{\vec r} ^2 + e \dot{\vec r} \cdot \vec A
$$
%
sono
%
$$
   m \ddot{\vec r} = e\dot{\vec r} \times \vec B =
      \frac{eg}{r^3}(\dot{\vec r} \times \vec r)
$$
%
dove il vettore potenziale che compare nella Lagrangiana deve soddisfare
$\vec B = \nabla \times \vec A = g \frac{\vec r}{r^3}$.\\

Come calcolato nella sezione \ref{sec:dirac_potential} un potenziale che soddisfa
questa relazione è:

\begin{equation}\label{eq:diracpotential}
  \vec A = g(1 + \cos\theta)\left( \frac{\sin\varphi}{r\sin\theta},
     \frac{-\cos\varphi}{r\sin\theta},0 \right) = - g (1+cos\theta) \nabla \varphi
\end{equation}

Si presenta immediatamente il seguente problema. Se si ammette l'esistenza di una carica
magnetica, il flusso del campo magnetico attraverso una qualsiasi supericie chiusa
contenente la carica $g$ è pari a $4\pi g$ e, applicando il teorema di Gauss,
si ottiene $\nabla \cdot \vec B = 4\pi g\delta^{(3)}(\vec r)$.
Ciò è in contraddizione con $\nabla \cdot \vec B = \nabla \cdot
(\nabla \times \vec A) = 0$, che genera un flusso nullo.

\begin{equation}\label{eq:B4gpi}
   \int_S \dd\sigma \vec B = \int_V d^3 x \, \nabla \cdot \vec B = 4g\pi
\end{equation}
\begin{equation}
   \int_S d\sigma  \vec B = \int_V d^3 x \, \nabla \cdot \vec B
      = \int_V d^3 x \, \nabla \cdot (\nabla \times \vec A) = 0
\end{equation}

Il potenziale $\vec A$ presenta però delle discontinuità per $\sin\theta = 0$.
Si verifica immediatamente che $\theta = 0$ è un punto singolare, mentre $\vec A$
è continuo in $\theta = \pi$. Il calcolo fatto non è quindi corretto lungo l'asse z
positivo ($\theta = 0$), e occorre regolarizzare il potenziale.\\
Siano $\varepsilon > 0$  e $r_\varepsilon = \sqrt{r^2 + \varepsilon^2}$.
Si definiscono ora\footnote{Si veda \cite{shnir} per la trattazione completa.}
%
\begin{align*}
   \vec A_\varepsilon &:= \frac{g}{r_\varepsilon}\frac{1}{r_\varepsilon - z} \vec u _\varphi  \\
   \vec B _\varepsilon & = \nabla \times \vec A_\varepsilon = \frac{g}{r_\varepsilon^3}\vec r
      - g\varepsilon^2 \left( \frac{1}{r_\varepsilon^3(r_\varepsilon - z)}
         + \frac{1}{r_\varepsilon^2(r_\varepsilon-z)^2} \right) \Theta(z) \vec u _z
\end{align*}
%
dove $\vec u _\varphi $ è il versore dell'angolo azimutale come nell'appendice
\ref{sec:notation} e $\Theta(z)$ è la funzione di Heaviside per l'asse z.\\
Si vuole definire $\tilde{\vec B} := \lim_{\varepsilon \to 0} \vec B _\varepsilon$
%
$$
   \lim_{\varepsilon \to 0} \vec B_\varepsilon = \lim_{\varepsilon \to 0} \left[
      \frac{g}{r^3}\vec r - 2g\varepsilon^2 \left( \frac{1}{r^2(x^2 + y^2 + \varepsilon^2)}
            + \frac{2}{(x^2 + y^2 + \varepsilon^2)^2} \right) \Theta(z) \vec u _z \right]
$$
%

Si vuole ora valutare il flusso del campo magnetico regolarizzato $\tilde{\vec B}$
attraverso una superficie chiusa $S$ centrata nell'origine e di raggio unitario, supponendo
che sia possibile scambiare il limite e l'integrale
%
$$
  \int_S \mathrm{d}{\sigma} \lim_{\varepsilon \to 0} \vec B_\varepsilon = \lim_{\varepsilon \to 0} \int_S \mathrm{d}\sigma \vec B_\varepsilon
$$
%
Si veda la sezione \ref{sec:flusso_regolarizzato} per il calcolo esplicito.
L'unico termine in $\varepsilon$ che porta contributo al flusso è il secondo e si
ha:
$$
   \int_S \dd\sigma \tilde{\vec B}
      = \int_S \dd \sigma \left( \frac{g}{r^3}\vec r
         - 4\pi g \delta(x)\delta(y)\Theta(z) \vec u _z \right)
      = 4\pi g - 4\pi g = 0
$$
%
Il campo regolarizzato è composto da due termini:
$$
   \tilde{\vec B} = \vec B + \vec B _{string} = \frac{g}{r^3}\vec r
      - 4\pi g \delta(x)\delta(y)\Theta(z) \vec u _z
$$
L'effetto del campo generato dalla stringa di singolarità lungo l'asse z positivo
è quello di annullare il flusso del campo prodotto dalla carica, risolvendo la
contraddizione evidenziata in precedenza.

%------------------------------------------------------------------------------%

\section{Trasformazione della stringa}\label{sec:gaugestring}
Come evidenziato nella sezione precedente, non è possibile definire ovunque un
potenziale di monopolo continuo e derivabile. L'effetto del potenziale singolare
\ref{eq:diracpotential} conduce a un termine extra di campo della stringa,
diretto lungo la singolarità.
Per rimuovere questo termine singolare, si vuole far sì che non corrisponda a una
configurazione fisica.\\
Naturalmente la scelta del sistema di coordinate per la descrizione del monopolo
è arbitraria, quindi il primo approccio è di richiedere che tutte le configurazioni
possibili della stringa ($\theta$) siano fisicamente equivalenti.
Ossia, si vogliono trovare le trasformazioni tra le varie configurazioni di stringa,
e le condizioni che rendano tali configurazioni identiche.\\

Si ricorda che l'elettrodinamica è invariante per trasformazioni del potenziale
vettore del tipo\footnote{$e =$ carica dell'elettrone}
\begin{equation}
   \vec A \mapsto \vec A' = A + \nabla \lambda (\vec r) = A - \frac{i}{e}
    U^{-1}\nabla U
\end{equation}
dove $U(\vec r) = \exp(ie\lambda(\vec r))$ e $\lambda$ è un'arbitraria funzione
delle coordinate. Una trasformazione di questo tipo è detta
\textbf{trasformazione di gauge}  del gruppo $U(1)$, in quanto $U$ appartiene
  al gruppo $U(1)$ delle matrici unitarie $1 \times 1$ a coefficienti complessi
  (semplicemente, la moltiplicazione per un fattore di fase).\\

Si vuole trovare la trasformazione di gauge che risolve la singolarità.\\

Si consideri il flusso del campo magnetico attraverso una superficie chiusa $S$
prima e dopo la trasformazione di gauge e si valuti quale è la sua variazione
$$
   \int_S \dd\sigma B' - \int_S \dd\sigma B = \int_S \dd\sigma (\nabla \times \nabla \lambda)
      = \oint_{\partial S} \dd\vec l \cdot \nabla \lambda
$$
Il flusso non cambia solo se la funzione $\nabla \lambda$ è periodica $\nabla
\lambda(\varphi) = \nabla \lambda( \varphi + 2\pi)$. Si consideri allora la trasformazione
$$
\lambda(\vec r) = 2g\varphi = 2g\arctan\left(\frac{y}{x}\right) \to U = \exp(2ieg\varphi)
$$
\begin{equation}\label{eq:gaugetrasf}
   \vec A ^+ = \vec A ^- + \nabla(2g \varphi) = \vec A ^- + 2g\nabla\varphi
\end{equation}
Si nomina $\vec A^-$ il potenziale di dirac trovato in precedenza (in quanto ben
definito per $z$ negative)
$$
   \vec A^- = \frac{g}{r}\frac{-1 - \cos\theta}{\sin\theta} \vec u _\varphi  \\
$$
$$
  \vec A ^- \mapsto \vec A ^- - \frac{i}{e} \exp( -2ieg\varphi )\nabla \exp(2ieg\varphi ) =
     -\frac{g}{r}\frac{1+\cos\theta}{\sin\theta}\vec u _\varphi
     + \frac{2g}{r\sin\theta}\vec u _\varphi
     = \frac{g}{r}\frac{1 - \cos\theta}{\sin\theta} \vec u _\varphi
     =: \vec A ^+
$$
Il potenziale trasformato è singolare per $\theta = \pi$ e regolare per
$\theta =  0$ quindi, analogamente a quanto visto per $\vec A^-$ produce un termine
di campo di stringa lungo l'asse $z$ negativo.\\
La trasformazione di gauge scelta agisce come una rotazione della stringa
di un angolo $\theta = \pi$, quindi il campo generato in questo modo non è
considerato fisico (?).\\

Si sottolinea che l'approccio di regolarizzazione dei potenziali singolari
utilizzato in questo capitolo è non rigoroso e serve solo per dare un'introduzone
al problema. Lo si potrebbe rendere rigoroso trattandolo in teoria delle
distribuzioni, che esula dallo scopo di questo elaborato. Si seguirà quindi
un approccio differente nelle sezioni successive

Si riporta l'espressione dei due potenziali di dirac definiti in precedenza, che
sarà utile nella trattazione successiva.
\begin{equation}\label{eq:localdiracpotential}
  \vec A^\pm = \frac{g}{r}\frac{\pm 1 - \cos\theta}{\sin\theta} \vec u _\varphi
\end{equation}

%------------------------------------------------------------------------------%

\section{Condizione di quantizzazione della carica}
Si vuole studiare il moto di una una particella di massa $m$ e di cariche elettrica $e$ e magnetica
$g$ in interazione con un campo di monopolo magnetico, descritto dal potenziale
$\vec A$. Sia $\psi$ la funzione d'onda della particella. Essa è soggetta
all'equazione di Schrödinger, in unità naturali:
\begin{equation}
\frac{1}{2m}\left( -i\nabla - {e} \vec{\hat{A}} \right)^2 \psi(\vec r) = E\psi(\vec r)
\end{equation}

Si ricorda che le funzioni d'onda sono sempre definite a meno di un fattore di fase,
che può in generale dipendere dalle coordinate $(t,\vec x)$. La differenza di fase
tra due punti distinti non è univocamente definita e dipende dal cammino $\gamma$
scelto tra i due punti.
Sia $S = \int_0^t \Ll \, \dd t'$ l'azione tra due punti, si assuma per semplicità
di calcolarla tra l'origine $(0,\vec 0)$ e un punto $(t, \vec x)$, suppondendo
che non contengano singolarità del potenziale. Dirac \ref{dirac}
arriva a scrivere la funzione d'onda della particella come
%
$$
   \psi (\vec x) = \psi_0(\vec x) e^{iS}
$$
%
dove $\psi_0$ è soluzione dell'equazione di Schrödinger e deve necessariamente
essere continua, ma la fase può essere discontinua.\\

In Meccanica Classica l'equazione del moto è invariante per trasformazione di gauge
$U = e^{ie \lambda(\vec x) = e^{i2eg\varphi}}$,
ma in Meccanica Quantistica è l'azione $S$, che non è gauge invariante, a produrre
l'equazione del moto. Una variazione $\delta S$ dell'azione per trasformazione di gauge porta
quindi a una variazione di fase della funzione d'onda di un fattore $e^{i\delta S}$,
che può appunto essere discontinuo. Affinchè $\psi$ in ogni punto sia
ben definita, non si può avere discontinuità di fase della funzione d'onda per
cammini chiusi $\gamma$ prodotti dalla trasformazione di gauge.
Per curve chiuse, per cui si ha
$$
   \delta S = 2eg \delta\varphi = 4\pi e g \, ,
$$
è necessario che la discontinuità della fase sia multipla di $2\pi$, ossia per $n \in \Z$
si deve avere $\delta S = 2\pi n$. Si ottiene allora la condizione:

\begin{equation}\label{eq:diracquantumcharge}
   \boxed{
      eg = \frac{n}{2} \quad, \,  n \in \Z
   }
\end{equation}

Questa condizione è la celebre \textbf{condizione di quantizazione della carica
di Dirac}, da cui risulta che se esiste una carica magnetica di monopolo $g$,
la carica elettrica è quantizzata. Poichè si osserva che la carica elettrica è
quantizzata, questa condizione costituirebbe uno spunto di ricerca dei monopoli
magnetici\footnote{
   Esistono altre teorie che spiegano equivalentemente la quantizzazione della carica
   elettrica, ma non verranno qui indagate.\\
}.\\

Si sottolinea infine che la condizione \ref{eq:diracquantumcharge} di quantizzazione
è solamente di natura topologica: non deriva dallo spettro di un operatore hermitiano,
come accade ad esempio per l'oscillatore armonico o altri sistemi quantistici semplici.
La natura topologica di tale condizione verrà chiarita nelle sezioni successive.
