\section{Tensori e Forme differenziali}
Sia $V$ uno spazio vettoriale su su $\mathbb{R}$.\\
Si definisce \textbf{tensore} misto con $r$ indici covarianti e $s$ indici
controvarianti un'\emph{applicazione multilineare} $F : V^r \times V^{*s} \to
\mathbb{R}$ , dove $V^r = V \times \dots \times V$ e
$V^{*s} =  V^* \times \dots \times V^*$.\\

Si indica con $\mathcal{T}^{(r,s)}(V)$ lo spazio dei (r,s)-tensori su $V$\\

Si definisce il \textbf{prodotto tensoriale} (indicato col simbolo $\otimes$) tra
$F \in \mathcal{T}^r$(V) e $G \in \mathcal{T}^s(V)$ il $r+s$ tensore definito da:
$$ F \otimes G (v_1,\dots,v_{r+s}) = F(v_1,\dots,v_r)G(v_{r+1},\dots,v_{r+s})$$
dove a destra dell'uguale si ha il prodotto tra i due numeri reali
$F(v_1,\dots,v_r)$ e $G(v_{r+1},\dots,v_{r+s})$.\\
Analogamente, si definisce il prodotto tensoriale tra due tensori controvarianti.\\

Si può facilmente dimostrare la seguente
\begin{proposition}\label{prop:base1}
   Sia $M$ varietà differenziale e $p \in M$. Siano $V = T_p(M)$ e $V^* = T_p^*(M)$
   Sia $F \in \mathcal{T}^{(r,s)}(V)$, allora esistono $a_{i_1,\dots i_r}^{j_1,\dots,j_s}$
   tali che in coordinate locali\\
   $$
      F = \sum_{\substack{(i_1,\dots, i_r)\\(j_1,\dots,j_s)}}
         a_{i_1,\dots, i_r}^{j_1,\dots,j_s}
         \frac{\partial}{\partial x^{i_1}} \otimes \dots
         \otimes \frac{\partial}{\partial x^{i_r}}
         \otimes dx^{j_1} \otimes \dots \otimes dx^{j_s}
   $$
\end{proposition}
Ovvero le basi di $T_p(M)$ e $T_p^*(M)$ inducono una base per $\mathcal{T}^r(T_p(M))$
e $\mathcal{T}^s(T_p^*(M))$.\\

Si definisce \textbf{campo tensoriale} sulla varietà $M$ con $r$ indici covarianti
e $s$ indici controvarianti un'applicazione $ M \to \mathcal{T}^{(r,s)}(T_p^*(M)) $
che a ogni $p$ associa un tensore con punto base $p$. Si vuole richiedere anche
una dipendenza continua o liscia da $p$.\\

Si definice \textbf{r-forma} un r-tensore covariante \emph{totalmente antisimmetrico}.\\

Lo spazio delle r-forme sulla varietà $M$ nel punto $p$ si indica con $\Lambda^r(p)$.
La dimensione di $\Lambda^r(p)$ è ${n}\choose{k}$ se $n$ è la dimensione di $M$
\footnote{ Si veda \cite{sernesi},\cite{boothby},\cite{nakahara} }.\\
%------------------------------------------------------------------------------%
\subsection{Forme differenziali}
\begin{definition}{(Forma differenziale)}
   Si definice \emph{r-forma differenziale} un campo tensoriale covariante
  totalmente antisimmetrico, ossia un'applicazione multilineare
  $\omega : M \to \Lambda^r(p)$ che a $p \in M$ associa la $r$-forma $\omega_p$
\end{definition}
Lo spazio delle r-forme differenziali su $M$ si indica con $\Omega^r(M)$.\\

Analogamente a quanto fatto per il prodotto tensoriale, si vuole definire un prodotto
tra $r$ e $s$ forme differenziali che dia una $r+s$ forma differenziale (cioè un
prodotto tensoriale che mantenga l'antisimmetria del tensore).\\

Si definisce \textbf{prodotto esterno} o \textbf{prodotto wedge} tra due forme
differenziali $\alpha \in \Omega^r(M)$ e $\omega \in \Omega^s(M)$ la $r+s$ forma
differenziale definita da:
$$ \alpha \wedge \omega (V_1,\dots,V_{r+s})(p) =
   \frac{1}{r!s!}\sum_{\sigma \in S^{r+s} } (-1)^\sigma
   \alpha_p\otimes\omega_p (V_{\sigma(1)}, \dots , V_{\sigma(r),
      V_{\sigma(r+1)}, \dots , V_,\sigma(r+s)})$$
Dove $S^{r+s}$ è il gruppo delle permutazioni di $r+s$ elementi e $(-1)^\sigma$
è il segno della permutazione $\sigma$.\\
E gode delle seguenti proprietà, di immediata dimostrazione:
\begin{enumerate}
    \item $(\alpha + \beta) \wedge \omega = \alpha\wedge\omega + \beta\wedge\omega$ e
          $\omega \wedge(\alpha + \beta)  = \omega\wedge\alpha + \omega\wedge\beta$
    \item $(c\alpha)\wedge\omega = c(\alpha\wedge\omega) = \alpha\wedge(c\omega)$
    \item $\omega\wedge\alpha = (-1)^{r+s} \alpha\wedge\omega$
    \item $(\alpha\wedge\omega)\wedge\tau = \alpha\wedge(\omega\wedge\tau)$
       (valida grazie alla normalizzazione scelta nella definizione di $\wedge$)
\end{enumerate}
$\forall c\in \mathbb{R} \: , \: \forall \alpha,\beta \in \Omega^r(M) \: , \:
   \forall \omega \in \Omega^s(M) \: , \: \forall \tau \in \Omega^k(M)$
\\

Analogamente alla \ref{prop:base1}, vale anche
\begin{proposition}\label{prop:base2}
   Sia $M$ varietà differenziale e $p \in M$. Sia $\omega \in \Omega^r(T_p(M))$,
   allora esistono le funzioni $a_{i_1,\dots i_r}$ tali che in coordinate locali\\
   $$
      \omega = \sum_{(i_1,\dots, i_r)} a_{i_1,\dots, i_r}
         dx^{i_1} \wedge \dots \wedge dx^{i_r}
   $$
\end{proposition}

Si osservi che per la proprietà di antisimmetria se $u = v$ si ha
$$\omega(v,\dots,u,\dots) = - \omega(u,\dots,v,\dots) = - \omega(v,\dots,u,\dots) = 0$$
Per una varietà di dimensione $n$ si ha al massimo $n$ vettori linearmente indipendenti.
Se si prende in considerazione un vettore aggiuntivo, esso è combinazione lineare dei precedenti.\\
Di conseguenza \textbf{tutte le (r$>$n)-forme} su una varietà di dimensione $n$ sono \textbf{nulle}.\\
Se $u= a_1v^1 + \dots + a_nv^n$
$$
   \omega(v^1,\dots,v^n,u) = \dots = a_1\omega(v^1,\dots,v^n,v^1) + \dots
      + a_n\omega(v^1,\dots,v^n,v^n) = 0
$$
%------------------------------------------------------------------------------%
%------------------------------------------------------------------------------%
\subsection{Differenziale esterno}
Si definisce l'operatore che a una r-forma
$\omega \in \Lambda^r(p)$ associa la (r+1)-forma $d\omega \in \Lambda^{r+1}(p)$,
definita in coordinate locali da\\
$$
   \omega = \sum_{(i_1,\dots, i_r)} a_{i_1,\dots, i_r}
      dx^{i_1} \wedge \dots \wedge dx^{i_r} \to
   d\omega = \sum_{(i_1,\dots, i_r)} da_{i_1,\dots, i_r}\wedge
      dx^{i_1} \wedge \dots \wedge dx^{i_r}
$$
ossia
$$
   d : \Lambda^r(p) \to \Lambda^{r+1}(p) \quad \omega \mapsto
   d\omega = \sum_{(i_1,\dots, i_r,k)} \frac {\partial a_{i_1,\dots, i_r}}{\partial x^k} dx^k
   \wedge dx^{i_1} \wedge \dots \wedge dx^{i_r}
$$
Dall'antisimmetria del prodotto wedge discende immediatamente che $d^2\omega = d(d\omega) = 0$
\begin{equation*}
   \begin{split}
      d^2\omega & = d \left( \sum_{(i_1,\dots, i_r,k)}
            \frac {\partial a_{i_1,\dots, i_r}}{\partial x^k} dx^k
               \wedge dx^{i_1} \wedge \dots \wedge dx^{i_r} \right)
          = \sum_{(i_1,\dots, i_r,k)}
            d\left( \frac {\partial a_{i_1,\dots, i_r}}{\partial x^k} \right)
               \wedge dx^k \wedge dx^{i_1} \wedge \dots \wedge dx^{i_r} \\
       & = \sum_{(i_1,\dots, i_r,k,j)}
            \frac {\partial^2 a_{i_1,\dots, i_r}}{\partial x^k \partial x^j}
               \wedge dx^j \wedge dx^k \wedge dx^{i_1} \wedge \dots \wedge dx^{i_r}
         = 0
   \end{split}
\end{equation*}
in quanto contrazione del termine simmetrico $\frac {\partial^2 a_{i_1,\dots, i_r}}
{\partial x^k \partial x^j}$ e del termine antisimmetrico $dx^j \wedge dx^k $\\

Si definisce forma \textbf{esatta} una r-forma $\omega$ se esiste una
(r-1)-forma $\alpha$ che verifica $\omega = d\alpha$\\
Si definisce forma \textbf{chiusa} una r-forma $\omega$ tale che $d\omega = 0$.\\
Segue immediatamente che ogni forma esatta è chiusa. L'inverso è vero solo localmente.
\begin{lemma}{(di Poincarè)}
    Sia $M \subset \mathbb{R}^n$ una palla aperta. Una r-forma $\omega$ chiusa
    definita su $M$ è esatta.
\end{lemma}
%------------------------------------------------------------------------------%
\subsection{Coomologia di de Rham}
\textcolor{red}{(da scrivere)}
%------------------------------------------------------------------------------%
\subsection{Integrazione}
L'integrazione di una forma su una varietà può essere definita
tramite coordinate locali e ricondotta a integrazione su aperti di $\mathbb{R}^n$.
Una trattazione rigorosa richiederebbe l'introduzione del concetto di
\emph{partizione dell'unità}, che esula dallo scopo di questo elaborato.
Si vedano \cite{boothby},\cite{nakahara} per una trattazione rigorosa. \\
In maniera intuitiva, sia $(U,\phi)$ una carta della
varietà $M$, dove $U$ intorno del punto $p \in M$, e $R = \phi(U)$ e
$\omega = dx^1 \wedge \dots \wedge dx^n$ una n-forma.
Sia $f : U \to \mathbb{R}$ una funzione integranda.
Si definisce automaticamente la misura di integrazione $d\mu = dx^1 \dots dx^n$.
e si può dare senso all'espressione in coordinate locali:
$$
   \int_U f\omega = \int_R f(x_1,\dots,x_n) dx^1 \dots dx ^n
$$
Si vuole poi ripetere questa operazione per tutte le carte di un atlante dato e
"incollare" assieme i risultati in maniera che l'integrale sulla varietà sia uguale
alla somma degli integrali sulle singole carte.\\

Si osserva che è possibile integrare solamente n-forme su varietà di dimensione $n$
perchè le forme con $r>n$ sono tutte nulle, e gli spazi di dimensione $r<n$ hanno
misura nulla in $\mathbb{R}^n$.\\

\textbf{Hodge star}:\\
\textbf{Teorema di Stokes}:\\
%------------------------------------------------------------------------------%
%------------------------------------------------------------------------------%
\subsection{Varietà Riemanniane}
\begin{definition}
   Una \emph{metrica Riemanniana} $g$ su una varietà $M$ è un (2,0)-campo tensoriale
   su $M$ che per ogni punto $p \in M$ soddisfa:
   \begin{enumerate}
      \item $ g_p(U,V) = g_p(V,U) $
      \item $ g_p(U,U) \geq 0 $ dove $ g(U,U) = 0 \iff U = 0$
   \end{enumerate}
   dove $U,V \in T_p(M)$.\\

   Una \emph{metrica pseudo-Riemanniana} $g$ su una varietà $M$ è un
   (2,0)-campo tensoriale su $M$ che per ogni punto $p \in M$ soddisfa:
   \begin{enumerate}
      \item $ g_p(U,V) = g_p(V,U) $
      \item $ g_p(U,V) = 0 \: \forall U \in T_p(M) \Rightarrow V = 0$
   \end{enumerate}
\end{definition}

Data una carta $(U,\phi)$ di $M$ con coordinate $\{x^\mu\}$ il tensore $g$ può
essere scritto come
$$ g_p = g_{\mu\nu}(p) dx^\mu dx^\nu =: ds^2$$
Dove $g_{\mu\nu}(p)$ può essere considerato come la $\mu\nu$-esima entrata di una matrice.\\
Il numero $(p,n)$ di autovalori positivi $p$ e negativi $n$ è detto \emph{indice} della
metrica. Se $n=1$ la metrica è detta Lorentziana.\\
Si può diagonalizzare la metrica e riscalare gli autovettori in modo da ottenere
solamente $\pm 1$ sulla diagonale. Ad esempio si ha la metrica Euclidea
$\delta = diag(1,1,\dots,1)$ o Minkowskiana $\eta = diag(-1,1,...,1)$.\\

Se una varietà differenziale $M$ è dotata di una metrica Riemanniana $g$, la coppia $(M,g)$
è detta \textbf{varietà Riemanianna} (analogamente se $g$ è pseudo-Riemann).\\

%------------------------------------------------------------------------------%
%------------------------------------------------------------------------------%
\subsection{Derivata covariante}
\textcolor{red}{Cosa è, Cosa è in questo contesto. (piccola intro)}\\
Si vuole estendere il concetto di derivata direzionale agli (r,s)-tensori definiti
su una varietà differenziale $M$\\

Sia $X$ un campo vettoriale sulla varietà $M$ (supponiamo $M = \mathbb{R}^n$,
per semplicità), $p \in M$ e $h \in M$ pensato come piccolo spostamento da $p$, in $M$.
Volendo definire la derivata direzionale di un campo vettoriale nel modo usuale
$$
   \lim_{|h| \to 0} \frac{X_{p+h}-X_p}{|h|}(f)
$$
si riscontra subito un problema. I due vettori $X_{p+h} \in T_{p+h}(M)$ e
$X_p \in T_p(M)$ appartengono a due spazi differenti e non possono essere confontati.\\
Occorre un modo di trasportare il vettore $X_{p+h}$ da $T_{p+h}(M)$ a $T_p(M)$
lasciandolo inalterato. Questo processo è chiamato \textbf{trasporto parallelo}.
Purtroppo non esiste una maniera univoca per trasportare un vettore tangente in
una varietà, quindi è necessario specificare come viene effettuato il trasporto
parallelo.\\

\textcolor{red}{(qua si può tagliare o riassumere ulteriormente)}
\begin{definition}{(Connessione affine):}\label{def:affineconnection}
   Una connessione affine $\nabla$ è una mappa $\nabla : \mathcal{X}(M) \times
   \mathcal{X}(M)\to \mathcal{X}(M)$ t.c $(X,Y) \mapsto \nabla_X Y$ che verifica
   \begin{enumerate}
      \item $ \nabla_X (Y+Z) = \nabla_X Y + \nabla_X Z $
      \item $ \nabla_{X+Y} Z = \nabla_X Z + \nabla_Y Z $
      \item $ \nabla_{fX} Y = f\nabla_X Y $
      \item $ \nabla_X (fY) = X[f] Y + f\nabla_X Y $
   \end{enumerate}
   per $f \in C^\infty(M)$ e $X,Y,Z \in \mathcal{X}(M)$
\end{definition}

Sia $(U,\phi)$ una carta di $M$ con coordinate $x = \phi(p) \:,\: p\in M$.
L'azione di $\nabla$ sugli elementi $\{ e_\mu = \partial/ \partial x^\mu \}$
della base di $T_p(M)$ ne determina univocamente l'azione su qualsiasi vettore $X_p$.\\

Si definiscono i \textbf{coefficienti di connessione} $\Gamma^\lambda_{\nu\mu}$ da
(per semplicità di notazione si indica $\nabla_{e_\nu} = \nabla_\nu$):
$$
   \nabla (e_\nu,e_\mu) = \nabla_\nu e_\mu =
      e_\lambda \Gamma^\lambda_{\nu\mu}
$$
Presi allora due campi $X = X^\mu e_\mu \: , \: Y = Y^\nu e_\nu \:
\in \mathcal{X}(M)$ si ha:
$$
   \nabla_X Y
      = X^\mu \nabla_\mu(Y^\nu e_\nu)
      = X^\mu(e_\mu[Y^\nu]e_\nu + Y^\nu \nabla_\mu e_\nu)
      = X^\mu \left(\frac{\partial Y^\lambda }{\partial x^\mu}
         + Y^\nu \Gamma^\lambda_{\nu\mu} \right)
$$
il risultato dipende solo dalle $(\mathrm{dim} M)^3$ funzioni $\Gamma^\lambda_{\nu\mu}$.\\

Una connessione $\nabla$ è dette simmetrica se in coordinate vale
$\Gamma^\lambda_{\nu\mu} = \Gamma^\lambda_{\mu\nu}$

\begin{theorem}{(Teorema fondamentale della geometria (pseudo-)Riemanniana):}
  Sia $(M,g)$ una varietà (pseudo-)Riemanniana. Esiste un unica connessione
  \emph{simmetrica} che è \emph{compatibile} con la metrica $g$. Questa connessione
  è chiamata \textbf{connessione di Levi-Civita} definita da
  $$
       \Gamma^\kappa_{\mu\nu} = \frac{1}{2} g^{\kappa\lambda}
           (\partial_\mu g_{\lambda\nu}  + \partial_\nu g_{\lambda\mu}
              - \partial_\lambda g_{\mu\nu}) = : \chrsym{\kappa}{\mu\nu}
  $$
\end{theorem}
$\chrsym{\kappa}{\mu\nu}$ è detto simbolo di Christoffel.
Si veda \cite{nakahara} per la dimostrazione del teorema.\\
