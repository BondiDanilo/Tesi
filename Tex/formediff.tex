\section{Tensori e Forme differenziali}
Sia $V$ uno spazio vettoriale su su $\R$.
Si definisce \textbf{tensore} misto con $r$ indici covarianti e $s$ indici
controvarianti un'\emph{applicazione multilineare} $F : V^r \times V^{*s} \to
\R$ , dove $V^r = V \times \dots \times V$ e
$V^{*s} =  V^* \times \dots \times V^*$. 
Si indica con $\mathcal{T}^{(r,s)}(V)$ lo spazio dei (r,s)-tensori su $V$\\

Si definisce il \textbf{prodotto tensoriale} (indicato col simbolo $\otimes$) tra
$F \in \mathcal{T}^r$(V) e $G \in \mathcal{T}^s(V)$ il $r+s$ tensore definito da:
$$ F \otimes G (v_1,\dots,v_{r+s}) = F(v_1,\dots,v_r)G(v_{r+1},\dots,v_{r+s})$$
dove a destra dell'uguale si ha il prodotto tra i due numeri reali
$F(v_1,\dots,v_r)$ e $G(v_{r+1},\dots,v_{r+s})$.\\
Analogamente, si definisce il prodotto tensoriale tra due tensori controvarianti.\\

Si definisce \textbf{campo tensoriale} sulla varietà $M$ con $r$ indici covarianti
e $s$ indici controvarianti un'applicazione $ M \to \mathcal{T}^{(r,s)}(T_p^*(M)) $
che a ogni $p$ associa un tensore con punto base $p$. Si vuole richiedere anche
una dipendenza continua o liscia da $p$.\\

Si definice \textbf{r-forma} un r-tensore covariante \emph{totalmente antisimmetrico}.\\

Lo spazio delle r-forme sulla varietà $M$ nel punto $p$ si indica con $\Lambda^r(p)$.
La dimensione di $\Lambda^r(p)$ è ${n}\choose{k}$ se $n$ è la dimensione di $M$
\footnote{ Si veda \cite{sernesi},\cite{boothby},\cite{nakahara} }.
%------------------------------------------------------------------------------%
\subsection{Forme differenziali}
\begin{definition}{(Forma differenziale)}
   Si definice \emph{r-forma differenziale} un campo tensoriale covariante
  totalmente antisimmetrico, ossia un'applicazione multilineare
  $\omega : M \to \Lambda^r(p)$ che a $p \in M$ associa la $r$-forma $\omega_p$.
  Lo spazio delle r-forme differenziali su $M$ si indica con $\Omega^r(M)$.
\end{definition}

Si osservi che per la proprietà di antisimmetria se $u = v$ si ha
$$\omega(v,\dots,u,\dots) = - \omega(u,\dots,v,\dots) = - \omega(v,\dots,u,\dots) = 0$$
Per una varietà di dimensione $n$ si ha al massimo $n$ vettori linearmente indipendenti.
Se si prende in considerazione un vettore aggiuntivo, esso è combinazione lineare dei precedenti.\\
Di conseguenza \textbf{tutte le (r$>$n)-forme} su una varietà di dimensione $n$ sono \textbf{nulle}.\\
Se $u= a_1v^1 + \dots + a_nv^n$
$$
   \omega(v^1,\dots,v^n,u) = \dots = a_1\omega(v^1,\dots,v^n,v^1) + \dots
      + a_n\omega(v^1,\dots,v^n,v^n) = 0
$$

Analogamente a quanto fatto per il prodotto tensoriale, si vuole definire un prodotto
tra $r$ e $s$ forme differenziali che dia una $r+s$ forma differenziale (cioè un
prodotto tensoriale che mantenga l'antisimmetria del tensore).\\

Si definisce \textbf{prodotto esterno} o \textbf{prodotto wedge} tra due forme
differenziali $\alpha \in \Omega^r(M)$ e $\omega \in \Omega^s(M)$ la $r+s$ forma
differenziale definita da:
$$
   \alpha \wedge \omega (V_1,\dots,V_{r+s})(p) =
   \frac{1}{r!s!}\sum_{\sigma \in S^{r+s} } (-1)^\sigma
   \alpha_p\otimes\omega_p (V_{\sigma(1)}, \dots , V_{\sigma(r),
      V_{\sigma(r+1)}, \dots , V_,\sigma(r+s)})
$$

Dove $S^{r+s}$ è il gruppo delle permutazioni di $r+s$ elementi e $(-1)^\sigma$
è il segno della permutazione $\sigma$.\\

Il prodotto wedge gode delle seguenti proprietà, di immediata dimostrazione:

\begin{enumerate}
    \item $(\alpha + \beta) \wedge \omega = \alpha\wedge\omega + \beta\wedge\omega$ e
          $\omega \wedge(\alpha + \beta)  = \omega\wedge\alpha + \omega\wedge\beta$
    \item $(c\alpha)\wedge\omega = c(\alpha\wedge\omega) = \alpha\wedge(c\omega)$
    \item $\omega\wedge\alpha = (-1)^{r+s} \alpha\wedge\omega$
    \item $(\alpha\wedge\omega)\wedge\tau = \alpha\wedge(\omega\wedge\tau)$
       (valida grazie alla normalizzazione scelta nella definizione di $\wedge$)
\end{enumerate}
$
   \forall c\in \R \: , \:
   \forall \alpha,\beta \in \Omega^r(M) \: , \:
   \forall \omega \in \Omega^s(M) \: , \:
   \forall \tau \in \Omega^k(M)
$\\

Vale la seguente proposizione.
\begin{proposition}\label{prop:base2}
   Sia $M$ varietà differenziale e $p \in M$. Sia $\{ x^i \}$ una base di coordinate
   locali per $T_p M$ e $ \{ \dd x^i \}$ la base duale in coordinate locali
   per $T^*_p M$. Queste due basi inducono una base locale per le forme differenziali.\\
   Ossia, sia $\omega \in \Omega^r(T_p M)$, $(U,\phi)$ una carta locale,
   allora esistono le funzioni funzioni continue $a_{i_1,\dots, i_r}$ tali che
   in coordinate locali:
   $$
      \phi^{-1*} \omega = \sum_{(i_1,\dots, i_r)} a_{i_1,\dots, i_r}
         \dd x^{i_1} \wedge \dots \wedge \dd x^{i_r}
   $$
   La forma differenziale $\phi^{-1*}\omega$ su $\phi(U)$ è esprimibile come
   combinazione lineare dei $\dd x^i$, che costituiscono una base per $\Omega(\phi(U))$.
\end{proposition}

%------------------------------------------------------------------------------%
%------------------------------------------------------------------------------%
\subsection{Differenziale esterno}
Si definisce l'operatore che a una r-forma
$\omega \in \Lambda^r(p)$ associa la (r+1)-forma $d\omega \in \Lambda^{r+1}(p)$,
definita in coordinate locali da\\
$$
   \omega = \sum_{(i_1,\dots, i_r)} a_{i_1,\dots, i_r}
      \dd x^{i_1} \wedge \dots \wedge \dd x^{i_r} \to
   \dd \omega = \sum_{(i_1,\dots, i_r)} \dd a_{i_1,\dots, i_r}\wedge
      \dd x^{i_1} \wedge \dots \wedge \dd x^{i_r}
$$
ossia
$$
   \dd : \Lambda^r(p) \to \Lambda^{r+1}(p) \quad \omega \mapsto
   \dd \omega = \sum_{(i_1,\dots, i_r,k)} \frac {\partial a_{i_1,\dots, i_r}}
   {\partial x^k} \dd x^k \wedge \dd x^{i_1} \wedge \dots \wedge \dd x^{i_r}
$$
Dall'antisimmetria del prodotto wedge discende immediatamente che $\dd ^2\omega
= \dd (\dd \omega) = 0$
\begin{equation*}
   \begin{split}
      \dd ^2\omega & = \dd  \left( \sum_{(i_1,\dots, i_r,k)}
            \frac {\partial a_{i_1,\dots, i_r}}{\partial x^k} \dd x^k
               \wedge \dd x^{i_1} \wedge \dots \wedge \dd x^{i_r} \right)
          = \sum_{(i_1,\dots, i_r,k)}
            \dd \left( \frac {\partial a_{i_1,\dots, i_r}}{\partial x^k} \right)
               \wedge \dd x^k \wedge \dd x^{i_1} \wedge \dots \wedge \dd x^{i_r} \\
       & = \sum_{(i_1,\dots, i_r,k,j)}
            \frac {\partial^2 a_{i_1,\dots, i_r}}{\partial x^k \partial x^j}
               \wedge \dd x^j \wedge \dd x^k \wedge \dd x^{i_1} \wedge \dots \wedge \dd x^{i_r}
         = 0
   \end{split}
\end{equation*}
in quanto contrazione del termine simmetrico $\frac {\partial^2 a_{i_1,\dots, i_r}}
{\partial x^k \partial x^j}$ e del termine antisimmetrico $\dd x^j \wedge \dd x^k $\\

Si definisce forma \textbf{esatta} una r-forma $\omega$ se esiste una
(r-1)-forma $\alpha$ che verifica $\omega = \dd \alpha$\\
Si definisce forma \textbf{chiusa} una r-forma $\omega$ tale che $\dd \omega = 0$.\\
Segue immediatamente che ogni forma esatta è chiusa. L'inverso è vero solo localmente.
\begin{lemma}{(di Poincarè)}
    Sia $M \subset \R^n$ una palla aperta. Una r-forma $\omega$ chiusa
    definita su $M$ è esatta.
\end{lemma}
%------------------------------------------------------------------------------%
\subsection{Coomologia di de Rham}
Poichè ogni forma esatta è anche chiusa, ma non vale il viceversa, si vuole
dare una classificazione delle forme chiuse.\\
Siano $W^k(M)$ lo spazio delle k-forme esatte su $M$ e $Z^k(M)$ lo spazio delle
forme chiuse, $Z^k(M) \subset W^k(M)$. Si definisce il \emph{k-esimo gruppo di
comologia di De Rham} lo spazio quoziente
$$
   H^k(M) = W^k(M)/Z^k(M)
$$
Gli elementi di $H^k(M)$ sono quindi forme che differiscono per una forma esatta
$$
    [\omega] = \{ \alpha \in W^k(M)\tc \omega -\alpha = \dd \theta \}
$$
I gruppi di comologia di De Rham sono strettamente legati alla topologia della
varietà. Si veda il teorema di De Rham.
%------------------------------------------------------------------------------%
\subsection{Integrazione}
L'integrazione di una forma su una varietà può essere definita
tramite coordinate locali e ricondotta a integrazione su aperti di $\R^n$.
Una trattazione rigorosa richiederebbe l'introduzione del concetto di
\emph{partizione dell'unità}, che esula dallo scopo di questo elaborato.
Si vedano \cite{boothby},\cite{nakahara} per una trattazione rigorosa. \\
In maniera intuitiva, sia $(U,\phi)$ una carta della
varietà $M$, dove $U$ intorno del punto $p \in M$, e $R = \phi(U)$ e
$\omega = \dd x^1 \wedge \dots \wedge \dd x^n$ una n-forma.
Sia $f : U \to \R$ una funzione integranda.
Si definisce automaticamente la misura di integrazione $\dd \mu = \dd x^1 \dots \dd x^n$.
e si può dare senso all'espressione in coordinate locali:
$$
   \int_U f\omega = \int_R f(x_1,\dots,x_n) \dd x^1 \dots \dd x ^n
$$
Si vuole poi ripetere questa operazione per tutte le carte di un atlante dato e
"incollare" assieme i risultati in maniera che l'integrale sulla varietà sia uguale
alla somma degli integrali sulle singole carte.\\

Si osserva che è possibile integrare solamente n-forme su varietà di dimensione $n$
perchè le forme con $r>n$ sono tutte nulle, e gli spazi di dimensione $r<n$ hanno
misura nulla in $\R^n$.

\subsubsection{Teorema di Stokes}
\begin{theorem}{(di Stokes)}:
   Sia $M$ una n-varietà differenziale con bordo $\partial M$ e si supponga assegnata
   un'orientazione. Sia $\omega$ una (n-1)-forma su $M$. Allora
   \begin{equation}\label{eq:stokestheorem}
      \int_M \dd \omega = \int_{\partial M} \omega
   \end{equation}
\end{theorem}
Questo teorema porta come corollari e casi particolari i vari teoremi del rotore,
divergenza, etc.
%------------------------------------------------------------------------------%
%------------------------------------------------------------------------------%
\subsection{Varietà Riemanniane}
\begin{definition}
   Una \emph{metrica Riemanniana} $g$ su una varietà $M$ è un (2,0)-campo tensoriale
   su $M$ che per ogni punto $p \in M$ soddisfa:
   \begin{enumerate}
      \item $ g_p(U,V) = g_p(V,U) $
      \item $ g_p(U,U) \geq 0 $ dove $ g(U,U) = 0 \iff U = 0$
   \end{enumerate}
   dove $U,V \in T_p(M)$.\\

   Una \emph{metrica pseudo-Riemanniana} $g$ su una varietà $M$ è un
   (2,0)-campo tensoriale su $M$ che per ogni punto $p \in M$ soddisfa:
   \begin{enumerate}
      \item $ g_p(U,V) = g_p(V,U) $
      \item $ g_p(U,V) = 0 \: \forall U \in T_p(M) \Rightarrow V = 0$
   \end{enumerate}
\end{definition}

Data una carta $(U,\phi)$ di $M$ con coordinate $\{x^\mu\}$ il tensore $g$ può
essere scritto come
$$ g_p = g_{\mu\nu}(p) \dd x^\mu \dd x^\nu =: \dd s^2$$
Dove $g_{\mu\nu}(p)$ può essere considerato come la $\mu\nu$-esima entrata di una matrice.\\
Il numero $(p,n)$ di autovalori positivi $p$ e negativi $n$ è detto \emph{indice} della
metrica. Se $n=1$ la metrica è detta Lorentziana.\\
Si può diagonalizzare la metrica e riscalare gli autovettori in modo da ottenere
solamente $\pm 1$ sulla diagonale. Ad esempio si ha la metrica Euclidea
$\delta = \mathrm{diag}(1,1,\dots,1)$ o Minkowskiana $\eta = \mathrm{diag}(-1,1,...,1)$.\\

Se una varietà differenziale $M$ è dotata di una metrica Riemanniana $g$, la coppia $(M,g)$
è detta \textbf{varietà Riemanianna} (analogamente se $g$ è pseudo-Riemann).\\

%------------------------------------------------------------------------------%
%------------------------------------------------------------------------------%
\subsection{Derivata covariante}
Si vuole estendere il concetto di derivata direzionale ai campi vettoriali definiti
su una varietà differenziale $M$. Lo stesso processo può essere usato per definire
la derivata di tensori e forme differenziali.\\

Sia $X$ un campo vettoriale sulla varietà $M$ (supponendo $M = \R^n$,
per semplicità), $p \in M$ e $h \in M$ pensato come piccolo spostamento da $p$, in $M$.
Volendo definire la derivata direzionale di un campo vettoriale nel modo usuale
$$
   \lim_{|h| \to 0} \frac{X_{p+h}-X_p}{|h|}(f)
$$
si riscontra subito un problema. I due vettori $X_{p+h} \in T_{p+h}(M)$ e
$X_p \in T_p(M)$ appartengono a due spazi differenti e non possono essere confontati.\\
Occorre un modo di trasportare il vettore $X_{p+h}$ da $T_{p+h}(M)$ a $T_p(M)$
lasciandolo inalterato. Questo processo è chiamato \textbf{trasporto parallelo}.
Purtroppo non esiste una maniera univoca per trasportare un vettore tangente in
una varietà, quindi è necessario specificare come viene effettuato il trasporto
parallelo. Di seguito alcuni esempi.

\subsubsection{Spazio euclideo}
Sia $X$ un campo vettoriale su $M = \R^n$ e $X_p$ il vettore tangente al punto $p \in \R ^n$.
Nel caso di uno spazio euclideo di solito si confrontano i vettori
$X_{p+h}$ e $X_p$ semplicemente traslando ($\tau$) il vettore $X_{p+h}$ fino al punto $p$,
mantenendolo parallelo, e valutando la differenza tra i due vettori ora tangenti
a $p$, definendo il vettore spostamento $\Delta X = \tau \cdot X_{p+h} - X_p$.
La derivata covariante allora è semplicemente la derivata direzionale in
direzione di $\Delta X$.

\subsubsection{Curva nel piano}
Sia ora la varietà differenziale $M$ una curva $\sigma : [0,1] \to \R^2$. Si scelga
una parametrizzazione della curva in coordinate polari $\sigma(t) = (r(t),\theta(t))$ e
siano $\vec u _r$ e $\vec u _\theta$ i versori delle coordinate polari tangenti
alla curva. In generale, a istanti $t$ successivi, i vettori
($\vec u _r (t+\Delta t) ,\vec u _\theta (t + \Delta t)$) risulteranno ruotati
rispetto a ($\vec u _r (t), \vec u _\theta (t) $), quindi la semplice traslazione
parallela di un vettore $X_{\sigma{t+\Delta t}}$ tangente al punto $\sigma(t + \Delta t)$
non è adeguata. Occorre tener conto anche della rotazione dei versori
$\vec u _r$ e $\vec u _\theta$, che è una proprietà intrinseca della curva legata
alla sua curvatura.

\subsubsection{Sfera}
Sia $M=S^2$ la sfera bidimensionale. Una sfera è uno spazio non curvo. Il tipico
esempio per spiegare la curvatura della sfera, legato anche al trasporto parallelo,
è il seguente. Sia $p$ un punto dell'equatore e $X$ un vettore tangente a $p$. Si
supponga di trasportare parralelamente $X$ lungo l'equatore fino a un altro punto
$q$ dell'equatore, dopodichè lo si trasporti da $q$ al polo nord lungo un meridiano
e di nuovo dal polo nord a $p$ lungo un altro meridiano. Si osserva che il vettore
$X'$ così trasportato non ritorna in se stesso, ma ha un'altra orientazione. Un
risultato diverso si otterrebbe se fosse scelto un altro cammino sulla sfera.
Questa è una proprietà intrinseca dovuta alla curvatura della sfera e di ciò
bisogna tener conto nel definire la derivata covariante.\\

La derivata covariante dipende dalla geometria intrinseca della varietà su cui
si vuole applicare l'operazione, e può essere usata (in senso opposto al ragionamento
appena fatto) per definire curvatura e torsione di una varietà.\\
Si vedano le referenze \cite{boothby},\cite{nakahara} per una definzione formale.
