\chapter{Teorie di gauge}
\textcolor{red}{(da scrivere)}\\
\begin{itemize}
   \item Principio di invarianza di gauge, trasformazioni, gruppi abeliani e non abeliani.\\
   \item simmetrie globali e locali (es rotazione rigida, forza centrifuga)\\
   \item formalismo dei Fibrati. analogia con forma di connessione\\
   fibrato banale > coperto da una sola carta > simmetria globale.\\
   fibrato non banale > non coperto da una sola carta > simmetria locale.
\end{itemize}

esempi:
 \begin{itemize}
   \item elettrodinamica classica\\
   \item teoria scalare O(n)\\
\end{itemize}
