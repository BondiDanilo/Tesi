\chapter{Teorie di gauge}

Una teoria di gauge è una teoria di campo la cui lagrangiana è invariante sotto
l'azione di un gruppo di Lie\footnote{
   Cfr. definizioni \ref{def:groupaction} e \ref{def:liegroup}.
} $G$, denominato \emph{gruppo di gauge}. Se il gruppo di gauge è un gruppo (non)
abeliano la teoria di gauge viene detta (non) abeliana.\\

Si consideri, ad esempio, la seguente lagrangiana per un campo scalare complesso
$\phi : \R \times \R^3 \to \C$ (dove $V$ è un generico funzionale, il potenziale).
$$
   \Ll [\phi] = \partial _\mu \phi^*  \partial ^\mu \phi + V[\phi ^* \phi]
$$
È invariante per una trasformazione di fase globale del campo, ossia in seguito
alla trasformazione
$
  \phi \mapsto \phi' = e^{i \alpha}\phi
$
(dove $\alpha \in \R$) la lagrangiana non cambia.
$$
   \Ll [\phi'] = (\partial _\mu \phi')^*  \partial ^\mu \phi' + V[(\phi')^* \phi']
                = \partial _\mu  e^{-i \alpha}\phi^*  \partial ^\mu  e^{i \alpha}\phi
                   +  V[e^{-i \alpha}\phi ^*  e^{i \alpha}\phi]
                = \partial _\mu \phi^*  \partial ^\mu \phi + V[\phi ^* \phi]
                = \Ll [\phi]
$$
Una simmetria di questo tipo è detta \emph{simmetria globale}, in quanto non dipende
dal punto dello spazio-tempo in cui è applicata ($\alpha$ è infatti una costante).\\

Un altro esempio di simmetria globale è, nella meccanica Newtoniana, una
trasformazione di Galileo tra due sistemi inerziali (es: una rotazione degli assi):
in seguito a tale trasformazione non cambia la descrizione fisica del fenomeno,
ma solamente i "numeri" che ogni osservatore usa come coordinate. È quindi importante
evidenziare quali siano le trasformazioni tra un sistema di coordinate e l'altro.\\

Un caso più interessante è quello in cui il sistema presenta una simmetria definita
non globalmente, ma localmente. Si può pensare, ad esempio, a una trasformazione
di coordinate tra un osservatore inerziale e uno non inerziale (ad esempio sottoposto
a un moto rotatorio). Le descrizioni che i due osservatori danno del sistema
non sono equivalenti: il secondo, ad esempio è costretto ad introdurre l'esistenza
di forze fittizie (ad esempio la forza centrifuga), che in generale possono dipendere
in maniera non banale dalle coordinate (ad esempio può dipendere dalla distanza
un un punto dall'osservatore).\\

In relazione alla lagrangiana scritta in precedenza, si può considerare una trasformazione
di fase in cui $\alpha = \alpha(x,t)$ è una generica funzione delle coordinate
$\phi \mapsto \phi' = e^{i\alpha(x,t)}\phi$. Se la lagrangiana è invariante per
trasformazione di questo tipo, tale trasformazione è detta \emph{simmetria locale}
o simmetria di gauge.\\

Il principio cardine delle teorie di gauge è allora promuovere le simmetrie globali
di una lagrangiana a simmetrie locali (simmetrie che possano essere applicate
solamente nell'intorno di un punto, senza affligere il resto dello spazio), e studiare
i casi in cui queste si conservano come simmetrie della teoria. \\

Matematicamente, una teoria di gauge è descritta da un fibrato principale\footnote{
   Cfr. definzione \ref{def:principalbundle}.
}, in cui
la varietà di base $M$ è (ad esempio) lo spaziotempo $\R\times\R^3$ con metrica
Minkowskiana ($g_{\mu\nu} =\eta= \mathrm{diag}(-1,1,1,1$)) e la fibra $G$ è il
gruppo di gauge.\\
Data la simmetria globale del gruppo di gauge, l'assegnazione di una simmetria
locale (ossia la scelta della particolare funzione $\alpha(x,t)$, nell' esempio
 precedente) corrisponde alla scelta di un ricoprimento $\{U_i\}$ della varietà
 e di sezioni locali $s_i$ sul fibrato\footnote{
   Cfr. definizione \ref{eq:section}.
 }, che nelle regioni di intersezione dei rispettivi domini ($U_i \cap U_j$)
 sono legate dalle fuzioni di transizione \ref{eq:transfunctions}.
Le funzioni di transizione sono dette \emph{trasformazioni di gauge}.\\

Assegnata una 1-forma di connessione $\omega$ sul fibrato e delle sezioni locali
$s_i$, i pullback di $\omega$ tramite le sezioni sono 1-forme sullo spaziotempo
$A_i = s_i^* \omega$ e sono detti \emph{potenziali di gauge}.\\
Si sottolinea l'importanza del teorema \ref{thm:gaugepotential} che dati i
potenziali di gauge definiti sugli intorni locali $U_i$ esiste sempre la 1-forma
di connessione $\omega$ sul fibrato.\\

Il pullback della curvatura $\Omega$, definita come differenziale esterno della
1-forma di connessione $\omega$, è detto \emph{tensore forza di campo} $F$.\\

Se il fibrato è banale, ossia può essere ricoperto da una sola carta ed ha la
struttura globale di prodotto diretto $M \times G$, allora esiste una simmetria
globale per il sistema. Se invece il fibrato è non banale, ossia non è descritto
tramite un'unica carta, non può essere definita una simmetria globale e il potenziale
di gauge può essere descritto solo tramite diverse carte locali, concordanti sulle
regioni di intersezione tramite una trasformazione di gauge\footnote{
  Si veda il monopolo di Wu-Yang nella sezione successiva \ref{sec:wuyangmonopole}.
}.\\

\subsubsection{Esempio: Elettrodinamica classica}
Si vuole descrivere ora l'accoppiamento di un campo complesso $\phi$ con il campo
elettromagnetico (si pensi ad esempio alla funzione d'onda di una particella carica).\\
Si consideri la lagrangiana $\Ll$ definita in precedenza, che si è già visto
essere invariante per trasformazioni \emph{globali} di fase $\phi \mapsto e^{iq\alpha}\phi = U\phi$,
dove il parametro $q \in \R$ è detto la costante di accoppiamento della teoria
(in questo caso la carica elettrica).
\begin{equation*}
  \begin{aligned}
     \Ll [\phi] & = \partial _\mu \phi^*  \partial ^\mu \phi + V[\phi ^* \phi] &
     & \Rightarrow \Ll[\phi] = \Ll[\phi'] & ,\quad
     \phi &\mapsto \phi' = e^{i\alpha}\phi
  \end{aligned}
\end{equation*}

Si vuole ora promuovere la simmetria a simmetria locale $\alpha \mapsto \alpha(x,t)$
e richiedere che la lagrangiana rimanga invariata per trasformazione di fase locale.
Si sostituisce la derivata pariale $\partial \mu$ con la derivata covariante
definita da
$$
   D _\mu := \partial _\mu -iq A _\mu
$$
dove il campo $A _\mu : \R \times \R^3 \to \R \times \R^3$ è il potenziale
di gauge (in questo caso, il potenziale elettromagnetico). La lagrangiana
gauge-invariante diventa:
$$
   \Ll ' =  D _\mu \phi^*  D ^\mu \phi + V[\phi ^* \phi].
$$
La richiesta di invarianza della lagrangiana per trasformazione
di fase locale si traduce nella richiesta che la derivata covariante $D _\mu$
sia invariante. $D _\mu \mapsto D' _\mu = U^{-1} D _\mu U = D _\mu $.
\begin{equation}
   \begin{aligned}
      iq \partial _\mu \alpha(x,t) + (\partial _\mu - iq A _\mu)
         & = (\partial _\mu -iq A' _\mu) \Rightarrow \\
      \Rightarrow A' _\mu = A _\mu - \partial _\mu  \alpha(x,t)
         & = U ^{-1} A _\mu U + \frac{i}{q} U \dd U
   \end{aligned}
\end{equation}
che si traduce nella condizione che il potenziale $A _\mu$ trasformi secondo la
trasformazione di gauge sopra scritta.\\

Se si calcola il tensore energia impulso della lagrangiana così scritta (si indica
$\phi| ^{\mu} = \partial \phi / \partial x_\mu$ e
$\phi| _{\mu} = \partial \phi / \partial x^\mu$ ), si vede immediatamente che non si
conserva.
\begin{equation}
   \begin{aligned}
      T'_{\mu\nu} & = \frac{\partial \Ll'}{\partial \phi|^{\mu}}\phi|_{\nu}
      - \eta _{\mu\nu} \Ll' &
      \Rightarrow \partial ^\mu T'_{\mu\nu} & \neq 0
   \end{aligned}
\end{equation}
Occorre allora tenere conto anche della dinamica del campo elettromagnetico, costruendo
la lagrangiana per l'accoppiamento del campo $\phi$ con il campo elettromagnetico
che è Lorentz-invariante, con simmetria di gauge $U(1)$ e che conserva energia e
quantità di moto\footnote{
   Si osserva che definendo la lagrangiana del campo elettromagnetico libero
   $$
   \Ll _{em} = - \frac{1}{4}F^{\mu\nu} F_{\mu\nu}
   $$
   Si ottengono come equazioni del moto esattamente le equazioni di Maxwell nel vuoto.
}.

\begin{equation}
\Ll = -\frac{1}{4} F^{\mu\nu}F_{\mu\nu}
+ \frac{1}{2} (D^\mu \phi)^* D _\mu \phi - V[\phi^* \phi]
\end{equation}
Dove $F_{\mu\nu}$ è il \textbf{tensore elettromagnetico}, cioè il tensore forza
di campo definito da $F = \dd A$, o in coordinate:
$$
   F_{\mu\nu} = \partial _\mu A _\nu - \partial _\nu A _\mu
$$
