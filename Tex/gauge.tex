\chapter{Teorie di gauge}
\textcolor{red}{(da scrivere)}\\
\begin{itemize}
   \item Principio di invarianza di gauge, trasformazioni, gruppi abeliani e non abeliani.\\
   \item simmetrie globali e locali (es rotazione rigida, forza centrifuga)\\
   \item formalismo dei Fibrati. analogia con forma di connessione\\
   \item esempi\\
   \item elettrodinamica classica\\
   \item teoria scalare O(n)\\
\end{itemize}
If a bundle admits a global section, the base can be covered by just one
chart. Such a trivial bundle can be characterized by the zero Chern class c 0
and has a global structure as a direct product of the base X and the Lie group
G. In general, the triviality of the bundle depends on the contractibility of
the base. For example, if we take the base to be just R 3 , it can be covered
by a single chart and therefore both the potential and the field strength
tensor are defined globally in this case.\\
