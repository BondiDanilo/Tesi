
\documentclass[10pt,a4paper]{report}
\usepackage{graphicx}
\usepackage[utf8]{inputenc}
\usepackage{amsthm}             %stili teoremi
\usepackage{amssymb,amsmath}
\usepackage[colorlinks = true,
          linkcolor = blue,
          urlcolor  = blue,
          citecolor = blue,
          anchorcolor = blue]{hyperref}
\usepackage{bookmark}
\usepackage{wrapfig}

\usepackage{geometry}
\geometry{
	top=25mm,
	bottom=25mm,
	left=30mm,
	right=25mm
}
\usepackage[ampersand]{easylist}
\usepackage[nottoc]{tocbibind} %bibliografia
%------------------------------------------------------------------------------%

% Stile+numerazione teoremi, definizinoni, etc...
\theoremstyle{plain}
\newtheorem{theorem}{Teorema}[equation]

\theoremstyle{definition}
\newtheorem{definition}{Definizione}[equation]

\theoremstyle{plain}
\newtheorem{proposition}{Proposizione}[equation]

\theoremstyle{plain}
\newtheorem{lemma}{Lemma}[equation]

\theoremstyle{remark}
\newtheorem{example}{Esempio}[equation]

\theoremstyle{definition}
\newtheorem{axiom}{}[section]

\renewcommand\qedsymbol{$\blacksquare$} % quadrato della dimostrazione
%------------------------------------------------------------------------------%
\renewcommand{\vec}[1]{\mathbf{#1}} % simbolo di vettore = grassetto
\newcommand{\chrsym}{\genfrac{\{}{\}}{0pt}{}} % Simboli Christoffel
%------------------------------------------------------------------------------%
\graphicspath{{/home/dan/Desktop/UNI/TESI/Images/}}	%Default path for graphics
%------------------------------------------------------------------------------%
% Remove default parindent
\newlength\tindent
\setlength{\tindent}{\parindent}
\setlength{\parindent}{0pt}
\renewcommand{\indent}{\hspace*{\tindent}}

\newcommand{\tab}[1][1cm]{\hspace*{#1}} % ridefinisce la tabulazione

%------------------------------------------------------------------------------%
% Metadati
\hypersetup{
	pdftitle={Tesi},%
	pdfauthor={Danilo Bondì},%
	pdfsubject={Aspetti classici e quantistici dei monopoli magnetici in teorie di gauge},%
	pdfkeywords={},%
	colorlinks=true,%
	linkcolor=blue,%
	linktocpage=true,%
	pageanchor=true
}

\begin{document}

No magnetic monopole has ever been observed in nature. Classical electrodynamics,
however, does not provide the reason their nonexistance. Since Maxwell equations
merely formalize the experimental observations on electric and magnetic phenomena,
no a priori rejection of the monopole is made.\\
Despite the lack of concrete experimental results, the interest in a consistent
theory of magnetic monopoles has not vanished throughout the past century, since
Dirac first published his original paper in 1931.\\

Our first step in introducing a theory of the magnetic monopole will be a naive
construction of an elementary classical model, aimed to write the equations of motion
of an electron in a Coulomb-like magnetic field. This is only meant to stress the
main problems of the subject.
We begin assuming the existence of a magnetic charge, analogous to the electric
charge, and consequently modify Maxwell equations, including the non-zero
divergence of the magnetic field $\vec B$, that has to be equal to the local
magnetic charge density, namely $\rho_g$.\\
Here arises the first contraddiction. If we define the electromagnetic potential
$\vec A$, $\vec B$ being the curl of $\vec A$, it is impossible to have both
$\vec B = \nabla \times \vec A$ and $\nabla \cdot \vec B = 4\pi \rho_g$ at the same
time. The reason is that $\vec A$ is not defined everywhere in space, but will
always have a string of singularities.\\

A non contraddictory monopole theory is incompatible with a global vector potential.
We are forced to dercribe our model using only local potentials, that are required
to agree on their overlap region via an appropriate trasformation.\\
The frame that fits the most to this picture is that of gauge theories.\\\

Since classical electrodynamics is a $U(1)$ gauge theory, we define local potentials
on two open sets, requiring that a $U(1)$ gauge trasformation connects them in the
overlapping region. This way it is possibile to define the correct monopole field
and to remove the previous contraddiction. Also we give a topological meaning
to the magnetic charge, via the characteristic Chern classes of the manifold of
the theory. A relationship between magnetic and electric charge is obtained
requiring the trasformation of the fields to be single valued.\\
One of the main problems of this Abelian model is to be found when we try to
extend our theory to a quantum field theory.\\

The next step is to generalize our theory to a non-Abelian gauge group, having
$U(1)$ as a subgroup, and that will restrict to our previous abelian theory
under normal conditions.
This spontaneous symmetry breaking process preserves all the earlier predictions,
gives a solution to the problems of the previous model.\\
Theories of this kind are named \emph{Yang-Mills theories}.\\

The first and simplest case is considering $SU(2)$ as the gauge group. We will
analyze two models of this type.

The first model, proposed by Wu and Yang in 1969, is capable of solving all of
the abelian model problems, but gives badly-defined configuration energies, leading
to divergencies.\\
The second one is a special case of the more general model proposed by Georgi and
Glashow in 1974. The gauge potential is coupled with another complex field,
the Higgs field. This immediatly breaks the symmetry of $SU(2)$ down to $U(1)$,
and solves the infinite energies problem. The last achievement is a definition
of the magnetic charge that is directly derived from the conservation of the
field strenght tensor, so from Maxwell equations themselves.\\ 

Unfortunately, the model has no analytic solution in general: numerical
solutions to the problem are to be found.
We conclude our dissertation briefly mentioning an ansatz proposed
by 't Hooft and Polyakov, a starting point for numerical solutions of the Georgi-Glashow
model.
\end{document}
