\chapter*{Introduzione}
\begin{equation*}
   \begin{aligned}
      \nabla \times \vec E & = -\frac{1}{c}\frac{\partial \vec B}{\partial t}&
      \nabla \times \vec B & = \frac{1}{c} \left(
         4\pi \, \vec J + \frac{\partial \vec E}{\partial t} \right) \\
      \nabla \cdot \vec E & = 4\pi \, \rho &
      &\boxed{\nabla \cdot \vec B = 0} \\
   \end{aligned}
\end{equation*}\\

La quarta delle equazioni di Maxwell asserisce che "non esistono sorgenti del
campo magnetico", ossia non esistono monopoli magnetici. Essendo le equazioni di
Maxwell di natura fenomenologica, non esite tutt'ora una teoria che ne escluda
totalmente l'esistenza. \\
La relatività speciale mostra che vi è una quasi totale simmetria tra campo elettrico
e campo magnetico. Il campo magnetico prodotto da un filo percorso da corrente
diventa un campo puramente elettrostatico se osservato nel sistema di riferimento
solidale con la carica in moto, e allo stesso modo il campo elettrostatico prodotto
da una carica ferma diventa campo magnetico se osservato in un sistema in moto rispetto
alla carica.\\
Questa simmetria è però rotta dal fatto che non sono mai stati tuttora osservati
monopoli magnetici, mentre i monopoli elettrici sono ben noti fin dalle prime
formulazioni dell'elettromagnetismo.\\
L'interesse nello studio dei monopoli magnetici è innanzitutto ripristinare
questa rottura di simmetria, processo molto comune in Fisica (ad esempio per le
Grandi Teorie Unificative, \emph{GUT}), ma presenta relazioni anche con altri
grandi problemi tutt'ora irrisolti della Fisica Teorica, ad esempio il problema
del Confinamento in Cromodinamica Quantistica, il problema del decadimento del
protone e la quantizzazione della carica elettrica.\\

Lo scopo di questo elaborato è dare una breve panoramica del monopolo magnetico,
secondo i formalismi della relatività ristretta, la meccanica quantistica non
relativistica e le teorie di campo classiche. Non verranno trattati in alcun modo
aspetti di teorie di campo quantistiche. \\
Saranno affrontati sostanzialmente tre tipi di monopoli: il monopolo di Dirac,
per introdurre il problema a livello classico, il monopolo di Wu-Yang in una
teoria di gauge abeliana, e il monopolo di 't-Hooft-Polyakov in teoria di gauge
non abeliana.

% TUTT'ORA
% Se noti, lo hai usato 3 volte: la prima e l'ultima con l'apostrofo,
% mentre la seconda senza. Quella senza è sbagliata: lì ci vuole proprio FINORA
% o ancora, come preferisci. (ancora ricopre i sensi di tutt'ora e finora,
% ma la cosa non vale anche negli altri due casi).
% 
% Primissima frase: "la quarta delle" è un po' bruttino per iniziare un testo:
% se tu avessi già parlato (premessa? Non so) delle equazioni di Maxwell in generale,
%  andrebbe bene, ma tieni a mente che "delle" contiene una implicita referenza a
%  "quelle, le specifiche" - in linea di massima meglio usarlo con qualcosa di già nominato.
% "La quarta equazione di Maxwell" non può andare bene? Tanto stai comunque dando
% per scontato il fatto che esistano. Che il lettore le conosca o meno non ha
% rilevanza in ogni caso, dato che le hai comunque escluse.
% 
% Una piccola nota sul primo paragrafo:
% Talvolta non è ben chiaro il rapporto che vuoi creare tra una Frase e un'altra.
% Tra la prima e seconda frase: stai implicando un tuttavia?
% O, siccome non vuoi implicare nulla, vuoi lasciare la giustapposizione?
% (In quella stessa seconda frase forse potresti
% sostituire "tutt'ora" con "ad oggi")
% 
% ", se osservato........,"
% (scrivo anche la virgola in fondo perché rende evidente che si tratta di un inciso)
% La cosa più importante, però
% "L'interesse nello studio dei vlablabla è innanzitutto di ripristinare
% questa rottura di simmetria"
% È un po' tremenda la frase più importante di tutte
% Sembra che tu voglia che ci sia una rottura che era sparita,
% ovvero il contrario di quello che vuoi dire.
% Proporrei:
% - "Lo scopo principale dello studio dei blabla è di ripristinare questa simmetria"
% - "La prima ragione di interesse per lo/nello studio dei blabla è la
% riparazione/il ripristino di questa/tale simmetria"
% 
% Attingi come preferisci, basta che si mantenga il senso attivo della frase -
%  non riportare in mezzo sia la rottura che la riparazione che la simmetria,
%  perché diventa confuso Sempre in quella frase, io scioglierei il ", ma presenta".
%  La frase era già abbastanza importante di suo,
%  mentre quella dopo è un lungo cocktail di esempi.
%  Peraltro diventa difficile riformulare il verbo perché funzioni
%  per entrambe senza fare giri strani...
% "Lo studio dei blabla ha come principale scopo il ripristino di
% questa simmetria (o quello che vuoi), ma presenta..." così va bene,
% perché IL SOGGETTO è lo studio. Nella tua frase il soggetto cambia da
% L'interesse a Lo studio - questo è abbastanza grave,
% per la frase più importante di tutta la tesi


