\chapter{Monopoli in teorie di gauge abeliane}
Nella sezione \ref{sec:gaugestring} si è visto come non è possibile definire globalmente
(su tutto lo spazio $\mathbb{R}^3 \setminus \{0\}$) un potenziale di monopolo regolare,
incorrendo nella stringa di singolarità. Il primo approccio è stato di rendere
equivalenti tutte le possibili configurazioni di direzione della stringa, tramite
trasformazione di gauge di tipo $U(1)$. Si è arrivato a definire due potenziali
(\ref{eq:localdiracpotential}) $\vec A ^\pm$ che hanno la stringa situata rispettivamente
lungo l'asse $z$ negativo/positivo, quindi regolari su $\pm z$ ripettivamente.\\
L'approccio qui seguito, proposto da Wu e Yang (1968) \cite{wuyang}, è di
rinunciare a una definizione globale del potenziale di Dirac
in favore di una descrizione tramite due potenziali definiti localmente su due aperti
$U^\pm$, che concordano nella regione di intersezione tramite una trasformazione
di gauge.\\

Innanzitutto, Poichè $\mathbb{R}^3 \setminus \{0\}$ è equivalente omotopicamente
a $S^2$, si vuole studiare il problema sulla sfera. Siano allora $U^\pm$ l'emisfero
nord e sud della sfera
\begin{equation*}
   \begin{aligned}
      U^+ &= \{(x,y,z) \in S^2 \: | \: z > 0 \}
          = \{(r,\theta,\phi) \in S^2 \: | \: 0 \leq \theta \leq \pi/2 \} \\
      U^- &= \{(x,y,z) \in S^2 \: | \: z < 0 \}
          = \{(r,\theta,\phi) \in S^2 \: | \: \pi/2 \leq \theta \leq \pi \}
   \end{aligned}
\end{equation*}
La regione di intersezione è l'equatore
$$
   U^0 = U^+ \cap U^- = \{(x,y,z) \in S^2 \: | \: z = 0 \}
       = \{(r,\theta,\phi) \in S^2 \: | \: \theta = \pi/2 \}
$$
Si possono allora definire i potenziali $\vec A^\pm$ su $U^\pm$, che nella regione
$U^0$ sono legati, come visto nella sezione \ref{sec:gaugestring}, da
$$
   \vec A ^+ - \vec A^- = \nabla \lambda = \nabla (2g\phi) = \frac{2g}{r\sin\theta} \, \vec u _\phi
$$
Si noti che $\nabla \lambda$ è singolare in $\theta = 0,\pi$, ma nella regione $U^0$
si ha $\theta = \pi/2$.\\
Si calcola allora il flusso totale come:
\begin{equation*}
   \begin{aligned}
      \int_{S^2} d\sigma \: \nabla \times \vec A & =
         \int_{U^+} d\sigma \: \nabla \times \vec A^+ +
         \int_{U^-} d\sigma \: \nabla \times \vec A^-
        = \oint_{U^0} d\vec l \: \vec A^+ -
          \oint_{U^0} d\vec l \: \vec A^- \\
      & = \oint_{U^0} d\vec l \: \nabla \, (2g\phi)
        = \int_0^{2\pi} \, 2g \, \phi \, d\phi
        = 4g\pi
   \end{aligned}
\end{equation*}
In accordo con \ref{eq:B4gpi}.\\

Il formalismo più naturale per descrivere il potenziale di monopolo in una teoria
di gauge è quello di un fibrato principale con spazio base $M = \mathbb{R}^3
\setminus \{0\}$ e gruppo di struttura $G$ il gruppo di gauge.\\
Si faccia riferimento all'appendice per la teoria sui fibrati \ref{sec:fibrati}.

Se il fibrato ammette una sezione globale, può essere coperto con un'unica carta.
Il fibrato è allora banale e ha struttura globale di prodotto diretto tra la
varietà $M$ e la fibra $G$, e la classe di Chern caratteristica è $c_0 = 1$.
La complessità della struttura del fibrato di solito è legata alla contraibilità
dello spazio base.\\

Si consideri una teoria di gauge con gruppo $G = U(1)$, come l'elettrodinamica classica.\\

Se si prende come base $M = \mathbb{R}^3$, che è uno spazio contraibile, può essere
ricoperto da un'unica carta e il potenziale di gauge $A$ è definito globalmente su $M$,
continuo e differenziabile, e il tensore elettromagnetico $F$ è una 2-forma
sia chiusa $DF = 0$ che esatta $F = dA$.\\

Se invece lo spazio base è $M = \mathbb{R}^3 \setminus \{0\}$, che non è uno spazio
contraibile, la situazione cambia drasticamente. $\mathbb{R}^3 \setminus \{0\}$ è
equivalente omotopicamente alla sfera $S^2$.
In riferimento all'esempio \ref{ex:monopolechern}, è possibile costruire due fibrati
principali: uno con classe di Chern $c_0 = 1$ e uno con $c_1 = F/2\pi$. Quest'ultimo,
detto appunto \emph{fibrato di monopolo} è quello che descrive correttamente il
monopolo di Wu-Yang.
%------------------------------------------------------------------------------%
%------------------------------------------------------------------------------%
\section{Fibrato di Monopolo di Wu–Yang}
Si consideri un fibrato principale con $M = S^2$ e $G = U(1)$.\\
Il gruppo di gauge $U(1) = \{g = e^{i\alpha} \}$ è parametrizzato da un
parametro $\alpha$, che è una coordinata ciclica lungo la fibra, ed è identificato
quindi con la sfera $S^1$.\\
Una parametrizzazione per la sfera di raggio unitario è data dalle usuali
coordinate polari $\phi \in[0,2\pi)$ e $\theta \in [0,\pi)$,
quindi si può definire una base per lo spazio tangente $T_pS^2$ come
$\{ \frac{\partial}{\partial \theta} \: , \: \frac{\partial}{\partial \phi } \}$
e per lo spazio cotangente $T^*_pS^2$ come $\{ d\theta \: , \: d\phi \}$.
La derivata esterna è data da
$$
   d = d\theta \frac{\partial}{\partial \theta}
     + d\phi   \frac{\partial}{\partial \phi }
$$
Siano $U^\pm$ emisferi nord e sud e $U^0$ l'equatore come sopra.
Si hanno allora le due carte locali per il fibrato
$$
   ( U^+ \times S^1 , \{ \theta, \phi,\alpha^+ \} ) \quad \mathrm{e}\quad
   ( U^- \times S^1 , \{ \theta, \phi,\alpha^- \} )
$$
La 1-forma di connesione $\omega$\footnote{Si veda \ref{}}
definita globalmente sul fibrato, portata su $U^\pm$ tramite pullback,
dà il potenziale $A$
$$
   A = A(\theta)\wedge dr =  \begin{cases}
      A^+ =  \frac{n}{2}(1 - \cos\theta ) d\phi & \mathrm{su \:} U^+ \\
      A^- = -\frac{n}{2}(1 + \cos\theta ) d\phi & \mathrm{su \:} U^-
   \end{cases}
$$
e la 2-forma di curvatura $\Omega$\footnote{Si veda \ref{}} viene portata
sul tensore elettromagnetico $F$
$$
   F = dA = \frac{n}{2} \: \sin\theta \: d\theta \wedge d\phi
$$
Osservando che i versori delle coordinate sferiche sono dati da
$\vec u _\phi = \sin\theta d\phi$ e $ \vec u _\theta = d\theta$, si riconosce in
$F$ la componente radiale del campo magnetico di un monopolo (scriverla).\\

La 2-forma $F$ è chiusa ($dF = d^2A = 0$ su entrambi gli emisferi),
analogamente al caso precedente, ma questa volta non è esatta perchè i due potenziali
sono definiti su insiemi differenti.\\

Nella regione di intersezione $U^0$ i due potenziali sono vincolati dalla condizione
di compatibilità \ref{eq:condcompatibility}, la trasformazione di gauge
$A^+ = A^- + n d \phi$. Si definiscono allora le funzioni di transizione \ref{}
che sono elementi di $\Phi \in U(1)$ che collegano le coordinate delle fibre
$e^{i\alpha^+} = \Phi e^{i\alpha^-}$. Su $U^0$ si ha fissata la coordinata polare
a $\theta = \pi/2$ e quindi le $\Phi$ sono funzioni della sola coordinata azimutale,
$ \Phi = e^{in\phi}$ con $n$ intero.\\

Le funzioni di transizione $\Phi$ sono una mappa dell'equatore $S^1$ nel gruppo
di gauge $U(1)$, sono quindi cammini in $U(1)$ (spiegare meglio), e possono essere
classificate in base alla classe di omotopia in $\pi_1(U(1))$ a cui appartengono.
L'intero $n$ assume il significato topologico di \emph{numero di avvolgimento}
della corrispondente classe di omotopia.\\

Come visto sopra, la topologia del monopolo è caratterizzata dalla classe di Chern
relativa a $c_1$
\begin{equation*}
%   \begin{aligned}
      c_1   = \frac{1}{2\pi} \int_{S^2} F
            = \frac{1}{2\pi} \left( \int_{U^+} dA^+ \int_{U^-} dA^- \right)
            = \frac{1}{2\pi} \int_{S^1} (A^+-A^-)
            = \frac{1}{2\pi} \int_0^{2\pi} nd\phi  = n
%   \end{aligned}
\end{equation*}
La carica magnetica coincide quindi con il primo numero di Chern. Per $n=0$ il
fibrato è banale e ha la forma $S^2 \times S^1$, mentre per $n=1$ si ha il
\emph{fibrato di Hopf}, di seguito descritto.

Si osserva infine che anche la funzione d'onda di una particella carica in un
campo di monopolo non può essere definita globalmente.
Si definiscono su $U^\pm$ le funzioni d'onda $\psi^\pm$ che su $U^0$ sono collegate
da $\psi^+ = \Phi \psi^-$.
%------------------------------------------------------------------------------%
%------------------------------------------------------------------------------%
\section{Fibrato di Hopf}
