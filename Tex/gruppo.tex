\section{Gruppo e azione di gruppo}
\begin{definition}{(Gruppo)}:
   Sia $G$ un insieme e $*$ un'operazione binaria su $G$. $G$ si definisce
   \emph{gruppo} con l'operazione $*$ se valgono le seguenti proprietà:
   \begin{enumerate}
      \item $ \forall f,g,h \in G \to f*(g*h) = (f*g)*h \quad$ associativa
      \item $ \exists e \in G $ t.c. $ e*g = g*e = g ,\quad \forall g \in G \quad$
       esistenza elemento neutro
      \item $ \forall a \in G \exists a^{-1} \in G $ t.c. $a*a^{-1} = a^{-1}*a = e \quad$
         esistenza dell'inverso
   \end{enumerate}
   Se vale anche la proprietà commutativa, il gruppo è detto \emph{abeliano}
   \begin{enumerate}\setcounter{enumi}{3}
      \item $ \forall g,h \in G \to g*h = h*g \quad$ commutativa
   \end{enumerate}
\end{definition}
Sovente si usa omettere il simbolo dell'operazione: $ g*h = gh $
%------------------------------------------------------------------------------%
\begin{definition}{(Azione di gruppo):}\label{def:groupaction}
   Sia $G$ un gruppo e $X$ un insieme non vuoto. \\
   Si definisce l'azione destra di $G$ su $X$
   una funzione $\phi : X \times G \to X$ che $(x,g) \mapsto \phi(x,g) = x \cdot g$
   con le seguenti proprietà:
   \begin{enumerate}
      \item $x \cdot e = x \quad$ identità
          ($e$ denota l'elemento neutro di $G$)
      \item $\forall g,h \in G \: , \: \forall x \in X \to
         x\cdot (gh) = (x \cdot g) \cdot h \quad $ compatibilità
   \end{enumerate}
   Si definisce l'azione sinistra di $G$ su $X$
   una funzione $\phi : G \times X \to X$ che $(g,x) \mapsto \phi(g,x) =  g \cdot x $
   con le seguenti proprietà:
   \begin{enumerate}
      \item $e \cdot x = x \quad$ identità
          ($e$ denota l'elemento neutro di $G$)
      \item $\forall g,h \in G \: , \: \forall x \in X \to
         (gh) \cdot x  = g \cdot (h \cdot x) \quad $ compatibilità
   \end{enumerate}
\end{definition}
La differenza tra azione destra e azione sinistra sta nell'ordine in cui $gh$
agisce sull'insieme, evidente per gruppi non abeliani.\\

L'azione di $G$ su $X$ si dice:
\begin{itemize}
   \item \emph{transitiva}: \quad $\forall x,y \in X \: \exists g \in G$ t.c. $g\cdot x = y$
   \item \emph{libera}: \quad Sia $g \in G$. Se $\exists x \in X$ t.c.$ g\cdot x = x
      \Rightarrow g$ è l'identità
\end{itemize}
%------------------------------------------------------------------------------%
\subsection{Gruppi e Algebre di Lie}
Si veda la sezione \ref{sec:vardiff} per una breve trattazione su varietà differenziali,
spazi tangenti e campi vettoriali.

\begin{definition}{(Gruppo di Lie):}
   Un gruppo di Lie $G$ è una varietà differenziale\footnote{Si veda la sezione
   \ref{sec:vardiff}}, dotata di di una struttura di gruppo in cui la
   moltiplicazione e l'inverso sono funzioni lisce. In altre parole è liscia la mappa
      $$ (x.y) \mapsto x^{-1}y \quad \forall x,y \in G$$
\end{definition}
La dimensione del gruppo equivale alla dimensione della varietà.
Esempi di gruppi di Lie sono $GL(n,\mathbb{R})$ e $GL(n,\mathbb{C})$ i gruppi
delle matrici quadrate $n \times n$ non singolari a coefficenti reali/complessi.\\
Vale il seguente Teorema che non verrà qui dimostrato.
\begin{theorem}
   Ogni sottogruppo chiuso $H$ di un gruppo di Lie $G$ è un sottogruppo di Lie
\end{theorem}
Che garantisce che $O(n), SO(n), SL(n,\mathbb{R})$ sono sottogruppi di Lie di
$GL(n,\mathbb{R})$.\\

Essendo un gruppo di Lie $G$ contemporaneamente gruppo e varietà differenziale, si
può definire l'azione di $G$ (come gruppo) su se stesso (come varietà differenziale).

\begin{definition}\label{def:adjrep}
   Sia $h \in G$ si definisce la \emph{rappresentazione aggiunta} di G come
   l'omomorfismo $ ad_a : G \to G $
   $$
      ad_h : g \mapsto h g h^{-1}
   $$
\end{definition}
E induce la mappa tra gli spazi tangenti $(ad_h)_* : T_g(G) \to T_{h^{-1}gh}$.
Poichè $ad_h e = e$ si può valutare la restrizione di $(ad_h)_*$ al solo $g=e$,
ottenendo una mappa di $T_e(G)$ in se stesso
$$
   Ad_h : T_e(G) \to T_e(G) \quad Ad_h := (ad_h)_* |_{T_e(G)}
$$

%------------------------------------------------------------------------------%
\begin{definition}{(Algebra di Lie):}\label{def:LieAlgebra}
   Un'algebra di Lie $\mathfrak{g}$ è uno spazio vettoriale (su un opportuno campo)
   dotato di un'operazione binaria $[,] : \mathfrak{g} \times \mathfrak{g} \to \mathfrak{g}$
   che rispetta le seguenti proprietà:
   \begin{enumerate}
       \item \emph{Bilinearità} \quad $[ax+by,z] = a[x,z]+b[y,z]$ \quad e \quad
          $[x,ay+bz] = a[x,y] + b[x,z]$ \\
          per tutti gli scalari $a,b$ e $\forall x,y,z \in \mathfrak{g}$
       \item \emph{Identità di Jacobi} \quad $\forall x,y,z \in \mathfrak{g} \quad
          [x,[y,z]] + [z,[x,y]] + [y,[z,x]] = 0$
       \item \emph{Anticommutatività} \quad $[x,y] = -[y,x] \quad
          \forall x,y \in \mathfrak{g} $
       \item \emph{Alternanza} \quad $[x,x]=0 \quad \forall x \in \mathfrak{g}$ \quad
          (discende dalla precedente)
   \end{enumerate}
\end{definition}
La dimensione dell'algebra di Lie è la sua dimensione come spazio vettoriale.\\
Un set di elementi dell'algebra $\mathfrak{g}$ si dice set di \emph{generatori}
se la sottoalgebra più piccola che lo contiene è $\mathfrak{g}$ stessa.\\

Ad ogni gruppo di Lie si può associare un'algebra di Lie.\\

Siano $a,g \in G$. Si definiscono la \textbf{traslazione destra} $R_a : G \to G$
e la \textbf{traslazione sinistra} $L_a : G \to G$
$$
   R_a(g) = ga \quad,\quad L_a(g) = ag
$$
$R_a$ e $L_a$ sono diffeomorfismi di $G$ in se stesso per costruzione e inducono
quindi le mappe sugli spazi tangenti $L_{a*} : T_g(G) \to T_{ag}(G)$ e
$R_{a*} : T_g(G) \to T_{ga}(G)$ (si veda \ref{}).\\

Dato un gruppo di Lie $G$ esiste una speciale classe di campi vettoriali
(si veda \ref{}) che sono invarianti sotto l'azione di gruppo\footnote{
Ciò non accade, ad esempio, con le varietà differenziali usuali, in cui non vi è
modo di evidenziare una classe privilegiata di campi vettoriali
}.\\

Sia X un campo vettoriale sul gruppo di Lie $G$. X si dice campo vettoriale
\textbf{invariante a sinistra} se vale $L_{a*}X|_g = X|_{ag}$ (analogamente,
invariante a destra).\\

Un vettore tangente all'identità $e$ del gruppo di Lie $V \in T_e(G)$
definisce un unico campo vettoriale $X_V$ su $G$ invariante a sinistra tramite
l'azione sinistra (analogamente a destra)
$$
   X_V|_g : = L_{g*} V \quad , \quad g \in G
$$
Poichè
$$
   X_V|_{ag} = (L_{ag})_* V = (L_a L_g)_* V = (L_a)_*(L_g)_* V = (L_a)_* X_V|_g
$$

Viceversa, un campo vettoriale $X_V$ invariante a sinistra (analogamente a destra)
definisce un unico vettore $V$ tangente all'identità in $G$
$$
   V := X_V|_e \quad \in T_e(G)
$$
\begin{definition}\label{def:GroupLieLlgebra}
  Si indica con $\mathfrak{g}$ l'insieme dei campi vettoriali su $G$ invarianti
  a sinistra (denotando con $\mathcal{X}(G)$ l'insieme dei campi vettoriali su G)
  $$
    \mathfrak{g} := \{ X_V \in \mathcal{X}(G) \mathrm{\: t.c. \:}
       L_{a*}X_V|_g = X_V|_{ag} \forall a \in G \}
  $$
\end{definition}
La mappa che a un vettore tangente all'identità in $G$ associa un campo vettoriale
invariante a sinistra
$$
   T_e(G) \to \mathfrak{g}  \quad V \mapsto X_V
$$
è un isomorfismo. \\
Si dimostra che $\mathfrak{g}$ è uno spazio vettoriale con l'operazione di
traslazione a sinistra (?), e quindi $\mathfrak{g} \cong T_e(G)$.
In particolare dim $\mathfrak{g}$ = dim $G$.\\

Resta da definire un'operazione di parentesi di Lie affinchè lo spazio dei campi
invarianti a sinistra sia un'algebra di Lie.\\
\begin{definition}\label{def:parentesiLie}
   Si definisce parentesi di Lie tra due campi $X,Y \in \mathcal{X}(G)$ l'operazione
   $[,] : \mathcal{X}(G) \times \mathcal{X}(G) \to \mathcal{X}(G)$ definita da
   $$
      [X,Y]f = X(Yf) - Y(Xf) \quad \forall f \in C^\infty(G)
   $$
\end{definition}
Si sottolinea che $Xf$ così scritto è una funzione liscia definita su $G$ da
$Xf : G \in \mathbb{R}$ che agisce $p \mapsto X_p[f]\in \mathbb{R}$\\

Si verifica immediatamente che l'operazione così definita rispetta le proprietà
definite in \ref{def:LieAlgebra}\footnote{Intuitivamente, $[X,Y]$ è il commutatore
dei campi $X$ e $Y$, ed è noto che un commutatore rispetti suddette proprietà.}\\

Siano $X,Y \in \mathcal{X}(G)$ e si fissino due punti $g, ag \in G$ dove
$ag = L_a g$. Applicando $L_{a*}$ a $[X,Y]$ si ha
$$
   L_{a*}[X,Y]|_g = [L_{a*}X|_g,L_{a*}Y|_g] = [X,Y]|_{ag} \in \mathfrak{g}
$$
quindi $\mathfrak{g}$ è chiuso rispetto all'operazione $[,]$ così definita.\\

\begin{definition}\label{}
   L'insieme $\mathfrak{g}$ dei campi vettoriali su $G$ invarianti a sinistra
   ($\mathcal{X}(G)$)
   dotato delle parentesi di lie \ref{def:parentesiLie} si definisce
   \textbf{Algebra di Lie} associata al gruppo di Lie $G$.
\end{definition}
L'algebra di Lie associata a un gruppo viene indicata con lo stesso nome del gruppo,
in carattere gotico minuscolo, ad esempio $SO(n)\to \mathfrak{so(n)}$.\\

\begin{definition}
   Identificando $\mathfrak{g}$ con $T_e(G)$ tramite l'isomorfismo dell'azione sinistra
   (indicato qua con $\lambda : T_e(G) \to \mathfrak{g}$) e richiamando la definizione
   \ref{def:adjrep}, si definisce la \textbf{mappa aggiunta} del gruppo di Lie $G$.
   $$
      Ad : G \times \mathfrak{g} \to \mathfrak{g} \quad , \quad
      (h,X) \mapsto \lambda \circ Ad_h \circ \lambda^{-1}(X)
   $$
   dove $h \in G$, $X \in \mathfrak{g}$.
   Applicando la definizione si trova che $Ad_a Ad_b = Ad_ab$ e $Ad_{a^{-1}} = Ad_a^{-1}$ .\\
\end{definition}

\begin{definition}\label{def:inducedvectfield}
   Sia $G$ un gruppo di Lie che agisce sulla varietà $M$ a sinistra. Sia $V \in T_e(G)$
   un vettore tangente all'identità di $G$.
   Si definisce il \textbf{campo vettoriale indotto} da $V$ il campo $V^\#$ su $M$
   $$
      V^\# |_p = \left. \frac{\dd}{\dd t} (\exp(tV)p) \right |_{t=0} \quad p \in M
   $$
   E si definisce quindi una mappa $\# : T_e(G) \to \mathcal{X}(M)$
   con $V \mapsto V^\#$
\end{definition}

Nel caso di un gruppo di Lie $G$ di matrici reali, l'algebra di Lie può essere
formulata in termini della funzione esponenziali di matrici, ovvero
$$
\mathfrak{g} = \{ A \in \mathrm{Mat}(n,\mathbb{C}) \: | \:
\forall t \in \mathbb{R} \:\to \exp (tA) \in G \}
$$
Gli elementi dell'algebra di Lie vengono detti \emph{generatori} del gruppo di Lie
e, come visto sopra, la struttura del gruppo $G$ è completamente determinata
dalle costanti di struttura. In questo caso in cui $G$ è un gruppo di matrici,
la parentesi di Lie è il commutatore di matrici e quindi le costanti di struttura
si dice sono determinate dalle regole di commutazione dei generatori $\{T_\mu\}$:
$$
   [T_\mu, T_\nu] = T_\mu T_\nu - T_\nu T_\mu
      = c_{\mu\nu}^{\hphantom{\mu\nu}\lambda} T_\lambda
$$

\textcolor{red}{(non ancora usato)}\\
Poichè si ha che $\mathfrak{g} \cong T_e(G)$ sia $\{ V_1,\dots,V_n \}$ una base
per $T_e(G)$. Questa definisce, tramite l'azione sinistra, un set di campi vettoriali
$\{ X_1, \dots, X_n \} \in \mathfrak{g}$ linearmente indipendenti in ogni punto $g \in G$
$$
   X_\mu |_g := L_{g*}V_\mu \quad \forall \mu = 1,\dots,n
$$
che per ogni $T_g(G)$ è una base. Poichè anche $[X_\mu,X_\nu]|_g \in \mathfrak{g}$
nel punto $g$, può essere sviluppato $ \forall g \in G$ in termini dei vettori
$\{X_\mu|_g\}$, e quindi si ha:
$$
   [X_\mu,X_\nu] = \sum_\lambda c_{\mu\nu}^{\hphantom{\mu\nu}\lambda} X_\lambda
$$
i coefficienti $c_{\mu\nu}^{\hphantom{\mu\nu}\lambda}$ si chiamano
\textbf{costanti di struttura}.
Si dimostrano essere indipendenti dal particolare $g \in G$ preso in considerazione,
e determinano quindi completamente la struttura del gruppo di Lie $G$ (Teorema di Lie).\\


%------------------------------------------------------------------------------%
