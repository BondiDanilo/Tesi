%------------------------------------------------------------------------------%
\begin{definition}(\emph{Prodotto di cammini}):
   Siano $\sigma,\tau$ cammini in $X$ tali che $\sigma(1) = \tau(0)$. Si definisce
   il \emph{cammino prodotto} o l'\emph{incollamento} $\sigma \cdot \tau$ in $X$:
   \begin{equation}
      \sigma \cdot \tau (t) = \begin{cases}
         \sigma(2t) & t \in [0,1/2] \\
         \tau(2t-1) & t \in (1/2,1]
      \end{cases}
   \end{equation}
\end{definition}

Se $\sigma,\tau$ sono lacci di base $x_0$ anche $\sigma \cdot \tau$ lo è, quindi
dati $[\sigma],[\tau] \in \pi_1(X,x_0)$ si definisce il loro \emph{prodotto}
la classe di $[\sigma \cdot \tau]$
$$ [\sigma][\tau] := [\sigma\cdot\tau] \in \pi_1(X,x_0) $$\\

Sia $x \in X$, si definisce $\varepsilon_x$ il \emph{cammino costante} nel punto $x$.
$$\varepsilon_x(t) = x \quad \forall t \in[0,1]$$

Sia $\sigma$ cammino in $X$, si definisce il \emph{cammino inverso}
$\sigma^{-1}(t) = \sigma(1-t)$ (si tratta dello stesso cammino, con senso
di percorrenza opposto).

\begin{proposition}
   Il gruppo fondamentale $\pi_1(X,x_0)$ è un \emph{gruppo} con l'operazione di
   prodotto tra due classi di equivalenza di cammini.
   L'elemento neutro è dato da $[\varepsilon_x]$,
   l'inverso è dato da $[\sigma]^{-1} = [\sigma^{-1}]$.
\end{proposition}
\begin{proof}
   La dimostrazione non è immediata. Si vedano le referenze.
\end{proof}

\begin{proposition}(\emph{Dipendenza del gruppo fondamentale dal punto base}):
   Siano $x_0, x_1 \in X$. Se esiste un cammino $\tau$ in $X$ che collega i due
   punti ($\tau(0)=x_0$ e $\tau(1)=x_1$) allora i gruppi fondamentali con base
   $x_0$ e $x_1$ sono isomorfi
      $$ \pi_1(X,x_0) \cong \pi_1(X,x_1) $$
   con isomorfismo dato da $\tau_\# : \pi_1(X,x_0) \to \pi_1(X,x_1)$
      $$  \tau_\# [\sigma] = [\tau^{-1} \cdot \sigma \cdot \tau] $$
\end{proposition}
\textcolor{red}{Intuitivamente, ridefinisci un cammino attaccandone un altro che
va dal punto nuovo a quello vecchio}
\begin{proof} Si vedano le referenze. \end{proof}
