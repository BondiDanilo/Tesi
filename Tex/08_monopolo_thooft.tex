\section{Monopolo di ’t Hooft–Polyakov}
Si è visto che il monopolo di Dirac può essere inserito in una teoria di gauge abeliana.
In questa sezione si vedrà che un modello non abeliano contiene soluzioni di tipo monopolo.
%------------------------------------------------------------------------%
\subsection{Modello di Georgi-Glashow}
Siano $\phi^a$ con $a=1,2,3$ tre campi scalari complessi e $\phi =
(\phi _1, \phi _2,\phi _3)$\footnote{
   Per non confondere la notazione, si indica con $a,b,c$ l'indice identificativo
   dei campi $\phi$ con $\mu,\nu,\rho$ l'indice spaziotemporale e con $i,j,k$
   l'indice solamente spaziale.
}.
Sia $\mathcal{M}$ lo spazio dei campi $\phi$, denominato spazio di \emph{isospin}.
Si consideri la seguente densità lagrangiana per $\phi$ ($\lambda$ e $v$ sono due
parametri).
$$
   \Ll = \frac{1}{2} \partial _\mu \phi ^a \partial ^\mu \phi^a
                + V(\phi)
               = \frac{1}{2} \partial _\mu \phi ^a \partial ^\mu \phi^a
                + \frac{\lambda}{4}( \phi^a \phi^a - v^2 )^2
$$
Si osserva che è invariante per rotazione del vettore $\phi$ nello spazio $\mathcal{M}$,
quindi ha una simmetria di tipo $SO(3)$\footnote{
  lo spazio $SO(3)$ è localmente isomorfo a $SU(2)$
} in $\mathcal{M}$.
In questo caso è più conveniente scegliere una rappresentazione aggiunta di $SU(2)$,
data dalle matrici ${T^a}$ che rispettano le regole di commutazione
$[T^a,T^b] = i \varepsilon_{abc} T^c$ e tali che Tr$(T^aT^b) = 1/2\delta_{ab}$.

\begin{equation}
   \begin{aligned}
      i T_1 & = \begin{pmatrix}
                   0  & 0  & 0 \\
                   0  & 0  & 1 \\
                   0  & -1 & 0 \\
                 \end{pmatrix} \, , &
      i T_2 & = \begin{pmatrix}
                   0  & 0  & -1 \\
                   0  & 0  & 0 \\
                   1  & 0  & 0 \\
                 \end{pmatrix} \, , &
      i T_3 & = \begin{pmatrix}
                   0  & 1  & 0 \\
                   -1 & 0  & 0 \\
                   0  & 0  & 0 \\
                \end{pmatrix} &
   \end{aligned}
\end{equation}
La simmetria di tipo $SU(2)$ è allora data dalla trasfomazione
$\phi \to \phi' = e^{iT^a\theta^a}$, per arbitrari $\theta^a \in \R ^3 $.\\

Si osserva immediatamente potenziale $V(\phi)$ ha un minimo quando il valore di
aspettazione del campo $\phi$ è pari a $v^2$. Un esempio di configurazione dei
campi che verifica tale condizione è:
   $$ \langle \phi \rangle = (0,0,v) $$
Si vede immediatamente che
\begin{equation}
   \begin{aligned}
      T_1 \langle \phi \rangle & \neq 0 \, , &
      T_2 \langle \phi \rangle & \neq 0 \, , &
      T_3 \langle \phi \rangle & = 0
   \end{aligned}
\end{equation}

Si ha così una \emph{rottura spontanea della simmetria} $SU(2)$ del modello, di cui l'unica
simmetria che si conserva è la simmetria di tipo $U(1)$ associata al generatore $T_3$.\\
%------------------------------------------------------------------------------%
\subsubsection{Descrizione del modello}
Il modello di Georgi-Glashow descrive l'accoppiamento dei 3 campi scalari
$\{\phi^a\}$ con il campo di gauge $A_\mu$, tramite la costante di accoppiamento $e$.
La densità lagrangiana del modello è
\begin{equation}
   \Ll = - \frac{1}{4} F^a_{\mu\nu}F^{a\mu\nu}
                 + \frac{1}{2} D^\mu\phi^a D_\mu \phi^a - V(\phi)
               = - \frac{1}{4} F^a_{\mu\nu}F^{a\mu\nu}
                 + \frac{1}{2} D^\mu\phi^a D_\mu \phi^a
                 - \frac{\lambda}{4}( \phi^a\phi^a - v^2 )^2
\end{equation}
Dove la derivata covariante è data da $D_\mu = \partial _\mu + ie A_\mu$.
Si richiama che $F_{\mu\nu} = F^a_{\mu\nu} T^a$, con
\begin{equation}
   \begin{aligned}
   F^a_{\mu\nu} & = \partial _\mu A_\nu^a - \partial _\nu A^a_\mu
                   - e \, \varepsilon_{abc} A_\mu^b A_\nu^c
   & \Rightarrow \quad
   F_{\mu\nu}  & = \partial _\mu A_\nu - \partial _\nu A_\mu + ie [A_\mu,A_\nu]
                = \frac{1}{ie}[D_\mu , D_\nu]
  \end{aligned}
\end{equation}

Le equazioni del moto per i campi sono allora
\begin{equation}\label{eq:nabeqmotion}
   \begin{aligned}
      D_\nu F ^{a\mu\nu} & = -e \, \varepsilon_{abc} \phi^b D^\mu \phi^c, &
      D_\mu D^\mu \phi^a & = - \lambda \phi^a (\phi^2 - v^2 )
   \end{aligned}
\end{equation}
dove $\phi^2 = \phi^a\phi^a$. Il tensore energia-impulso simmetrizzato è
\begin{equation}
   \begin{aligned}
      T_{\mu\nu} & =
         F^a _{\mu\rho}F^{a\rho}_\nu + (D_\mu \phi^a)(D_\nu \phi^a) - g_{\mu\nu} \Ll
   \end{aligned}
\end{equation}
E l'energia è data da
\begin{equation}
   E = \int_{\R^3} d^3x T_{00}
     = \int_{\R^3} d^3x \left[ \frac{1}{4} F^a_{\mu\nu}F^{a\mu\nu}
        + \frac{1}{2} (D_\mu \phi^a) (D^\mu \phi^a)
        + \frac{\lambda}{4}(\phi^2 - v^2)^2 \right]
\end{equation}
Analogamente all'esempio precedente, si vede immediatamente che l'energia è minima se
\begin{equation}
   \begin{aligned}
      \phi^2 & = v^2 \, , &
      F^a_{ij} & = 0 \, , &
      D_i \phi^a & = 0
   \end{aligned}
\end{equation}
Una configurazione di campi $\phi^a$ ad energia \emph{minima} è chiamata
\emph{stato di vuoto del sistema}. La costante $v$ rappresenta il valore di
aspettazione del campo sullo stato di vuoto. Essa quantifica la rottura spontanea
della simmetria $SU(2) \to U(1)$ del modello (se $v=0$ la simmetria è mantenuta).\\
Si evidenzia che il gruppo di simmetria $U(1)$ che si conserva corrisponde a
una rotazione nello spazio degli isospin attorno all'isovettore corrispondente
allo stato di vuoto $\phi$.\\

L'insieme delle configurazioni di campi che costituiscono minimi locali
del funzionale di energia è detto \emph{varietà di vuoto}
$\mathcal{W} \subset \mathcal{M}$. Per questo sistema la varietà di vuoto è una
sfera $S^2$ nello spazio degli isospin (definita da $\phi^2 = cost$).\\
%------------------------------------------------------------------------------%
\subsubsection{Classificazione delle soluzioni}
Lo spettro energetico perturbativo può essere calcolato dando una piccola fluttuazione
allo stato di vuoto $\phi = (0,0,v) \mapsto \phi' = (0,0,v + \varepsilon)$ e applicando
i soliti metodi di teoria delle perturbazioni.\\

La condizione $\phi^2 = v^2$ presenta una degenerazione molto ampia, quindi
lo spettro delle soluzioni di vuoto del modello di Georgi-Glashow è non banale.
Una classificazione delle soluzioni può essere data con considerazioni topologiche.
In particolare, se due soluzioni non possono essere deformate omotopicamente una
nell'altra appartengono a classi distinte. Ciò che caratterizzerà le soluzioni
sarà il comportamento asintotico (per $r \to \infty$) dei campi $\phi(\vec r)$.\\

Si definisce allora la \emph{sfera all'ininito} spaziale $S^2_\infty$ come il bordo di
$\R^3$ e si vuole caratterizzare il comportamento dei campi $\phi$ ristretti a
$S^2_\infty$. Le restrizioni $\phi^*$ dei campi sono funzioni continue da $S^2_\infty$
a $\mathcal{W}$.
Topologicamente sono funzioni $\phi : S^2 \to S^2$ e allora possono essere
classificate in base ai gruppi di omotopia, in particolare $\pi_2(S^2) \cong \Z$.\\

Le soluzioni del modello di Georgi-Glashow sono allora classificate topologicamente
in base al rappresentante della classe di $\pi_2(S^2)$ a cui appartengono.
Tale rappresentante è detto \textbf{numero di avvolgimento} $n \in \Z$.\\

Se si tenta di deformare una soluzione in un'altra con diverso numero di avvolgimento
il funzionale di energia diverge. Le soluzioni sono pertanto separate da barriere
infinite di energia e sono quindi stabili.\\

La soluzione banale è il caso $n = 0$ e corrisponde a un campo che asintoticamente
non dipende dalle componenti spaziali, della forma
 $$\phi \to \phi^a = (0,0,v).$$

Le soluzioni non banali sono anch'esse tali che all'infinito $|\phi| = v$,
ma hanno dipendenza esplicita dalle coordinate spaziali anche all'infinito
\begin{equation}\label{eq:nonabelianw1phi}
   \phi^a(r) \to \frac{vx^a}{r}
\end{equation}
Queste soluzioni sono caratterizzate da un numero di avvolgimento $n=1$ e vengono
dette \textbf{campo a riccio}.

%------------------------------------------------------------------------------%
\subsubsection{Carica magnetica}
Si consideri la configurazione non banale di campo a riccio. Si vuole porre la
condizione di annullamento della derivata covariante $D _\mu$ all'infinito, che
produce la condizione sull'andamento asintotico del campo di gauge

\begin{equation}\label{eq:nonabelianw1A}
   A_k^a (r) \to \frac{1}{e} \varepsilon_{abk} \frac{x^b}{r^2}
\end{equation}

Si definice quindi il tensore elettromagnetico gauge invariante
\begin{equation}
   \mathcal{F}_{\mu\nu} := \hat{\phi}^a F^a_{\mu\nu}
          + \frac{1}{e} \varepsilon_{abc} \hat{\phi}^a D_\mu\hat{\phi}^b D_\nu \hat{\phi}^c
\end{equation}
dove $\hat{\phi}^a = \phi^a/|\phi^a|$ è un campo normalizzato.

Per le configurazioni non banali, con andamento \ref{eq:nonabelianw1phi} del campo
e \ref{eq:nonabelianw1A} del potenziale, le equazioni di Maxwell diventano\footnote{
   Si veda \cite{nakahara} per la trattazione completa.
}
$$
   \partial ^\nu \tilde{F}_{\nu\mu} :=
       \frac{1}{2} \varepsilon_{\mu\nu\rho\sigma} \: \partial ^\nu F^{\rho\sigma} =
       \frac{1}{2 v^3 e} \: \varepsilon_{\mu\nu\rho\sigma}\varepsilon_{abc} \:
       \partial^\nu \phi^a \: \partial^\rho \phi^b \: \partial^\sigma \phi^c
    =: k_\mu
$$

Si osserva che il vettore $k_\mu$ si conserva ($\partial ^\mu k_\mu = 0$).
Si può allora definire la carica magnetica come
$$
   g = \int d^3 x \: k_0 = \int_{S^2_\infty} d^2s \: k_0
$$
Svolgendo i conti, si ottiene
\begin{equation}\label{eq:nonabelianquantumcharge}
   g = \frac{4\pi n}{e}
\end{equation}
dove $n \in \Z$ è il numero di avvolgimento del campo a riccio.
Questa condizione è l'analogo della condizione di quantizzazione
della carica alla Dirac \ref{eq:diracquantumcharge}.
%------------------------------------------------------------------------------%
\subsection{Ansatz di ’t Hooft–Polyakov}
Come soluzione del modello di Georgi-Glashow, ’t Hooft e Polyakov proposero questo Ansatz:
\begin{equation}
   \begin{aligned}
      \phi^a & = \frac{x^a}{er^2}H(r) \, ,&
      A_0^a & = 0 \, ,&
      A_k^a = \varepsilon_{abk}\frac{x^b}{er^2}(1-K(r))
   \end{aligned}
\end{equation}
con $H$ e $K$ funzioni da determinarsi inserendo queste espressioni nelle equazioni
del moto \ref{eq:nabeqmotion} con condizioni al contorno
\begin{equation}
   \begin{aligned}
      K(r) &\to 1 \, ,& H(r) &\to 0 \quad \mathrm{per} \quad r & \to 0 \\
      K(r) &\to 0 \, ,& H(r) &\to r \quad \mathrm{per} \quad r & \to \infty \\
   \end{aligned}
\end{equation}
e risolvendo numericamente le equazioni differenziali che ne derivano.\\

Sostituendo l'Ansatz nell'espressione della carica magnetica definita in precedenza
si ottiene
\begin{equation}
   g = \frac{4\pi}{e}
\end{equation}
che corrisponde a un monopolo di carica magnetica unitaria, ossia a un
numero di avvolgimento $n = 1$.\\
%------------------------------------------------------------------------------%

Riassumendo, il monopolo magnetico può essere meglio descritto da una teoria di
gauge non abeliana che si riduca a una sottoteoria abeliana tramite il processo
di rottura spontanea della simmetria. In questo modo viene risolto anche il
problema della malpositura delle energie di configurazione, riuscendo ad ottenere
configurazioni di campo stabili ad energia finita.\\
