\chapter{Monopoli in teorie di gauge non abeliane}
\textcolor{red}{(da scrivere)}\\
\textcolor{red}{Si veda Shnir, pag 106 per connessione tra non abeliano e dirac}\\
%We can look at this map from a different point of view. Note that the
%sphere S 3 coincides with the group manifold of the group SU(2). Thus we
%reformulate the Hopf map (3.84) in terms of elements of this group, the
%matrices U (theta, phi, alpha) in SU (2) (we use a slightly different parameterization
%in Appendix A, where general properties of the SU (2) group matrices are
%described)
%------------------------------------------------------------------------%
\section{Monopolo di ’t Hooft–Polyakov}
%------------------------------------------------------------------------%
