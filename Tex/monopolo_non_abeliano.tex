\chapter{Monopoli in teorie di gauge non abeliane}
\textcolor{red}{(da scrivere)}\\
\textcolor{red}{Si veda Shnir, pag 106 per connessione tra non abeliano e dirac}\\

%------------------------------------------------------------------------------%
Si vuole dare ora un'interpretazione differente alla mappa di Hopf $\pi : S^3 \to S^2$
definita in \ref{eq:hopfmap}. Innanzitutto si nota che la sfera $S^3$
coincide\footnote{
   Una matrice $U \in SU(2)$ ha la forma \ref{eq:matSU2} per due numeri complessi
   $z,w \in \mathbb{C} \: | \: zz^* + ww* = 1$ (condizione di determinante unitario).\\
   Se $z = z_1 + i \, z_2$ e $w = w_1 + i \, w_2$, la condizione diventa
   $zz^* + ww^* = z_1^2 + z_2^2 + w_1^2 + w_2^2 = 1$. \\
   Si vede immediatamente allora che la mappa $SU(2) \to S^3$ che
   $U(z,w) \mapsto (z_1,z_2,w_1,w_2)$ identifica i due spazi
}
con il gruppo $SU(2)$ inteso come varietà differenziale. Una matrice $U \in SU(2)$
può essere parametrizzata nella forma
\begin{equation}\label{eq:matSU2}
   U =
        \begin{pmatrix}
          z^* & w^* \\
          -w  & z   \\
        \end{pmatrix}
     =
       \begin{pmatrix}
          \cos \frac{\theta}{2} e^{-i\alpha} & \sin \frac{\theta}{2} e^{-i(\phi + \alpha)} \\
         -\sin \frac{\theta}{2} e^{i(\phi + \alpha)} & \cos \frac{\theta}{2} e^{i\alpha}  \\
       \end{pmatrix}
\end{equation}
Dove $z,w \in \mathbb{C}$ tali che $zz^* + ww^* = 1$
(affinchè $U^\dagger U = UU^\dagger = 1$), $\theta,\phi$ sono gli angoli delle
coordinate polari e $\alpha$ è il parametro ciclico del gruppo $U(1)$\footnote{
   $g \in U(1) \to g = e^{i\alpha}$
}.

Identificando $U = U(z,w) \in SU(2)$ con $\vec z = (z,w) \in \mathbb{C} \times \mathbb{C}$,
la mappa di Hopf può essere alternativamente vista\footnote{
   Si confronti con l'espressione \ref{eq:hopfmap}
}
come $\mathbb{C} \times \mathbb{C} \to S^2$ che $\vec z \mapsto x = (x_1,x_2,x_3)$
$$
   x_i = \vec z ^\dagger \sigma_i \vec z
$$
dove $\sigma_i$ sono le matrici di Pauli \ref{eq:paulimatrix}. Si ha allora
\begin{equation}\label{eq:hopfmap2}
   \sigma_i \cdot x_i = U^{-1} \sigma_3 U
\end{equation}
che corrisponde a una rotazione\footnote{
   Le matrici $SU(2)$ possono essere scritte $U = \sum_i x_i \sigma_i$ quindi le
   matrici di pauli possono essere intese come un sistma di assi (base dello spazio)
   di cui $\sigma_3$ è il terzo asse.
} nello spazio del gruppo $SU(2)$ del terzo asse $\sigma_3$

\begin{equation}
   \begin{aligned}
      U & \mapsto gU = e^{i \sigma_3 \alpha} U ,  g = e^{i\sigma_3 \alpha \in U(1)} \\
      \sigma_i x_i & \mapsto
         (U^{-1} g^*) \sigma_3 (g U)
         = U^{-1}e^{-i \sigma_3 \alpha}\, \sigma_3 \, e^{i \sigma_3 \alpha}U
         = U^{-1}\sigma_3 U =\sigma_i x_i
   \end{aligned}
\end{equation}
%------------------------------------------------------------------------------%
\section{Monopolo di Wu-Yang non Abeliano}
Il monopolo magnetico in una teoria di gauge abeliana può essere definito tramite
un potenziale non definito globalmente sulla varietà base. Questo però presenta
vari problemi quando lo si vuole estendere a una teoria quantistica\footnote{
   Si veda \cite{nakahara}.
}.
La situazione può essere risolta se si considera l'elettrodinamica, teoria
di gauge $U(1)$ abeliana, come parte di una teoria unificata più ampia. La versione
più semplice è una teoria di gauge $SU(2)$ che ha come sottogruppo il gruppo
di gauge dell'elettrodinamica $U(1)$. Una teoria di questo tipo prende il nome
di \emph{modello di Yang-Mills}. \\

Si consideri il potenziale di Dirac \ref{eq:diracpotential}, qui denotato con
$\tilde{\vec A}$, e si definisca il corrispondente quadrivettore $\tilde{A}_\mu$
$$
   \tilde{\vec A} = -g(1 + \cos\theta) \nabla \phi
   \quad \to \quad
   \tilde{A}_\mu = -g(1 + \cos\theta) \partial_\mu \phi
$$
Si definisce ora il \emph{potenziale vettore (di Dirac) non-Abeliano} $A_\mu$, \textcolor{red}{
immerso in $SU(2)$,} tramite una combinazione delle matrici di Pauli
$A_\mu := A_\mu^a \sigma_i/2$ con $a = 1,2,3$ e con coefficienti $A_\mu^a$ dati da

\begin{equation}\label{eq:wuyangpotential}
   \begin{aligned}
      A_\mu^1 &= 0, & A_\mu^2 &= 0, & A_\mu^3 &= \tilde{A}_\mu =
         -g(1 + \cos\theta) \partial_\mu \phi
   \end{aligned}
\end{equation}
Si ricorda che $\partial_\mu \phi$, dalla definizione, è singolare lungo l'asse $z$
$$
   \partial_\mu \phi = \frac{1}{r \sin\theta} (0,-\sin\phi,\cos\phi,0)
$$

Sia $U \in SU(2)$ come \ref{eq:matSU2}. La trasformazione di gauge del potenziale
vettore non-Abeliano è
\begin{equation}\label{eq:nonabeliangauge}
   A_\mu \quad \mapsto \quad A'-\mu = U A_\mu U^{-1} + \frac{i}{e} U \partial_i U^{-1}
\end{equation}
Sia $\alpha = 0$ in \ref{eq:matSU2} e si consideri la trasformazione di gauge con
matrice data dalla seguente matrice
\begin{equation}
   U = U(\theta,\phi) = e^{i \sigma_3 \frac{\phi}{2}} \, e^{i \sigma_2 \frac{\theta}{2}}
      \, e^{-i \sigma_3 \frac{\theta}{2}}
      = \begin{pmatrix}
         \cos \frac{\theta}{2} & -\sin \frac{\theta}{2} e^{-i \phi} \\
         \sin \frac{\theta}{2} e^{i \phi} & \cos \frac{\theta}{2}  \\
      \end{pmatrix}\footnote{
         l'angolo $\phi$ varia da $4\pi$ in corrispondenza di $\theta = 0$
         a $0$ per $\theta = \phi$. Per un gruppo $SU(N)$ $\phi$ è periodico di $2\pi N$
      }
\end{equation}
Inserendo $U$ nella trasformazione di gauge si ottiene (per le componenti spaziali
$k = 1,2,3$):
\begin{equation}
   A'_k = A_k^a \frac{\sigma _a}{2} = \epsilon_{ibc} \frac{r_b}{r^2}\frac{\sigma_c}{2}
\end{equation}
Questo potenziale è regolare ovunque tranne che nell'origine. Si è rimossa
la stringa di singolarità iniziale.\\

Il motivo è che la trasformazione di gauge stessa \ref{eq:nonabeliangauge} è singolare
e ha la stessa singolarità del potenziale $\tilde{A}_\mu$, infatti si ha che
\begin{equation}
   \begin{aligned}
      (U \partial _i U^{-1})_1 & = 0 \, , &
      (U \partial _i U^{-1})_2 & = 0 \, , &
      (U \partial _i U^{-1})_3 & = -(1 + \cos\theta) \partial _i \phi
   \end{aligned}
\end{equation}
L'esatta cancellazione delle singolarità è possibile solo se è soddisfatta la condizione
di quantizzazione della caria \ref{eq:diracquantumcharge}.\\

L'espressione \ref{eq:wuyangpotential} è la soluzione classica della teoria di Yang-Mills puramente
$SU(2)$, ad opera di Wu e Yang.\\

La connessione tra la soluzione \ref{eq:wuyangpotential} e il monopolo magnetico
è immediata inserendo l'espressione trovata nel tensore elettromagnetico non-Abeliano
\begin{equation}
   F_{\mu\nu} = \partial _\mu A_\nu - \partial _\nu A _\mu + i e [A_\mu,A_\nu]
              = \frac{1}{2} F_{\mu\nu}^a \sigma _a ,
\end{equation}
dove $F_{\mu\nu}^a = \partial _\mu A_\nu^a - \partial _\nu A_\mu^a$
ottenendo
\begin{equation}
   F^a_{ij} = \epsilon_{ijk} \frac{r^a r^k}{e r^4} \sim \frac{1}{r^2}
\end{equation}

La soluzione presenta però ancora alcune problematiche. Se si prova a calcolare
il flusso di questo campo su una superficie chiusa, si ottiene zero. Inoltre, si
hanno problemi di definizone dell'energia della configurazione.\\

L'accoppiamento del campo di gauge con il campo di Higgs e il meccanismo di
rottura spontanea della simmetria risolveranno questi probemi.
%------------------------------------------------------------------------%
