\subsection{Classi caratteristiche}
Riprendere la notazione della sezione precedente \\

Si è visto (esempi \ref{ex:cilindro} e \ref{ex:mobius}) che a uno stesso spazio
base e a una stessa fibra possono corrispondere diversi fibrati. Si vuole trovare
un modo per classificare le diverse tipologie di fibrati principali possibili dato uno
spazio base $M$ e un gruppo di struttura $G$.\\

Si consideri il seguente esempio. Sia $H^r(M)$ il r-gruppo di Coomologia di de Rham
di una varietà $M$, compatta e senza bordo $\partial M = 0$.
Sia $[\omega] \in H^r(M)$.\\
Un modo per classificare le classi di equivalenza si ottiene integrando una forma
nella stessa classe $[\omega]$. Si ricorda che esiste quindi una forma $\alpha$
tale che $\omega' = \omega + d\alpha$. Per il teorema di Stokes:
$$ \int_M \omega' = \int_M (\omega + \dd \alpha) = \int_M \omega
   + \int_{\partial M} \alpha = \int_M \omega $$
I due integrali si equivalgono perchè la varietà è senza bordo. Allora questo integrale
può essere usato come caratteristica, per classificare le classi di equivalenza.\\

In analogia con l'esempio appena citato, si vuole definire una caratteristica
analoga per la forma di curvatura $\Omega$. Una classe di equivalenza è formata
da tutte le 1-forme di connessione a cui corrisponde la stessa 2-forma di curvatura.\\
Si introduce ora il \emph{polinomio caratteristico} della forma $\Omega$ una quantità
invariante della algebra di Lie $\mathfrak{g}$ nella fibra
%
$$
   \mathrm{det}(1+\Omega) = 1 + \mathrm{tr}\Omega
      + \frac{1}{2}[ ( \mathrm{tr}\Omega )^2 - \mathrm{tr}(\Omega^2) ] + \dots
$$
%
che permette di definire una forma sullo spazio base $M$, detta \emph{forma di Chern}
$$ det\left(1 + \frac{\lambda}{2\pi}F\right) = \sum_{k=0}^\infty \lambda^k c_k $$
dove $1/2\pi$ è solo un fattore di normalizzazione. Si ricorda che $F$ è il pullback
di $\Omega$. I coefficienti di espansione $c_k$ sono forme chiuse invarianti su $M$,
la cui espressione esplicita per $k = 1,2$ è:

\begin{equation}
   \begin{aligned}
      c_1 &= \frac{1}{2\pi} \mathrm{tr} F &,
      c_2 & = \frac{1}{8\pi^2}[ \mathrm{tr} F \wedge \mathrm{tr} F
           - \mathrm{tr} (F \wedge F)]
   \end{aligned}
\end{equation}

I fattori del tipo $1/2\pi$ fanno si che sia normalizzato l'integrale sulla varietà
compatta senza bordo $M$

\begin{equation}
   \int_M c_k = n \quad , \quad n \in \Z
\end{equation}

Questi integrali sono detti \emph{classi caratteristiche di Chern}.\\

\begin{example}\label{ex:monopolechern}
   Si consideri un fibrato principale con $M = S^2$ e $G = U(1)$. Si hanno i due
   coefficienti $c_0 = 1$ (fibrato banale) e $c_1 = F/2\pi$. Di quest'ultimo si
   terra conto in \ref{}.\\
\end{example}

\begin{example}\label{ex:istantonchern}
   Si consideri un fibrato principale su $M = S^4$ con gruppo di struttura
   $G = SU(2)$. La forma di curvatura è a valori in $\mathfrak{su(2)}$ e può
   essere scritta localmente come $F = 1/2\sigma_a F^a$ dove $\sigma_a$ sono le
   matrici di Pauli \ref{eq:paulimatrix} con $a = 1,2,3$.

   Si ha quindi trF = 0 e i coefficienti $k = 0,1,2$ sono
   \begin{equation}
      \begin{aligned}
         c_0 & = 1 \, ,&
         c_1 & = 0 \, ,&
         c_2 & = \frac{1}{32\pi^2}[ F^a \wedge F^a ]
      \end{aligned}
   \end{equation}
   Il fibrato caratterizzato dalla seconda classe di Chern è noto come
   \emph{fibrato di istantone}.
\end{example}
