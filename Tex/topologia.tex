\chapter{Cenni preliminari di matematica}
\section{Topologia Generale e Algebrica}
Per una trattazione completa degli argomenti qua accennati si vedano le referenze.

%------------------------------------------------------------------------------%
%------------------------------------------------------------------------------%
\section{Definizioni base}

\begin{definition}{(Topologia)}
  Sia $X$ un insieme, è detta \emph{topologia} una collezione di sottoinsiemi
  di $X$ che rispettino i seguenti assiomi ($T = \{A_\alpha\} \subset X$):
     \begin{itemize}
          \item $\varnothing \in T$ e $X \in T$
          \item $ \bigcup\limits_\alpha A_\alpha \in T, \;
             \forall \alpha $ t.c. $A_\alpha \in T$
          \item $ A_\alpha \cap A_\beta \in T \;
             \forall A_\alpha, A_\beta \in T$
     \end{itemize}
\end{definition}
Si sottolinea che la seconda condizione richiede che l'unione \textbf {qualsiasi}
(finita, infinita, numerabile, non numerabile, etc) di elementi della topologia
appartienga alla topologia, mentre la terza richiede solamente che l'intersezione
di due elementi della topologia appartienga ad essa (è immediato generalizzare
a una qualsiasi intersezione \textbf{finita} di elementi della topologia)\\

Lo spazio $X$, dotato della topologia $T$ viene detto \textbf{spazio topologico}.
Gli elementi della topologia $A \in T$ vengono detti \textit{aperti} di $X$.\\

Piccoli esempi più commenti. Si veda \cite{sernesi} per ulteriori esempi di spazi
topologici \footnote{Capitoli 1 e 2}.\\
%------------------------------------------------------------------------------%
\begin{definition}{(Intorno)}
   Sia $(X,T)$ uno spazio topologico e $x \in X$. Un insieme $U \subset X$ è detto
   \emph{intorno di x} se esiste un aperto contenuto in $U$, contenente $x$.
   $$ \exists A \in T, \, A \subset U \:\mathrm{t.c.}\: x \in A \rightarrow U \:
   \mathrm{ \emph{intorno} \: di } \: x.$$
\end{definition}
%------------------------------------------------------------------------------%
\begin{definition}{(Base della topologia)}
   Sia $(X,T)$ spazio topologico, $x \in X$. Una \emph{Base} per la topologia $T$
   è una famiglia $\mathfrak{B}$ di aperti tale che ogni aperto $A \in T$
   è unione di insiemi di $\mathfrak{B}$
   $$ \forall A \in T \: A = \bigcup \limits_i B_i \: , \: \{B_i\} \in \mathfrak{B}$$
\end{definition}
%------------------------------------------------------------------------------%
%\begin{definition}{(Base di intorni)}
%   Sia $(X,T)$ spazio topologico, $x \in X$. Sia $N_x$ l'insieme di tutti gli
%   intorni di $x$. Una \emph{Base di intorni} è una famiglia $B_x$ di $x$
%   tale che per ogni intorno $U$ di $x$ esista un $B \in B_x$ contenuto in $U$
%   $$ \forall U \in T \: \exists B \in (B_x \subset N_x) \mathrm{\: t.c. \:} B \subset U $$
%\end{definition}
%------------------------------------------------------------------------------%
\begin{definition}{(Secondo assioma di Numerabilità)}
\label{def:IInumerabile}
   Lo spazio $X$ ha una base con cardinalità numerabile.
\end{definition}
Esempio di $\mathbb{R}^n$
%------------------------------------------------------------------------------%
 \begin{definition}{(Continuità)}
    Sia $f:X \to Y$ una funzione tra gli spazi topologici $X,Y$ dotati rispettivamente
    delle topologie $T_X, T_Y$ tale che la \emph{controimmagine} di ogni aperto in $Y$
    è un aperto in $X$
    $$\forall A_Y \in T_Y \rightarrow f^{-1}(A_Y) \in T_X$$
    allora $f$ è detta una funzione \emph{continua}
 \end{definition}
Per funzioni $\mathbb{R}^n \to \mathbb{R}^m$ questa definizione coincide con
la usuale definizione di continuità di Analisi Matematica (si veda \cite{soardi} per una
dimostrazione per funzioni $\mathbb{R} \to \mathbb{R}$).\footnote{Soardi, capitolo 7, sezione 7.3}\\

Se una funzione $f$ continua è invertibile e la sua inversa $f^{-1}$ è continua allora
$f$ è detta un \textbf{omeomorfismo}.
%------------------------------------------------------------------------------%
\begin{definition}{(Varietà topologica)}
\label{def:var_topologica}
   Sia $(X ,T)$ uno spazio topologico con le seguenti proprietà:
  \begin{itemize}
     \item (Proprietà di Hausdorff) Punti distinti di $X$ hanno intorni disgiunti
     $$\forall x_\alpha,x_\beta \in X, \: x_\alpha \neq x_\beta , \:
       \exists A_\alpha , A_\beta \in T \: (x_\alpha \in A_\alpha , \:
       x_\beta \in A_\beta ,\: \mathrm{intorni}), \: \mathrm{t.c.} \: A_\alpha \cap A_\beta = \varnothing $$
     \item (Localmente n-Euclideo) Ciascun punto di $X$ ha un intorno che è
     omeomorfo a un aperto di $\mathbb{R}^n$.
     $$
       \forall x \in X \: \exists U \in T \:,\: \exists\phi : U \to \mathbb{R}^n
          \mathrm{\: t.c.\:} \phi \mathrm{\: omeomorfismo}
     $$
     \item (Secondo assioma di Numerabilità) Lo spazio $X$ rispetta il secondo assioma di Numerabilità (\ref{def:IInumerabile}).
  \end{itemize}
  $X$ è allora detto una \emph{varietà topologica}
\end{definition}
Il numero $n$ è detto la \emph{dimensione} della varietà. Si può dimostrare che è unico.\\
La coppia $(U,\phi)$ è detta \emph{intorno coordinato} o \emph{carta}.
Una famiglia di carte $\mathcal{A} = \{ (U_\alpha , \phi_\alpha) \}$ è detta
\emph{atlante} se rispetta le seguenti proprietà:
\begin{itemize}
    \item L'insieme degli intorni $\{U_\alpha\}$ ricopre $X$ :
        $X \subset \bigcup \limits_\alpha U_\alpha$
    \item Gli intorni coordinati sono a due a due \emph{compatibili}: per ogni coppia
    di intorni coordinati
       $(U_\alpha , \phi_\alpha), (U_\beta , \phi_\beta) \mathrm{\:t.c.\:} \:
       U_\alpha \cap U_\beta \neq \varnothing $
    le funzioni di transizione (cambio di coordinate)
    \begin{equation*}
        \begin{split}
           \psi = \phi_\beta \circ \phi_\alpha &: \phi_\alpha(U_\alpha \cap U_\beta)
              \to \phi_\beta (U_\alpha \cap U_\beta) \\
           \psi^{-1} = \phi_\alpha \circ \phi_\beta &: \phi_\beta(U_\alpha \cap U_\beta)
              \to \phi_\alpha (U_\alpha \cap U_\beta) \\
        \end{split}
    \end{equation*}
    sono funzioni continue
\end{itemize}
Si veda \cite{sernesi} per esempi di varietà topologiche\\

Si sottolinea che una varietà topologica è uno spazio che può essere descritto
localmente con un sistema di coordinate euclidee e applicare quindi tutti i metodi
noti di Analisi.
Ad esempio si può richiedere che le funzioni coordinate siano funzioni differenziabili
di classe $C^{k}(U)$, per qualche $k$ (senza perdita di generalità, si richiedere che le funzioni siano $C^\infty$). Questo porta alla definizione di
\textbf{varietà differenziale} (si veda \ref{sec:vardiff}).\\

Cosa significa "in coordinate locali"

%------------------------------------------------------------------------------%
%------------------------------------------------------------------------------%
\subsection*{Teoremi}
dimostrazioni di invarianti topologici per omeomorfismo
