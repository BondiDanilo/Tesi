\section{Calcoli}
%------------------------------------------------------------------------------%
\subsection{Calcolo del potenziale vettore di Dirac}
\label{sec:dirac_potential}
Si consideri il campo classico di monopolo generato da una carica magnetica $g$ posta
nell'origine del sistema di riferimento, $\vec B : \R ^3 \to \R ^3$
    $$ \vec B = g\frac{\vec r}{r^3} = \frac{g}{r^2} \vec u _r $$
Si vuole trovare un potenziale vettore $\vec A : U \subset \R ^3 \to \R ^3$ tale che
$ B = \nabla \times \vec A $ in $U$. Si fa la seguente ipotesi sulla forma di $\vec A$:
   $$ \vec A = g \cdot a(\theta) \nabla \varphi $$
per un'opportuna funzione $a(\theta)$.
L'espressione del rotore in coordinate sferiche è:
$$
    \nabla \times \vec A =
         \frac{1}{r\sin\theta} \left(
              \frac{\partial}{\partial\theta}(A_\varphi\sin\theta) -
              \frac{\partial A_\theta}{\partial\varphi}
              \right) \vec u _r \\
         + \frac{1}{r} \left(
              \frac{1}{\sin\theta}\frac{\partial A_r}{\partial\varphi} -
              \frac{\partial}{\partial r}(r A_\varphi)
              \right)\vec u _\theta \\
         + \frac{1}{r} \left(
              \frac{\partial}{\partial r}(r A_\theta) - \frac{\partial A_r}{\partial\theta}
              \right)\vec u _\varphi
$$
ove i versori sono dati da \ref{eq:versors}. Inoltre
$$
   \nabla \varphi = \left( \frac{-\sin\varphi}{r\sin\theta},\frac{\cos\varphi}{r\sin\theta},0 \right)
$$
Si deve avere che $ \nabla \times \vec A = g \cdot \frac{1}{r^2} \vec u _r$,
ossia
\begin{equation*}
   \begin{dcases}
       \frac{1}{r\sin\theta} \left(
            \frac{\partial}{\partial\theta}(A_\varphi\sin\theta) -
            \frac{\partial A_\theta}{\partial\varphi}
            \right) = g \cdot \frac{1}{r^2} \\
       \frac{1}{r} \left(
            \frac{1}{\sin\theta}\frac{\partial A_r}{\partial\varphi} -
            \frac{\partial}{\partial r}(r A_\varphi)
            \right) = 0 \\
       \frac{1}{r} \left(
            \frac{\partial}{\partial r}(r A_\theta) - \frac{\partial A_r}{\partial\theta}
            \right) = 0
   \end{dcases}
\end{equation*}
Osservando che $A_\theta = \vec u _\theta \cdot \vec A $ e
$A_\varphi = \vec u _\varphi \cdot \vec A$ e inserendo $ \vec A = g \cdot a(\theta) \nabla \varphi $
\begin{equation*}
   \begin{aligned}
         A_\varphi & = (-\sin\varphi \vec u _x + \cos\varphi \vec u _y)
                       \cdot a(\theta) g \nabla \varphi \\
                   & = a(\theta)g\left( -\sin\varphi \frac{-\sin\varphi}{r\sin\theta}
                      + \cos\varphi \frac{\cos\varphi}{r\sin\theta}\right)\\
                   & = \frac{a(\theta)g}{r\sin\theta} \\
        A_\theta & = (\cos\theta \cos\varphi u_x + \cos\theta \sin\varphi u_y
                          - \sin\theta u_z) \cdot a(\theta) g \nabla \varphi \\
                 & = a(\theta) g \left( \frac{-\sin\varphi}{r\sin\theta} \cos\theta \cos\varphi
                          + \frac{\cos\varphi}{r\sin\theta} \cos\theta \sin\varphi \right) \\
                 & = 0
   \end{aligned}
\end{equation*}
Segue quindi che
$$
   \Rightarrow
   \frac{1}{\sin\theta} \frac{\partial}{\partial\theta}(A_\varphi \sin\theta)
      = \frac{1}{\sin\theta} \frac{g}{r} \frac{\partial a(\theta)}{\partial \theta}
      = \frac{g}{r}
$$
Da cui si ricava facilmente l'espressione di $a(\theta)$ integrando ambo i lati
$$
   \int_0 ^\theta \frac{\partial a(\theta')}{\partial \theta'} \dd \theta'
      = \int_0^\theta \sin\theta' \dd \theta' 
   \Rightarrow a(\theta) = -(\cos\theta+1) + cost  = -(\cos\theta+1)
$$
Ponendo a zero la costante arbitraria di integrazione.\\
%
Si ottiene allora:
$$
   \boxed{
          \vec A = -g(1 + \cos\theta) \nabla \varphi
                 = g(1 + \cos\theta)\left( \frac{\sin\varphi}{r\sin\theta},
                    \frac{-\cos\varphi}{r\sin\theta},0 \right)
   }
$$
È di immediata verifica che $\nabla \times \vec A = \vec B = \frac{g}{r^2} \vec u _r$.\\
%
Inoltre, si nota che
$$
   \nabla \varphi = \left( \frac{-\sin\varphi}{r\sin\theta},\frac{\cos\varphi}{r\sin\theta},0 \right)
      = \frac{1}{(r\sin\theta)^2}(-x,y,0) = \frac{1}{(r\sin\theta)^2}\vec u _\varphi
      = \frac{1}{r^2(1-\cos^2\theta)} \vec u _\varphi
$$
Allora
$$
   A = -g(1+\cos\theta)\nabla\varphi = -\frac{g}{r^2} \frac{1+\cos\theta}{1-\cos^2\theta} \vec u _\varphi
     = -\frac{g}{r^2} \frac{1}{1-\cos\theta} \vec u _\varphi
     = -\frac{g}{r} \frac{1}{r-r\cos\theta} \vec u _\varphi
     = -\frac{g}{r(r-z)} \vec u _\varphi
$$
Ciò giustifica la definizione della forma differenziale $ A : \R^3 \setminus \{0\}
\to \Omega(\R^3 \setminus \{0\})$ che mappa ogni punto $p \in \R^3 \setminus \{0\} , p = (x,y,z)$ nella forma
$$
    A_p = \frac{g}{r(r-z)}(x \dd  y - y \dd  x) = \frac{g}{r(r-z)} \dd  \varphi
$$
con $r = |p| = \sqrt{x^2+y^2+z^2}$.

%-------------------------------------------------------------------------------%

\subsection{Calcolo del flusso del campo regolarizzato}
\label{sec:flusso_regolarizzato}
Supponendo valga
$$
  \int_S \dd \sigma \lim_{\varepsilon \to 0} \vec B_\varepsilon = \lim_{\varepsilon \to 0} \int_S \dd \sigma \vec B_\varepsilon
$$
Sia $S = \{(x,y,z) \in \R^3 \setminus \{0\} : x^2 + y^2 \leq \varepsilon ^2\ ,-h<z<h\}$,
cilindro centrato attorno all'asse z di raggio $\varepsilon$ e altezza $2h$, con $h>0$.

\begin{equation*}
   \begin{split}
      \int_S \dd \sigma \tilde{\vec B}  &= \int_S \dd \sigma \left[\frac{g}{r^3}\vec r
         - 2g\varepsilon^2 \left( \frac{1}{r^2(x^2 + y^2 + \varepsilon^2)}
               + \frac{2}{(x^2 + y^2 + \varepsilon^2)^2}\right) \Theta(z)  \right] \\
         &= \int_S \dd \sigma \frac{g}{r^3}\vec r
            - \int_S \dd \sigma \left[ 2g\varepsilon^2 \left( \frac{1}{r^2(x^2 + y^2 + \varepsilon^2)}
                  + \frac{2}{(x^2 + y^2 + \varepsilon^2)^2}\right) \Theta(z) \right] \\
          &= \int_S \dd \sigma [1] - \int_S \dd \sigma [2]
   \end{split}
\end{equation*}

Il secondo integrale ha contributo non nullo solamente sulla faccia superiore del cilindro
$C = \{ (x,y,z) \in \R^3 \setminus \{0\} : x^2 + y^2 \leq \varepsilon^2, z = h \}$
poichè l'integranda è nulla per $z<0$ e il campo è parallelo all'asse z, quindi il flusso
attraverso le pareti del cilindro è nullo.
\begin{equation*}
   \begin{split}
       \int_S \dd \sigma [2] &= 2g\varepsilon^2 \int_C \dd \sigma
             \left( \frac{1}{r^2(x^2 + y^2 + \varepsilon^2)}
                + \frac{2}{(x^2 + y^2 + \varepsilon^2)^2} \right) \\
       &= 2g\varepsilon^2 \int_0^\varepsilon\rho\dd \rho \int_0^{2\pi} \dd \varphi
       \left( \frac{1}{r^2(x^2 + y^2 + \varepsilon^2)}
           + \frac{2}{(x^2 + y^2 + \varepsilon^2)^2}\right)
   \end{split}
\end{equation*}
I punti di $C$ hanno $r^2 = \varepsilon^2 + h^2$, $x = \rho \cos\varphi$, $y = \rho \sin\varphi$, quindi
\begin{equation*}
   \begin{split}
      &= 2g\varepsilon^2 \int_0^\varepsilon\rho\dd \rho \int_0^{2\pi} \dd \varphi
      \left( \frac{1}{(\varepsilon^2 + h^2)(\rho^2 + \varepsilon^2)}
          + \frac{2}{(\rho^2 + \varepsilon^2)^2} \right) \\
   \end{split}
\end{equation*}
Sia $u := \rho^2 + \varepsilon^2$ e $\dd u = 2\rho \dd \rho$
\begin{equation*}
   \begin{split}
      &= 4\pi g\varepsilon^2 \int_{\varepsilon^2}^{2\varepsilon^2} \dd u
         \left( \frac{1}{(\varepsilon^2 + h^2)u} + \frac{2}{u^2} \right)
      = 4\pi g\varepsilon^2
         \left( \frac{1}{\varepsilon^2 + h^2}( \log(2\varepsilon^2)-\log(\varepsilon^2) )
            - 2\left( \frac{1}{2\varepsilon^2} - \frac{1}{\varepsilon^2} \right) \right) \\
      &= 4\pi g\varepsilon^2
         \left( \frac{1}{\varepsilon^2 + h^2} \log\left( \frac{2\varepsilon^2}{\varepsilon^2}\right)
            - \frac{2}{\varepsilon^2}\left( \frac{1}{2} - 1 \right) \right)
      = 4\pi g
         \left( \frac{\varepsilon^2}{\varepsilon^2 + h^2} \log 2
            + \frac{\varepsilon^2}{\varepsilon^2} \right) \\
   \end{split}
\end{equation*}
Allora
$$
   \lim_{\varepsilon \to 0} \int_S \dd \sigma [2] = 4\pi g
      \lim_{\varepsilon \to 0} \left( \frac{\varepsilon^2}{\varepsilon^2 + h^2} \log 2
         + \frac{\varepsilon^2}{\varepsilon^2} \right) = 4\pi g
$$

%-----------------------------------------------------------------------------%
