\section{Calcoli}
%------------------------------------------------------------------------------%
\textcolor{red}{(si può togliere)}
\subsection{Calcolo del potenziale vettore di Dirac}
\label{sec:dirac_potential}
Si consideri il campo classico di monopolo generato da una carica magnetica $g$ posta
nell'origine del sistema di riferimento, $\vec B : \mathbb{R} ^3 \to \mathbb{R} ^3$
    $$ \vec B = g\frac{\vec r}{r^3} = \frac{g}{r^2} \vec u _r $$
Si vuole trovare un potenziale vettore $\vec A : U \subset \mathbb{R} ^3 \to \mathbb{R} ^3$ tale che
$ B = \nabla \times \vec A $ in $U$. Si osserva innanzitutto che per la simmetria
sferica del campo $\vec B$, il potenziale vettore può essere scritto come:
   $$ \vec A = g \cdot a(\theta) \nabla \phi $$
per un'opportuna funzione $a(\theta)$.
L'espressione del rotore in coordinate sferiche è:
$$
    \nabla \times \vec A =
         \frac{1}{r\sin\theta} \left(
              \frac{\partial}{\partial\theta}(A_\phi\sin\theta) -
              \frac{\partial A_\theta}{\partial\phi}
              \right) \vec u _r \\
         + \frac{1}{r} \left(
              \frac{1}{\sin\theta}\frac{\partial A_r}{\partial\phi} -
              \frac{\partial}{\partial r}(r A_\phi)
              \right)\vec u _\theta \\
         + \frac{1}{r} \left(
              \frac{\partial}{\partial r}(r A_\theta) - \frac{\partial A_r}{\partial\theta}
              \right)\vec u _\phi
$$
ove i versori sono dati da:
$$
\begin{cases}
    u_r = \sin\theta \cos\phi u_x + \sin\theta \sin\phi u_y + \cos\theta u_z\\
    u_\theta = \cos\theta \cos\phi u_x + \cos\theta \sin\phi u_y - \sin\theta u_z\\
    u_\phi = -\sin\phi u_x + \cos\theta u_y
\end{cases}
$$
Inoltre
$$
   \nabla \phi = \left( \frac{-\sin\phi}{r\sin\theta},\frac{\cos\phi}{r\sin\theta},0 \right)
$$
Si deve avere che $ \nabla \times \vec A = g \cdot \frac{1}{r^2} \vec u _r$,
ossia
$$
\begin{cases}
    \frac{1}{r\sin\theta} \left(
         \frac{\partial}{\partial\theta}(A_\phi\sin\theta) -
         \frac{\partial A_\theta}{\partial\phi}
         \right) = g \cdot \frac{1}{r^2} \\
    \frac{1}{r} \left(
         \frac{1}{\sin\theta}\frac{\partial A_r}{\partial\phi} -
         \frac{\partial}{\partial r}(r A_\phi)
         \right) = 0 \\
    \frac{1}{r} \left(
         \frac{\partial}{\partial r}(r A_\theta) - \frac{\partial A_r}{\partial\theta}
         \right) = 0
\end{cases}
$$
Osservando che $A_\theta = \vec u _\theta \cdot \vec A $ e
$A_\phi = \vec u _\phi \cdot \vec A$ e inserendo $ \vec A = g \cdot a(\theta) \nabla \phi $
$$
         A_\phi = (-\sin\phi \vec u _x + \cos\phi \vec u _y)
                       \cdot a(\theta) g \nabla \phi
                = a(\theta)g\left( -\sin\phi \frac{-\sin\phi}{r\sin\theta}
                      + \cos\phi \frac{\cos\phi}{r\sin\theta}\right)\\
                = \frac{a(\theta)g}{r\sin\theta}
$$
$$         A_\theta = (\cos\theta \cos\phi u_x + \cos\theta \sin\phi u_y
                          - \sin\theta u_z) \cdot a(\theta) g \nabla \phi
                    = a(\theta) g \left( \frac{-\sin\phi}{r\sin\theta} \cos\theta \cos\phi
                          + \frac{\cos\phi}{r\sin\theta} \cos\theta \sin\phi \right)
                    = 0
$$
Segue quindi che
$$
   \frac{1}{\sin\theta} \frac{\partial}{\partial\theta}(A_\phi \sin\theta)
      = \frac{1}{\sin\theta} \frac{g}{r} \frac{\partial a(\theta)}{\partial \theta}
      = \frac{g}{r}
$$
Da cui si ricava facilmente l'espressione di $a(\theta)$ integrando ambo i lati
$$
   \int_0 ^\theta \frac{\partial a(\theta')}{\partial \theta'} \mathrm{d}\theta'
      = \int_0^\theta \sin\theta' \mathrm{d}\theta' \Rightarrow\\
   \Rightarrow a(\theta) = -(\cos\theta+1) + cost  = -(\cos\theta+1)
$$
Ponendo a zero la costante arbitraria di integrazione.\\
%
Si ottiene allora:
$$
   \boxed{
          \vec A = -g(1 + \cos\theta) \nabla \phi
                 = g(1 + \cos\theta)\left( \frac{\sin\phi}{r\sin\theta},
                    \frac{-\cos\phi}{r\sin\theta},0 \right)
   }
$$
È di immediata verifica che $\nabla \times \vec A = \vec B = \frac{g}{r^2} \vec u _r$.\\
%
Inoltre, si nota che
$$
   \nabla \phi = \left( \frac{-\sin\phi}{r\sin\theta},\frac{\cos\phi}{r\sin\theta},0 \right)
      = \frac{1}{(r\sin\theta)^2}(-x,y,0) = \frac{1}{(r\sin\theta)^2}\vec u _\phi
      = \frac{1}{r^2(1-\cos^2\theta)} \vec u _\phi
$$
Allora
$$
   A = -g(1+\cos\theta)\nabla\phi = -\frac{g}{r^2} \frac{1+\cos\theta}{1-\cos^2\theta} \vec u _\phi
     = -\frac{g}{r^2} \frac{1}{1-\cos\theta} \vec u _\phi
     = -\frac{g}{r} \frac{1}{r-r\cos\theta} \vec u _\phi
     = -\frac{g}{r(r-z)} \vec u _\phi
$$
Ciò giustifica la definizione della forma differenziale $ A : \mathbb{R}^3 \setminus \{0\}
\to \Omega(\mathbb{R}^3 \setminus \{0\})$ che mappa ogni punto $p \in \mathbb{R}^3 \setminus \{0\} , p = (x,y,z)$ nella forma
$$
    A_p = \frac{g}{r(r-z)}(x\mathrm{d}y-y\mathrm{d}x) = \frac{g}{r(r-z)}\mathrm{d}\phi
$$
con $r = |p| = \sqrt{x^2+y^2+z^2}$.

%-------------------------------------------------------------------------------%

\subsection{Calcolo del flusso del campo regolarizzato}
\label{sec:flusso_regolarizzato}
Supponendo valga (dimostrare)
$$
  \int_S \mathrm{d}\sigma \lim_{\epsilon \to 0} \vec B_\epsilon = \lim_{\epsilon \to 0} \int_S \mathrm{d}\sigma \vec B_\epsilon
$$
Sia $S = \{(x,y,z) \in \mathbb{R}^3 \setminus \{0\} : x^2 + y^2 \leq \epsilon ^2\ ,-h<z<h\}$,
cilindro centrato attorno all'asse z di raggio $\epsilon$ e altezza $2h$, con $h>0$.

\begin{equation*}
   \begin{split}
      \int_S \mathrm{d}\sigma \tilde{\vec B}  &= \int_S \mathrm{d}\sigma \left[\frac{g}{r^3}\vec r
         - 2g\epsilon^2 \left( \frac{1}{r^2(x^2 + y^2 + \epsilon^2)}
               + \frac{2}{(x^2 + y^2 + \epsilon^2)^2}\right) \Theta(z)  \right] \\
         &= \int_S \mathrm{d}\sigma \frac{g}{r^3}\vec r
            - \int_S \mathrm{d}\sigma \left[ 2g\epsilon^2 \left( \frac{1}{r^2(x^2 + y^2 + \epsilon^2)}
                  + \frac{2}{(x^2 + y^2 + \epsilon^2)^2}\right) \Theta(z) \right] \\
          &= \int_S \mathrm{d}\sigma [1] - \int_S \mathrm{d}\sigma [2]
   \end{split}
\end{equation*}

Il secondo integrale ha contributo non nullo solamente sulla faccia superiore del cilindro
$C = \{ (x,y,z) \in \mathbb{R}^3 \setminus \{0\} : x^2 + y^2 \leq \epsilon^2, z = h \}$
poichè l'integranda è nulla per $z<0$ e il campo è parallelo all'asse z, quindi il flusso
attraverso le pareti del cilindro è nullo.
\begin{equation*}
   \begin{split}
       \int_S \mathrm{d}\sigma [2] &= 2g\epsilon^2 \int_C \mathrm{d}\sigma
             \left( \frac{1}{r^2(x^2 + y^2 + \epsilon^2)}
                + \frac{2}{(x^2 + y^2 + \epsilon^2)^2} \right) \\
       &= 2g\epsilon^2 \int_0^\epsilon\rho\mathrm{d}\rho \int_0^{2\pi} \mathrm{d}\phi
       \left( \frac{1}{r^2(x^2 + y^2 + \epsilon^2)}
           + \frac{2}{(x^2 + y^2 + \epsilon^2)^2}\right)
   \end{split}
\end{equation*}
I punti di $C$ hanno $r^2 = \epsilon^2 + h^2$, $x = \rho \cos\phi$, $y = \rho \sin\phi$, quindi
\begin{equation*}
   \begin{split}
      &= 2g\epsilon^2 \int_0^\epsilon\rho\mathrm{d}\rho \int_0^{2\pi} \mathrm{d}\phi
      \left( \frac{1}{(\epsilon^2 + h^2)(\rho^2 + \epsilon^2)}
          + \frac{2}{(\rho^2 + \epsilon^2)^2} \right) \\
   \end{split}
\end{equation*}
Sia $u := \rho^2 + \epsilon^2$ e $\mathrm{d}u = 2\rho \mathrm{d}\rho$
\begin{equation*}
   \begin{split}
      &= 4\pi g\epsilon^2 \int_{\epsilon^2}^{2\epsilon^2} \mathrm{d}u
         \left( \frac{1}{(\epsilon^2 + h^2)u} + \frac{2}{u^2} \right)
      = 4\pi g\epsilon^2
         \left( \frac{1}{\epsilon^2 + h^2}( \log(2\epsilon^2)-\log(\epsilon^2) )
            - 2\left( \frac{1}{2\epsilon^2} - \frac{1}{\epsilon^2} \right) \right) \\
      &= 4\pi g\epsilon^2
         \left( \frac{1}{\epsilon^2 + h^2} \log\left( \frac{2\epsilon^2}{\epsilon^2}\right)
            - \frac{2}{\epsilon^2}\left( \frac{1}{2} - 1 \right) \right)
      = 4\pi g
         \left( \frac{\epsilon^2}{\epsilon^2 + h^2} \log 2
            + \frac{\epsilon^2}{\epsilon^2} \right) \\
   \end{split}
\end{equation*}
Allora
$$
   \lim_{\epsilon \to 0} \int_S \mathrm{d}\sigma [2] = 4\pi g
      \lim_{\epsilon \to 0} \left( \frac{\epsilon^2}{\epsilon^2 + h^2} \log 2
         + \frac{\epsilon^2}{\epsilon^2} \right) = 4\pi g
$$

%-----------------------------------------------------------------------------%
