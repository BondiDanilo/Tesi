\chapter{Teorie di gauge}\label{cap:gauge}

Una teoria di gauge è una teoria di campo la cui lagrangiana è invariante sotto
l'azione di un gruppo di Lie\footnote{
   Oltre che Lorentz-invariante.
   Si vedano le definizioni \ref{def:groupaction} e \ref{def:liegroup}
   (azione di gruppo e gruppo di Lie, rispettivamente).
} $G$, denominato \emph{gruppo di gauge}. Se il gruppo di gauge è un gruppo (non)
abeliano la teoria di gauge viene detta (non) abeliana.\\

Una simmetria della lagrangiana (ossia  l'azione di un gruppo di gauge) è detta
\emph{simmetria globale} se non dipende dal punto dello spazio-tempo in cui è applicata.\\

Si vuole costruire degli invarianti di gauge, da utilzzare nella lagrangiana,
per una teoria che descrive l'accoppiamento di un campo scalare $\phi$ con un campo
di gauge $A_\mu$.

Sia $g \in G$ e $\rho(g)$ una sua
rappresentazione, che si suppone unitaria senza perdita di generalità.
In seguito alla trasformazione $\phi \mapsto  \phi' = \rho(g) \cdot \phi$,
si vuole valutare come trasformano le derivate del campo $\partial _\mu \phi$.
Se $g$ è una simmetria globale\footnote{
   Abuso di notazione. Si sta identificando un singolo elemento del gruppo, con
   l'azione di gruppo valutata in quell'elemento.
} (indipendente dallo spaziotempo), si ha

$$
   \partial _\mu [\rho(g) \phi]
      = \partial _\mu [\rho(g)] \phi + \rho(g) \partial _\mu [\phi]
      = \rho(g) \partial _\mu [\phi]
$$

Allora è di immediata verifica che, grazie alla unitarietà di $\rho$, le seguenti
quantità\footnote{Prodotti interni.} sono gauge-invarianti:

\begin{equation*}
   \begin{aligned}
      \phi & \cdot \phi \,,& \quad
      \partial _\mu \phi & \cdot \partial ^\mu \phi \,,& \quad
      \phi & \cdot \partial _\mu \phi
   \end{aligned}
\end{equation*}

Si consideri, ad esempio, la seguente lagrangiana per un campo scalare complesso
$\phi : \R \times \R^3 \to \C$ (dove $V$ è un generico funzionale, il potenziale).
$$
   \Ll [\phi] = \partial _\mu \phi^*  \partial ^\mu \phi + V[\phi ^* \phi]
$$
È invariante per una trasformazione di fase globale del campo, ossia in seguito
alla trasformazione
$
  \phi \mapsto \phi' = e^{i \alpha}\phi
$
(dove $\alpha \in \R$) la lagrangiana non cambia.
$$
   \Ll [\phi'] = (\partial _\mu \phi')^*  \partial ^\mu \phi' + V[(\phi')^* \phi']
                = \partial _\mu  e^{-i \alpha}\phi^*  \partial ^\mu  e^{i \alpha}\phi
                   +  V[e^{-i \alpha}\phi ^*  e^{i \alpha}\phi]
                = \partial _\mu \phi^*  \partial ^\mu \phi + V[\phi ^* \phi]
                = \Ll [\phi]
$$

Un altro esempio di simmetria globale è, nella meccanica Newtoniana, una
trasformazione di Galileo tra due sistemi inerziali (es: una rotazione degli assi):
in seguito a tale trasformazione non cambia la descrizione fisica del fenomeno,
ma solamente i "numeri" che ogni osservatore usa come coordinate. È quindi importante
evidenziare quali siano le trasformazioni tra un sistema di coordinate e l'altro.\\

In relazione alla lagrangiana scritta in precedenza, si può considerare una trasformazione
di fase in cui $\alpha = \alpha(x,t)$ è una generica funzione delle coordinate
$\phi \mapsto \phi' = e^{i\alpha(x,t)}\phi$. Se la lagrangiana è invariante per
trasformazione di questo tipo, tale trasformazione è detta \emph{simmetria locale}
o simmetria di gauge.\\

Il principio cardine delle teorie di gauge è allora promuovere le simmetrie globali
di una lagrangiana a simmetrie locali (simmetrie che possano essere applicate
solamente nell'intorno di un punto, senza affligere il resto dello spazio), e studiare
i casi in cui queste si conservano come simmetrie della teoria. \\

Per simmetrie locali (dove $g = g(x)$), la derivata $\partial _\mu$ non trasforma
più in maniera omogenea\footnote{
    Si veda Derivata covariante e trasporto parallelo, sezione \ref{sec:covariantderivative}.
}, ossia per $\phi \mapsto \phi'(x) = \rho(g(x))\phi(x)$:
$$
   \partial _\mu [\, \rho(\, g(x) \,) \, \phi(x)\, ]
       \neq  \rho(\, g(x) \,) \partial _\mu [\, \phi(x) \,]
$$
Si sostituisce allora alla derivata tradizionale la derivata covariante $D _\mu$,
definita in maniera tale che
\begin{equation}\label{eq:covdercondition}
   D _\mu[\, \rho(\, g(x) \,) \phi(x) \,] = \rho(\, g(x) \,) D _\mu[\, \phi(x) \,]
\end{equation}
Una derivata covariante così definita si costruisce nel modo seguente.\\

Si consideri l'algebra di Lie $\mathfrak{g}$ associata al gruppo di gauge e sia
$\{ t^a \}$ una base ($a = 1, \dots, \mathrm{dim}\mathfrak{g}$). Ad ogni generatore
$ t^a $ è associato un campo di gauge $ A^a_\mu $. Si costruisce allora $ D _\mu $ tramite
combinaizione lineare dei campi di gauge, dove $q$ è la costante di accoppiamento della
teoria.
\begin{equation}
   D _\mu := \partial _\mu + q \, A^a_\mu t^a
\end{equation}
Si può definire il campo matriciale $A _\mu = (A^a_\mu t^a)$. Dalla condizione
\ref{eq:covdercondition} si ottiene allora una condizione per la trasformazione
 $A _\mu \mapsto A' _\mu$, che risulta essere\footnote{ Il calcolo per ricavare
la condizione è immediato.}
\begin{equation}\label{eq:gaugetrasform}
   A'_\mu = g A_\mu g^{-1} + \frac{1}{q} g \partial _\mu g^ {-1}
\end{equation}

%------------------------------------------------------------------------------%
\section{Caso abeliano: Elettrodinamica classica}
Si vuole descrivere ora l'accoppiamento di un campo complesso $\phi$ con il campo
elettromagnetico (si pensi ad esempio alla funzione d'onda di una particella carica).\\
Si consideri la lagrangiana $\Ll$ definita in precedenza, che si è già visto
essere invariante per trasformazioni \emph{globali} di fase $\phi \mapsto e^{iq\alpha}\phi = g\phi$,
dove il parametro $q \in \R$ è la costante di accoppiamento della teoria
(in questo caso la carica elettrica). Trasfornazioni di questo tipo appartengono
al gruppo di Lie $U(1)$.
%
\begin{equation*}
  \begin{aligned}
     \Ll [\phi] & = \partial _\mu \phi^*  \partial ^\mu \phi + V[\phi ^* \phi] &
     & \Rightarrow \Ll[\phi] = \Ll[\phi'] & ,\quad
     \phi &\mapsto \phi' = e^{i\alpha}\phi
  \end{aligned}
\end{equation*}

Si vuole ora promuovere la simmetria globale appena definita a simmetria locale
$\alpha \mapsto \alpha(x,t)$ e richiedere che la lagrangiana rimanga invariata
per trasformazione di fase locale.
Si sostituisce la derivata tradizionale $\partial _\mu$ con la derivata covariante
definita da
%
$$
   D _\mu := \partial _\mu -iq A _\mu
$$
dove il campo $A _\mu : \R \times \R^3 \to \R \times \R^3$ è il potenziale
di gauge (in questo caso, il potenziale elettromagnetico). La lagrangiana
gauge-invariante diventa:
$$
   \Ll ' =  D _\mu \phi^*  D ^\mu \phi + V[\phi ^* \phi].
$$

La richiesta di invarianza della lagrangiana per trasformazione
di fase locale si traduce nella richiesta che la derivata covariante $D _\mu$
sia invariante. $D _\mu \mapsto D' _\mu = g^{-1} D _\mu g = D _\mu $.

\begin{equation}
   \begin{aligned}
      iq \, \partial _\mu \alpha(x,t) + (\partial _\mu - iq A _\mu)
         & = (\partial _\mu -iq A' _\mu) \Rightarrow \\
      \Rightarrow A' _\mu = A _\mu - \partial _\mu  \alpha(x,t)
         & = g ^{-1} A _\mu \, g + \frac{i}{q} \, g \, \partial _\mu g
   \end{aligned}
\end{equation}

che si traduce nella condizione che il potenziale $A _\mu$ trasformi secondo la
trasformazione di gauge sopra scritta.
Si sottolinea che nell'ultimo passaggio si è potuto seplificare $g$ e $ g^{-1} $
perchè il gruppo di gauge è abeliano ed è stato possibile commutare $ A _\mu $
con $g$. Ciò non può accadere se il gruppo di gauge non è abeliano.  \\

Se si calcola il tensore energia impulso della lagrangiana così scritta (si indica
$\phi| ^{\mu} = \partial \phi / \partial x_\mu$ e
$\phi| _{\mu} = \partial \phi / \partial x^\mu$ ), si vede immediatamente che non si
conserva\footnote{
   $\eta _{\mu\nu} = \mathrm{diag}(-1,1,1,1)$ è il tensore metrico Minkowskiano
}.

\begin{equation}
   \begin{aligned}
      T'_{\mu\nu} & = \frac{\partial \Ll'}{\partial \phi|^{\mu}}\phi|_{\nu}
      - \eta _{\mu\nu} \Ll' &
      \Rightarrow \partial ^\mu T'_{\mu\nu} & \neq 0
   \end{aligned}
\end{equation}

Occorre allora tenere conto anche della dinamica del campo elettromagnetico, costruendo
la lagrangiana per l'accoppiamento del campo $\phi$ con il campo elettromagnetico
che è Lorentz-invariante, con simmetria di gauge $U(1)$ e che conserva energia e
quantità di moto\footnote{
   Si osserva che definendo la lagrangiana del campo elettromagnetico libero
   $
      \Ll _{em} = - \frac{1}{4}F^{\mu\nu} F_{\mu\nu}
   $
   si ottengono come equazioni del moto esattamente le equazioni di Maxwell nel vuoto.
}.

\begin{equation}
\Ll = -\frac{1}{4} F^{\mu\nu}F_{\mu\nu}
+ \frac{1}{2} (D^\mu \phi)^* D _\mu \phi - V[\phi^* \phi]
\end{equation}

Dove $F_{\mu\nu}$ è il \textbf{tensore elettromagnetico} definito da

$$
   F_{\mu\nu} = \partial _\mu A _\nu - \partial _\nu A _\mu
$$

%------------------------------------------------------------------------------%
\section{Caso non abeliano}
Se il gruppo di gauge non è abeliano i generatori $t^a$ non commutano: $[t^a,t^b] \neq 0$.
Inoltre, $[t^a,t^b]$ è ancora un elemento dell'algebra $\mathfrak{g}$, quindi può a sua
volta essere espresso rispetto alla base dei generatori, tramite i coefficienti
$C^{ab}_{\hphantom{ab}c}$ detti costanti di struttura
$$
   [t^a,t^b] = C^{ab}_{\hphantom{ab}c} \, t^c
$$
La struttura del gruppo di gauge è quindi determinata dalle regole di commutazione
dei generatori, che vengono solitamente scelti tali da rispettare la condizione di
normalizzazione Tr$(t^a t^b) = 1/2 \delta_{ab}$.\\

Poichè il gruppo di gauge è non commutativo, nella trasformazione di gauge
\ref{eq:gaugetrasform}, non è possibile semplificare $g$ e $g^{-1}$
$$
   A'_\mu = g A_\mu g^{-1} + \frac{1}{q} g \partial _\mu g^ {-1}.
$$

Di conseguenza il tensore elettromagnetico, se definito come nel caso abeliano,
non può rispettare la corretta regola di trasformazione $F_{\mu\nu}
\mapsto F'_{\mu\nu} = g F_{\mu\nu} g^{-1}$ (di immediata verifica). Occorre quindi
correggerlo per un termine che tiene conto della commutazione dei generatori. Si
definisce allora il \textbf{tensore elettromagnetico non abeliano}:

\begin{equation}
   F _{\mu\nu} := \partial _\mu A_\nu - \partial _\nu A_\mu + g \, [A_\mu,A_\nu]
\end{equation}

Ricordando che $ A_\mu = A^a_\mu t^a $, le tre componenti del tensore matrciale sopracitato
sono

\begin{equation}
   F_{\mu\nu}^a = \partial _\mu A^a_\nu
       - \partial _\nu A^a_\mu + g \,C^{a}_{\hphantom{a}bc} A^b_\mu A^c_\nu
\end{equation}

Il termine corretto da inserire nella lagrangiana per ottenere l'invarianza di gauge
è, analogamente al precedente caso abeliano:
$$
   \mathrm{Tr}(F_{\mu\nu}F^{\mu\nu})
$$
Seguendo la costruzione delineata ad inizio capitolo, si può allora costruire
un esempio di lagrangiana per una teoria di gauge non abeliana:
\begin{equation}\label{eq:nonabelianlagrangian}
  \Ll = -\frac{1}{4} \, F^a_{\mu\nu}F^{a\mu\nu}
        + \frac{1}{2} \, D^\mu \phi^a D_\mu \phi^a
        - V[\, \phi^a \phi^a \,]
\end{equation}\\

%------------------------------------------------------------------------------%

\section{Formalismo dei fibrati}
Matematicamente, una teoria di gauge è descritta da un fibrato principale\footnote{
   Cfr. definzione \ref{def:principalbundle}.
}, in cui
la varietà di base $M$ è (ad esempio) lo spaziotempo $\R\times\R^3$ con metrica
Minkowskiana ($ \eta _{\mu\nu} = \mathrm{diag}(-1,1,1,1$)) e la fibra $G$ è il
gruppo di gauge.\\
Data una simmetria globale della lagrangiana, ossia che agisce in maniera rigida su
tutto lo spazio
come la simmetria di fase globale $e^{i\alpha}$ nel caso abeliano,
l'assegnazione di una simmetria locale
corrisponde alla scelta di un ricoprimento $\{U_i\}$ della varietà
e di sezioni locali $\phi_i$ sul fibrato\footnote{
  Cfr. definizione \ref{def:sectionbundle}.
}. Queste, nelle regioni di intersezione dei rispettivi domini ($U_i \cap U_j$),
sono legate dalle fuzioni di transizione \ref{eq:transfunctions},
sono dette \emph{trasformazioni di gauge}.\\
In analogia all'esempio abeliano, occore definire delle
funzioni $\alpha_{ij}(x,t)$ in maniera tale che le funzioni di transizione siano
date da4
$\Phi_{ij} = e^{i\alpha_{ij}(x,t)}$.\\

Assegnata una 1-forma di connessione $\omega$ sul fibrato e delle sezioni locali
$s_i$, i pullback di $\omega$ tramite le sezioni sono 1-forme sullo spaziotempo
$A_i = s_i^* \omega$ e sono detti \emph{potenziali di gauge}.\\
Si sottolinea l'importanza del teorema \ref{thm:gaugepotential} che dati i
potenziali di gauge definiti sugli intorni locali $U_i$ esiste sempre la 1-forma
di connessione $\omega$ sul fibrato.\\

Il pullback della curvatura $\Omega$, definita come differenziale esterno della
1-forma di connessione $\omega$, è detto \emph{tensore forza di campo} $F = \dd A$.\\

Se il fibrato è banale, ossia può essere ricoperto da una sola carta ed ha la
struttura globale di prodotto diretto $M \times G$, allora esiste una simmetria
globale per il sistema. Se invece il fibrato è non banale, ossia non è descritto
tramite un'unica carta, non può essere definita una simmetria globale e il potenziale
di gauge può essere descritto solo tramite diverse carte locali, concordanti sulle
regioni di intersezione tramite una trasformazione di gauge\footnote{
  Si veda il monopolo di Wu-Yang nella sezione successiva \ref{sec:wuyangmonopole}.
}.\\
