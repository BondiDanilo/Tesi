%------------------------------------------------------------------------------%
%------------------------------- Preamble -------------------------------------%
%------------------------------------------------------------------------------%

\documentclass[10pt,a4paper]{report}
\usepackage{graphicx}
\usepackage[utf8]{inputenc}
\usepackage{amsthm}             %stili teoremi
\usepackage{amssymb,amsmath}
\usepackage{geometry}
\usepackage[colorlinks = true,
          linkcolor = blue,
          urlcolor  = blue,
          citecolor = blue,
          anchorcolor = blue]{hyperref}
\usepackage{bookmark}
\usepackage{wrapfig}
\geometry{
	top=25mm,
	bottom=25mm,
	left=30mm,
	right=25mm
}
\usepackage[ampersand]{easylist}
\usepackage[nottoc]{tocbibind} %bibliografia
%------------------------------------------------------------------------------%
\theoremstyle{plain}
\newtheorem{theorem}{Teorema}[equation]

\theoremstyle{definition}
\newtheorem{definition}{Definizione}[equation]

\theoremstyle{plain}
\newtheorem{proposition}{Proposizione}[equation]

\theoremstyle{plain}
\newtheorem{lemma}{Lemma}[equation]

\theoremstyle{remark}
\newtheorem{example}{Esempio}[equation]

\theoremstyle{definition}
\newtheorem{axiom}{}[section]

\newcommand{\tab}[1][1cm]{\hspace*{#1}}
\graphicspath{{/home/dan/Desktop/UNI/TESI/Images/}}	%Default path for graphics
\renewcommand{\vec}[1]{\mathbf{#1}}
\newcommand{\chrsym}{\genfrac{\{}{\}}{0pt}{}} % Simboli Christoffel
\renewcommand\qedsymbol{$\blacksquare$} % quadrato della dimostrazione


% Remove default parindent
\newlength\tindent
\setlength{\tindent}{\parindent}
\setlength{\parindent}{0pt}
\renewcommand{\indent}{\hspace*{\tindent}}

\hypersetup{
	pdftitle={Tesi},%
	pdfauthor={Danilo Bondì},%
	pdfsubject={Aspetti classici e quantistici dei monopoli magnetici in teorie di gauge},%
	pdfkeywords={},%
	colorlinks=true,%
	linkcolor=blue,%
	linktocpage=true,%
	pageanchor=true
}

%------------------------------------------------------------------------------%
%------------------------------- Document -------------------------------------%
%------------------------------------------------------------------------------%

\begin{document}

%%%% INIZIO PRIMA PAGINA DI INTRODUZIONE
\begin{titlepage}

	\centering{
		\huge \textbf{ Università degli studi di Milano Bicocca}\par
		\LARGE Facoltà di Fisica \\
		 Corso di laurea triennale in Fisica \par
	}

\vspace{0.5cm}

\begin{figure}[h]
\centering
\includegraphics[width=0.35\textwidth]{logo.jpg}
\end{figure}

\vspace{2cm}

\centering\huge\textbf{ASPETTI CLASSICI E QUANTISTICI DEI MONOPOLI MAGNETICI IN TEORIE DI GAUGE}

\vspace{2cm}

\Large
\raggedright Relatore: Prof. Zaffaroni Alberto\par
\begin{table}[h]
	Tesi di laurea di: \\
	Danilo BONDÌ \\
	Matr. N. 801827\\
\end{table}

\vspace{1cm}

\vfill\centering{Anno Accademico 2017/2018}

\end{titlepage}

%%%% FINE PRIMA PAGINA DI INTRODUZIONE


% %%%% PAGINA VUOTA DI RISPETTO
% 		\setcounter{page}{2}
%         \null
%         \thispagestyle{empty}
%         \newpage


\pdfbookmark{\contentsname}{Index}
\tableofcontents
        % Intro page and index
\chapter{Monopolo di Dirac}
%------------------------------------------------------------------------%
\section{Potenziale di Dirac}
%
introduzione, dire cose bla bla bla
Campo magnetico generato da una carica magnetica $g$ situata nell'origine
$$
   B = g \frac{\vec r}{r^3} = \frac{g}{r^2} \vec u _r
$$
dove $\vec r = (x,y,z)$ , $r = \sqrt{x^2 + y^2 + z^2}$ e $\vec u _r = \frac{\vec r}{r}$
il versore radiale.\\
Le equazioni del moto di una carica elettrica $e$ nel campo $\vec B$,
generate dalla Lagrangiana di minima interazione tra una carica elettrica e un campo
magnetico
$$
   L = \frac{1}{2} m \dot{\vec r} ^2 + e \dot{\vec r} \cdot \vec A
$$
sono
$$
   m \ddot{\vec r} = e\dot{\vec r} \times \vec B =
      \frac{eg}{r^3}(\dot{\vec r} \times \vec r)
$$
dove il vettore potenziale che compare nella Lagrangiana deve soddisfare
$B = \nabla \times \vec A = g \frac{\vec r}{r^3}$.\\

Come calcolato nella sezione \ref{sec:dirac_potential} un potenziale che soddisfa
questa relazione è:
\begin{equation}\label{eq:diracpotential}
  \vec A = g(1 + \cos\theta)\left( \frac{\sin\phi}{r\sin\theta},
     \frac{-\cos\phi}{r\sin\theta},0 \right)
\end{equation}
Si presenta immediatamente il seguente problema. Se si ammette l'esistenza di una carica
magnetica, il teorema di Gauss garantsce che $\nabla \cdot \vec B = 4\pi g\delta^{(3)}(\vec r)$,
ossia che il flusso del campo attraverso una qualsiasi superficie chiusa contenente
la carica magnetica $g$ è pari a $4\pi g$, in contraddizione con
$\nabla \cdot \vec B = \nabla \cdot (\nabla \times \vec A) = 0$, che genera un
flusso nullo.
\begin{equation}\label{eq:B4gpi}
   \int_S d\sigma \vec B = \int_V d^3 x \nabla \cdot \vec B = 4g\pi
\end{equation}
\begin{equation}
   \int_S d\sigma \vec B = \int_V d^3 x \nabla \cdot \vec B
      = \int_V d^3 x \nabla \cdot (\nabla \times \vec A) = 0
\end{equation}
Il potenziale $\vec A$ presenta però delle discontinuità per $\sin\theta = 0$.
Si verifica immediatamente che $\theta = 0$ è un punto singolare, mentre $\vec A$
è continuo in $\theta = \pi$. Il calcolo fatto non è quindi corretto lungo l'asse z
positivo ($\theta = 0$), e occorre regolarizzare il potenziale.\\
Siano $\epsilon > 0$  e $r_\epsilon = \sqrt{r^2 + \epsilon^2}$. Si definiscono ora
%
\begin{align*}
   \vec A_\epsilon &:= \frac{g}{r_\epsilon}\frac{1}{r_\epsilon - z} \vec u _\phi  \\
   \vec B _\epsilon & = \nabla \times \vec A_\epsilon = \frac{g}{r_\epsilon^3}\vec r
      - g\epsilon^2 \left( \frac{1}{r_\epsilon^3(r_\epsilon - z)}
         + \frac{1}{r_\epsilon^2(r_\epsilon-z)^2} \right)\vec u _z
\end{align*}
%
e si vuole definire $\tilde{\vec B} := \lim_{\epsilon \to 0} \vec B _\epsilon$
$$
   \lim_{\epsilon \to 0} \vec B_\epsilon = \lim_{\epsilon \to 0} \left[
      \frac{g}{r^3}\vec r - 2g\epsilon^2 \left( \frac{1}{r^2(x^2 + y^2 + \epsilon^2)}
            + \frac{2}{(x^2 + y^2 + \epsilon^2)^2} \right) \Theta(z) \vec u _z \right]
$$
dove $\Theta(z)$ è la funzione di Heaviside per l'asse z.\\
Si vuole ora valutare il flusso del campo magnetico regolarizzato $\tilde{\vec B}$
attraverso una superficie chiusa $S$ centrata nell'origine e di raggio unitario, supponendo
che sia possibile scambiare il limite e l'integrale
$$
  \int_S \mathrm{d} \sigma \lim_{\epsilon \to 0} \vec B_\epsilon = \lim_{\epsilon \to 0} \int_S \mathrm{d}\sigma \vec B_\epsilon
$$
Si veda la sezione \ref{sec:flusso_regolarizzato} per il calcolo esplicito.
L'unico termine in $\epsilon$ che porta contributo al flusso è il secondo e si
ha:
$$
   \int_S \mathrm{d}\sigma \tilde{\vec B}
      = \int_S \mathrm{d}\sigma \left( \frac{g}{r^3}\vec r
         - 4\pi g \delta(x)\delta(y)\Theta(z) \vec u _z \right)
      = 4\pi g - 4\pi g = 0
$$
%
Il campo regolarizzato è composto da due termini:
$$
   \tilde{\vec B} = \vec B + \vec B _{string} = \frac{g}{r^3}\vec r
      - 4\pi g \delta(x)\delta(y)\Theta(z) \vec u _z
$$
L'effetto del campo generato dalla stringa di singolarità lungo l'asse z positivo
è quello di annullare il flusso del campo prodotto dalla carica, risolvendo la
contraddizione evidenziata in precedenza.

%------------------------------------------------------------------------------%

\section{Trasformazione della stringa}\label{sec:gaugestring}
Come evidenziato nella sezione precedente, non è possibile definire ovunque un
potenziale di monopolo continuo e derivabile. L'effetto del potenziale singolare
\ref{eq:diracpotential} conduce a un termine extra di campo della stringa,
diretto lungo la singolarità.
Per rimuovere questo termine singolare, si vuole far sì che non corrisponda a una
configurazione fisica.\\
Naturalmente la scelta del sistema di coordinate per la descrizione del monopolo
è arbitraria, quindi il primo approccio è di richiedere che tutte le configurazioni
possibili della stringa ($\theta$) siano fisicamente equivalenti.
Ossia si vogliono trovare le trasformazioni tra le configurazioni di stringa e
le condizioni che rendano tali configurazioni identiche.\\

Si ricorda che l'elettrodinamica è invariante per trasformazioni del potenziale
vettore del tipo
\begin{equation}
   \vec A \mapsto \vec A' = A + \nabla \lambda (\vec r) = A + \frac{i}{e
   \footnote{Carica dell'elettrone} } U^{-1}\nabla U
\end{equation}
dove $U(\vec r) = \exp(ie\lambda(\vec r))$ e $\lambda$ è un'arbitraria funzione
delle coordinate. Una trasformazione di questo tipo è detta
\textbf{trasformazione di gauge}  del gruppo $U(1)$, in quanto $U$ appartiene
al gruppo $U(1)$ delle matrici unitarie $1x1$ a coefficienti complessi.\\

Si vuole trovare la trasformazione di gauge che risolve la singolarità.\\

Si consideri il flusso del campo magnetico attraverso una superficie chiusa $S$
prima e dopo la trasformazione di gauge e si valuti quale è la sua variazione
$$
   \int_S d\sigma B' - \int_S d\sigma B = \int_S d\sigma (\nabla \times \nabla \lambda)
      = \oint_{\partial S} d\vec l \cdot \nabla \lambda
$$
Il flusso non cambia solo se la funzione $\nabla \lambda$ è periodica $\nabla
\lambda(\phi) = \nabla \lambda( \phi + 2\pi)$. Si consideri allora la trasformazione
$$
\lambda(\vec r) = 2g\phi = 2g\arctan\left(\frac{y}{x}\right) \to U = \exp(2ieg\phi)
$$
Si nomina $\vec A^-$ il potenziale di dirac trovato in precedenza (in quanto ben
definito per $z$ negative)
$$
   \vec A^- = \frac{g}{r}\frac{-1 - \cos\theta}{\sin\theta} \vec u _\phi  \\
$$
$$
  \vec A ^- \mapsto \vec A ^- - \frac{i}{e} \exp( -2ieg\phi )\nabla \exp(2ieg\phi ) =
     -\frac{g}{r}\frac{1+\cos\theta}{\sin\theta}\vec u _\phi
     + \frac{2g}{r\sin\theta}\vec u _\phi
     = \frac{g}{r}\frac{1 - \cos\theta}{\sin\theta} \vec u _\phi
     =: \vec A ^+
$$
Il potenziale trasformato è singolare per $\theta = \pi$ e regolare per
$\theta =  0$ quindi, analogamente a quanto visto per $\vec A^-$ produce un termine
di campo di stringa lungo l'asse $z$ negativo.\\
La trasformazione di gauge scelta agisce come una rotazione della stringa
di un angolo $\theta = \pi$, quindi il campo generato in questo modo non è
considerato fisico (?).\\

Si sottolinea che l'approccio di regolarizzazione dei potenziali singolari
utilizzato in questo capitolo è non rigoroso e serve solo per dare un'introduzone
al problema. Lo si potrebbe rendere rigoroso trattandolo in teoria delle
distribuzioni, che esula dallo scopo di questo elaborato. Si seguirà quindi
un approccio differente nelle sezioni successive

Si riporta l'espressione dei due potenziali di dirac definiti in precedenza, che
sarà utile nella trattazione successiva.
\begin{equation}\label{eq:localdiracpotential}
  \vec A^\pm = \frac{g}{r}\frac{\pm 1 - \cos\theta}{\sin\theta} \vec u _\phi
\end{equation}

%------------------------------------------------------------------------------%

\subsection{Condizione di quantizzazione della carica}
Si consideri una particella di massa $m$ e di cariche elettrica $e$ e magnetica
$g$ in interazione con un campo di monopolo magnetico, descritto dal potenziale
$A$. Sia $\psi$ la funzione d'onda della particella. L'equazione di Schrödinger
della particella è
\begin{equation}
   \frac{1}{2m}\left( \vec{\hat{p}} - \frac{e}{c} \vec{\hat{A}} \right)^2 \psi(\vec r) = E\psi(\vec r)
\end{equation}
Siano $\vec A^\pm$ su $U^\pm$ e le funzioni d'onda $\psi^\pm$ su $U^\pm$ e $\lambda =2g\phi$.
In seguito alla trasformazione di gauge $ \vec A^+ \mapsto \vec A + \nabla (2g\phi)$ (dimostrare)
la funzione d'onda trasforma
$$ \psi^- \mapsto \exp \left( - \frac{2ige}{\hbar c} \phi \right) \psi^+$$
Affinchè la funzione d'onda non sia multivalore, la fase deve essere periodica di
periodo $2\pi$ in $\phi$, ossia
$$
   \exp \left( - \frac{2ige}{\hbar c} \phi \right)
      = \exp \left( - \frac{2ige}{\hbar c} (\phi + 2\pi) \right)
   \Rightarrow
   \exp \left( - \frac{2ige}{\hbar c} 2\pi \right) = 1
   \Rightarrow
   \left( - \frac{2ige}{\hbar c} 2\pi \right) = 2n\pi \quad n \in \mathbb{Z}
$$
Allora si deve avere che

\begin{equation}\label{eq:diracquantumcharge}
   \boxed{
      \frac{2eg}{\hbar c} = n \quad  n \in \mathbb{Z}
   }
\end{equation}
Questa condizione è la celebre \textbf{condizione di quantizazione della carica
di Dirac}, da cui risulta che se esiste una carica magnetica di monopolo $g$,
la carica elettrica è quantizzata. Poichè si osserva che la carica elettrica è
quantizzata, questa condizione costituirebbe uno spunto di ricerca dei monopoli
magnetici.\\
Esistono altre teorie che spiegano equivalentemente la quantizzazione della carica
elettrica, ma non verranno qui indagate.\\

\section{Calcoli}
%------------------------------------------------------------------------------%
\textcolor{red}{(si può togliere)}
\subsection{Calcolo del potenziale vettore di Dirac}
\label{sec:dirac_potential}
Si consideri il campo classico di monopolo generato da una carica magnetica $g$ posta
nell'origine del sistema di riferimento, $\vec B : \mathbb{R} ^3 \to \mathbb{R} ^3$
    $$ \vec B = g\frac{\vec r}{r^3} = \frac{g}{r^2} \vec u _r $$
Si vuole trovare un potenziale vettore $\vec A : U \subset \mathbb{R} ^3 \to \mathbb{R} ^3$ tale che
$ B = \nabla \times \vec A $ in $U$. Si osserva innanzitutto che per la simmetria
sferica del campo $\vec B$, il potenziale vettore può essere scritto come:
   $$ \vec A = g \cdot a(\theta) \nabla \varphi $$
per un'opportuna funzione $a(\theta)$.
L'espressione del rotore in coordinate sferiche è:
$$
    \nabla \times \vec A =
         \frac{1}{r\sin\theta} \left(
              \frac{\partial}{\partial\theta}(A_\varphi\sin\theta) -
              \frac{\partial A_\theta}{\partial\varphi}
              \right) \vec u _r \\
         + \frac{1}{r} \left(
              \frac{1}{\sin\theta}\frac{\partial A_r}{\partial\varphi} -
              \frac{\partial}{\partial r}(r A_\varphi)
              \right)\vec u _\theta \\
         + \frac{1}{r} \left(
              \frac{\partial}{\partial r}(r A_\theta) - \frac{\partial A_r}{\partial\theta}
              \right)\vec u _\varphi
$$
ove i versori sono dati da \ref{eq:versors}. Inoltre
$$
   \nabla \varphi = \left( \frac{-\sin\varphi}{r\sin\theta},\frac{\cos\varphi}{r\sin\theta},0 \right)
$$
Si deve avere che $ \nabla \times \vec A = g \cdot \frac{1}{r^2} \vec u _r$,
ossia
$$
\begin{cases}
    \frac{1}{r\sin\theta} \left(
         \frac{\partial}{\partial\theta}(A_\varphi\sin\theta) -
         \frac{\partial A_\theta}{\partial\varphi}
         \right) = g \cdot \frac{1}{r^2} \\
    \frac{1}{r} \left(
         \frac{1}{\sin\theta}\frac{\partial A_r}{\partial\varphi} -
         \frac{\partial}{\partial r}(r A_\varphi)
         \right) = 0 \\
    \frac{1}{r} \left(
         \frac{\partial}{\partial r}(r A_\theta) - \frac{\partial A_r}{\partial\theta}
         \right) = 0
\end{cases}
$$
Osservando che $A_\theta = \vec u _\theta \cdot \vec A $ e
$A_\varphi = \vec u _\varphi \cdot \vec A$ e inserendo $ \vec A = g \cdot a(\theta) \nabla \varphi $
$$
         A_\varphi = (-\sin\varphi \vec u _x + \cos\varphi \vec u _y)
                       \cdot a(\theta) g \nabla \varphi
                = a(\theta)g\left( -\sin\varphi \frac{-\sin\varphi}{r\sin\theta}
                      + \cos\varphi \frac{\cos\varphi}{r\sin\theta}\right)\\
                = \frac{a(\theta)g}{r\sin\theta}
$$
$$         A_\theta = (\cos\theta \cos\varphi u_x + \cos\theta \sin\varphi u_y
                          - \sin\theta u_z) \cdot a(\theta) g \nabla \varphi
                    = a(\theta) g \left( \frac{-\sin\varphi}{r\sin\theta} \cos\theta \cos\varphi
                          + \frac{\cos\varphi}{r\sin\theta} \cos\theta \sin\varphi \right)
                    = 0
$$
Segue quindi che
$$
   \frac{1}{\sin\theta} \frac{\partial}{\partial\theta}(A_\varphi \sin\theta)
      = \frac{1}{\sin\theta} \frac{g}{r} \frac{\partial a(\theta)}{\partial \theta}
      = \frac{g}{r}
$$
Da cui si ricava facilmente l'espressione di $a(\theta)$ integrando ambo i lati
$$
   \int_0 ^\theta \frac{\partial a(\theta')}{\partial \theta'} \dd \theta'
      = \int_0^\theta \sin\theta' \dd \theta' \Rightarrow\\
   \Rightarrow a(\theta) = -(\cos\theta+1) + cost  = -(\cos\theta+1)
$$
Ponendo a zero la costante arbitraria di integrazione.\\
%
Si ottiene allora:
$$
   \boxed{
          \vec A = -g(1 + \cos\theta) \nabla \varphi
                 = g(1 + \cos\theta)\left( \frac{\sin\varphi}{r\sin\theta},
                    \frac{-\cos\varphi}{r\sin\theta},0 \right)
   }
$$
È di immediata verifica che $\nabla \times \vec A = \vec B = \frac{g}{r^2} \vec u _r$.\\
%
Inoltre, si nota che
$$
   \nabla \varphi = \left( \frac{-\sin\varphi}{r\sin\theta},\frac{\cos\varphi}{r\sin\theta},0 \right)
      = \frac{1}{(r\sin\theta)^2}(-x,y,0) = \frac{1}{(r\sin\theta)^2}\vec u _\varphi
      = \frac{1}{r^2(1-\cos^2\theta)} \vec u _\varphi
$$
Allora
$$
   A = -g(1+\cos\theta)\nabla\varphi = -\frac{g}{r^2} \frac{1+\cos\theta}{1-\cos^2\theta} \vec u _\varphi
     = -\frac{g}{r^2} \frac{1}{1-\cos\theta} \vec u _\varphi
     = -\frac{g}{r} \frac{1}{r-r\cos\theta} \vec u _\varphi
     = -\frac{g}{r(r-z)} \vec u _\varphi
$$
Ciò giustifica la definizione della forma differenziale $ A : \mathbb{R}^3 \setminus \{0\}
\to \Omega(\mathbb{R}^3 \setminus \{0\})$ che mappa ogni punto $p \in \mathbb{R}^3 \setminus \{0\} , p = (x,y,z)$ nella forma
$$
    A_p = \frac{g}{r(r-z)}(x \dd  y - y \dd  x) = \frac{g}{r(r-z)} \dd  \varphi
$$
con $r = |p| = \sqrt{x^2+y^2+z^2}$.

%-------------------------------------------------------------------------------%

\subsection{Calcolo del flusso del campo regolarizzato}
\label{sec:flusso_regolarizzato}
Supponendo valga (dimostrare)
$$
  \int_S \dd \sigma \lim_{\varepsilon \to 0} \vec B_\varepsilon = \lim_{\varepsilon \to 0} \int_S \dd \sigma \vec B_\varepsilon
$$
Sia $S = \{(x,y,z) \in \mathbb{R}^3 \setminus \{0\} : x^2 + y^2 \leq \varepsilon ^2\ ,-h<z<h\}$,
cilindro centrato attorno all'asse z di raggio $\varepsilon$ e altezza $2h$, con $h>0$.

\begin{equation*}
   \begin{split}
      \int_S \dd \sigma \tilde{\vec B}  &= \int_S \dd \sigma \left[\frac{g}{r^3}\vec r
         - 2g\varepsilon^2 \left( \frac{1}{r^2(x^2 + y^2 + \varepsilon^2)}
               + \frac{2}{(x^2 + y^2 + \varepsilon^2)^2}\right) \Theta(z)  \right] \\
         &= \int_S \dd \sigma \frac{g}{r^3}\vec r
            - \int_S \dd \sigma \left[ 2g\varepsilon^2 \left( \frac{1}{r^2(x^2 + y^2 + \varepsilon^2)}
                  + \frac{2}{(x^2 + y^2 + \varepsilon^2)^2}\right) \Theta(z) \right] \\
          &= \int_S \dd \sigma [1] - \int_S \dd \sigma [2]
   \end{split}
\end{equation*}

Il secondo integrale ha contributo non nullo solamente sulla faccia superiore del cilindro
$C = \{ (x,y,z) \in \mathbb{R}^3 \setminus \{0\} : x^2 + y^2 \leq \varepsilon^2, z = h \}$
poichè l'integranda è nulla per $z<0$ e il campo è parallelo all'asse z, quindi il flusso
attraverso le pareti del cilindro è nullo.
\begin{equation*}
   \begin{split}
       \int_S \dd \sigma [2] &= 2g\varepsilon^2 \int_C \dd \sigma
             \left( \frac{1}{r^2(x^2 + y^2 + \varepsilon^2)}
                + \frac{2}{(x^2 + y^2 + \varepsilon^2)^2} \right) \\
       &= 2g\varepsilon^2 \int_0^\varepsilon\rho\dd \rho \int_0^{2\pi} \dd \varphi
       \left( \frac{1}{r^2(x^2 + y^2 + \varepsilon^2)}
           + \frac{2}{(x^2 + y^2 + \varepsilon^2)^2}\right)
   \end{split}
\end{equation*}
I punti di $C$ hanno $r^2 = \varepsilon^2 + h^2$, $x = \rho \cos\varphi$, $y = \rho \sin\varphi$, quindi
\begin{equation*}
   \begin{split}
      &= 2g\varepsilon^2 \int_0^\varepsilon\rho\dd \rho \int_0^{2\pi} \dd \varphi
      \left( \frac{1}{(\varepsilon^2 + h^2)(\rho^2 + \varepsilon^2)}
          + \frac{2}{(\rho^2 + \varepsilon^2)^2} \right) \\
   \end{split}
\end{equation*}
Sia $u := \rho^2 + \varepsilon^2$ e $\dd u = 2\rho \dd \rho$
\begin{equation*}
   \begin{split}
      &= 4\pi g\varepsilon^2 \int_{\varepsilon^2}^{2\varepsilon^2} \dd u
         \left( \frac{1}{(\varepsilon^2 + h^2)u} + \frac{2}{u^2} \right)
      = 4\pi g\varepsilon^2
         \left( \frac{1}{\varepsilon^2 + h^2}( \log(2\varepsilon^2)-\log(\varepsilon^2) )
            - 2\left( \frac{1}{2\varepsilon^2} - \frac{1}{\varepsilon^2} \right) \right) \\
      &= 4\pi g\varepsilon^2
         \left( \frac{1}{\varepsilon^2 + h^2} \log\left( \frac{2\varepsilon^2}{\varepsilon^2}\right)
            - \frac{2}{\varepsilon^2}\left( \frac{1}{2} - 1 \right) \right)
      = 4\pi g
         \left( \frac{\varepsilon^2}{\varepsilon^2 + h^2} \log 2
            + \frac{\varepsilon^2}{\varepsilon^2} \right) \\
   \end{split}
\end{equation*}
Allora
$$
   \lim_{\varepsilon \to 0} \int_S \dd \sigma [2] = 4\pi g
      \lim_{\varepsilon \to 0} \left( \frac{\varepsilon^2}{\varepsilon^2 + h^2} \log 2
         + \frac{\varepsilon^2}{\varepsilon^2} \right) = 4\pi g
$$

%-----------------------------------------------------------------------------%

\chapter{Monopoli in teorie di gauge abeliane}
Nella sezione \ref{sec:gaugestring} si è visto come non è possibile definire globalmente
(su tutto lo spazio $\mathbb{R}^3 \setminus \{0\}$) un potenziale di monopolo regolare,
incorrendo nella stringa di singolarità. Il primo approccio è stato di rendere
equivalenti tutte le possibili configurazioni di direzione della stringa, tramite
trasformazione di gauge di tipo $U(1)$. Si è arrivato a definire due potenziali
(\ref{eq:localdiracpotential}) $\vec A ^\pm$ che hanno la stringa situata rispettivamente
lungo l'asse $z$ negativo/positivo, quindi regolari su $\pm z$ ripettivamente.\\
L'approccio qui seguito, proposto da Wu e Yang (1968) \cite{wuyang}, è di
rinunciare a una definizione globale del potenziale di Dirac
in favore di una descrizione tramite due potenziali definiti localmente su due aperti
$U^\pm$, che concordano nella regione di intersezione tramite una trasformazione
di gauge.\\

Innanzitutto, Poichè $\mathbb{R}^3 \setminus \{0\}$ è equivalente omotopicamente
a $S^2$, si vuole studiare il problema sulla sfera. Siano allora $U^\pm$ l'emisfero
nord e sud della sfera
\begin{equation*}
   \begin{aligned}
      U^+ &= \{(x,y,z) \in S^2 \: | \: z > 0 \}
          = \{(r,\theta,\phi) \in S^2 \: | \: 0 \leq \theta \leq \pi/2 \} \\
      U^- &= \{(x,y,z) \in S^2 \: | \: z < 0 \}
          = \{(r,\theta,\phi) \in S^2 \: | \: \pi/2 \leq \theta \leq \pi \}
   \end{aligned}
\end{equation*}
La regione di intersezione è l'equatore
$$
   U^0 = U^+ \cap U^- = \{(x,y,z) \in S^2 \: | \: z = 0 \}
       = \{(r,\theta,\phi) \in S^2 \: | \: \theta = \pi/2 \}
$$
Si possono allora definire i potenziali $\vec A^\pm$ su $U^\pm$, che nella regione
$U^0$ sono legati, come visto nella sezione \ref{sec:gaugestring}, da
$$
   \vec A ^+ - \vec A^- = \nabla \lambda = \nabla (2g\phi) = \frac{2g}{r\sin\theta} \, \vec u _\phi
$$
Si noti che $\nabla \lambda$ è singolare in $\theta = 0,\pi$, ma nella regione $U^0$
si ha $\theta = \pi/2$.\\
Si calcola allora il flusso totale come:
\begin{equation*}
   \begin{aligned}
      \int_{S^2} d\sigma \: \nabla \times \vec A & =
         \int_{U^+} d\sigma \: \nabla \times \vec A^+ +
         \int_{U^-} d\sigma \: \nabla \times \vec A^-
        = \oint_{U^0} d\vec l \: \vec A^+ -
          \oint_{U^0} d\vec l \: \vec A^- \\
      & = \oint_{U^0} d\vec l \: \nabla \, (2g\phi)
        = \int_0^{2\pi} \, 2g \, \phi \, d\phi
        = 4g\pi
   \end{aligned}
\end{equation*}
In accordo con \ref{eq:B4gpi}.\\

Il formalismo più naturale per descrivere il potenziale di monopolo in una teoria
di gauge è quello di un fibrato principale con spazio base $M = \mathbb{R}^3
\setminus \{0\}$ e gruppo di struttura $G$ il gruppo di gauge.\\
Si faccia riferimento all'appendice per la teoria sui fibrati \ref{sec:fibrati}.

Se il fibrato ammette una sezione globale, può essere coperto con un'unica carta.
Il fibrato è allora banale e ha struttura globale di prodotto diretto tra la
varietà $M$ e la fibra $G$, e la classe di Chern caratteristica è $c_0 = 1$.
La complessità della struttura del fibrato di solito è legata alla contraibilità
dello spazio base.\\

Si consideri una teoria di gauge con gruppo $G = U(1)$, come l'elettrodinamica classica.\\

Se si prende come base $M = \mathbb{R}^3$, che è uno spazio contraibile, può essere
ricoperto da un'unica carta e il potenziale di gauge $A$ è definito globalmente su $M$,
continuo e differenziabile, e il tensore elettromagnetico $F$ è una 2-forma
sia chiusa $DF = 0$ che esatta $F = dA$.\\

Se invece lo spazio base è $M = \mathbb{R}^3 \setminus \{0\}$, che non è uno spazio
contraibile, la situazione cambia drasticamente. $\mathbb{R}^3 \setminus \{0\}$ è
equivalente omotopicamente alla sfera $S^2$.
In riferimento all'esempio \ref{ex:monopolechern}, è possibile costruire due fibrati
principali: uno con classe di Chern $c_0 = 1$ e uno con $c_1 = F/2\pi$. Quest'ultimo,
detto appunto \emph{fibrato di monopolo} è quello che descrive correttamente il
monopolo di Wu-Yang.
%------------------------------------------------------------------------------%
%------------------------------------------------------------------------------%
\section{Fibrato di Monopolo di Wu–Yang}
\textcolor{red}{Riscrivere meglio}\\
Si consideri un fibrato principale con $M = S^2$ e $G = U(1)$.\\
Il gruppo di gauge $U(1) = \{g = e^{i\alpha} \}$ è parametrizzato da un
parametro $\alpha$, che è una coordinata ciclica lungo la fibra, ed è identificato
quindi con la sfera $S^1$.\\
Una parametrizzazione per la sfera di raggio unitario è data dalle usuali
coordinate polari $\phi \in[0,2\pi)$ e $\theta \in [0,\pi)$,
quindi si può definire una base per lo spazio tangente $T_pS^2$ come
$\{ \frac{\partial}{\partial \theta} \: , \: \frac{\partial}{\partial \phi } \}$
e per lo spazio cotangente $T^*_pS^2$ come $\{ d\theta \: , \: d\phi \}$.
La derivata esterna è data da
$$
   d = d\theta \frac{\partial}{\partial \theta}
     + d\phi   \frac{\partial}{\partial \phi }
$$
Siano $U^\pm$ emisferi nord e sud e $U^0$ l'equatore come sopra.
Si hanno allora le due carte locali per il fibrato
$$
   ( U^+ \times S^1 , \{ \theta, \phi,\alpha^+ \} ) \quad \mathrm{e}\quad
   ( U^- \times S^1 , \{ \theta, \phi,\alpha^- \} )
$$
La 1-forma di connesione $\omega$\footnote{Si veda \ref{}}
definita globalmente sul fibrato, portata su $U^\pm$ tramite pullback,
dà il potenziale $A$
$$
   A = A(\theta)\wedge dr =  \begin{cases}
      A^+ =  \frac{n}{2}(1 - \cos\theta ) d\phi & \mathrm{su \:} U^+ \\
      A^- = -\frac{n}{2}(1 + \cos\theta ) d\phi & \mathrm{su \:} U^-
   \end{cases}
$$
e la 2-forma di curvatura $\Omega$\footnote{Si veda \ref{}} viene portata
sul tensore elettromagnetico $F$
$$
   F = dA = \frac{n}{2} \: \sin\theta \: d\theta \wedge d\phi
$$
Osservando che i versori delle coordinate sferiche sono dati da
$\vec u _\phi = \sin\theta d\phi$ e $ \vec u _\theta = d\theta$, si riconosce in
$F$ la componente radiale del campo magnetico di un monopolo
{\textcolor{red}(scriverla)}.\\

La 2-forma $F$ è chiusa ($dF = d^2A = 0$ su entrambi gli emisferi),
analogamente al caso precedente, ma questa volta non è esatta perchè i due potenziali
sono definiti su insiemi differenti.\\

Nella regione di intersezione $U^0$ i due potenziali sono vincolati dalla condizione
di compatibilità \ref{eq:condcompatibility}, la trasformazione di gauge
$A^+ = A^- + n d \phi$. Si definiscono allora le funzioni di transizione \ref{}
che sono elementi di $\Phi \in U(1)$ che collegano le coordinate delle fibre
$e^{i\alpha^+} = \Phi e^{i\alpha^-}$. Su $U^0$ si ha fissata la coordinata polare
a $\theta = \pi/2$ e quindi le $\Phi$ sono funzioni della sola coordinata azimutale,
$ \Phi = e^{in\phi}$ con $n$ intero.\\

Le funzioni di transizione $\Phi$ sono una mappa dell'equatore $S^1$ nel gruppo
di gauge $U(1)$, sono quindi cammini in $U(1)$,
\textcolor{red}{(spiegare meglio)}
e possono essere
classificate in base alla classe di omotopia in $\pi_1(U(1))$ a cui appartengono.
L'intero $n$ assume il significato topologico di \emph{numero di avvolgimento}
della corrispondente classe di omotopia.\\

Come visto sopra, la topologia del monopolo è caratterizzata dalla classe di Chern
relativa a $c_1$
\begin{equation*}
%   \begin{aligned}
      c_1   = \frac{1}{2\pi} \int_{S^2} F
            = \frac{1}{2\pi} \left( \int_{U^+} dA^+ \int_{U^-} dA^- \right)
            = \frac{1}{2\pi} \int_{S^1} (A^+-A^-)
            = \frac{1}{2\pi} \int_0^{2\pi} nd\phi  = n
%   \end{aligned}
\end{equation*}
La carica magnetica coincide quindi con il primo numero di Chern. Per $n=0$ il
fibrato è banale e ha la forma $S^2 \times S^1$, mentre per $n=1$ si ha il
\emph{fibrato di Hopf}, di seguito descritto.\\

Si osserva infine che anche la funzione d'onda di una particella carica in un
campo di monopolo non può essere definita globalmente.
Si definiscono su $U^\pm$ le funzioni d'onda $\psi^\pm$ che su $U^0$ sono collegate
da $\psi^+ = \Phi \psi^-$.
%------------------------------------------------------------------------------%
%------------------------------------------------------------------------------%
\section{Fibrato di Hopf}
Il fibrato di Hopf descrive la sfera $S^3$ come fibrato con spazio
base $S^2$ (parametrizzato dagli angoli $\theta,\phi$) e come fibra $S^1$ (parametrizzato,
dal paramentro $\alpha$, come sopra).\\

La sfera $S^3$ è definita da:
$$ S^3 = \{ \vec p \in \mathbb{R}^4 \: : \: |\vec p|^2 = 1 \} $$
e può essere parametrizzata in coordinate cartesiane da $(p_1,p_2,p_3,p_4)$
tali che:
\begin{equation}
   \begin{cases}
      p_1 = \cos \frac{\theta}{2} \cos \alpha \\
      p_2 = \cos \frac{\theta}{2} \sin \alpha \\
      p_3 = \sin \frac{\theta}{2} \cos (\phi + \alpha) \\
      p_4 = \sin \frac{\theta}{2} \sin (\phi + \alpha) \\
   \end{cases}
\end{equation}

Assegnata una base $ \{ \partial/\partial p_\mu \} $ dello spazio tangente
$T_p S^3$, si può definire la metrica
\begin{equation}
   d s^2 = g_{\mu\nu} dx^\mu dx^\nu
         = \frac{1}{4} d\theta^2 + d\alpha^2 + sin^2\frac{\theta}{2}d\phi^2
         + 2\sin^2\frac{\theta}{2} d\phi d\alpha
\end{equation}
La proiezione $\pi : S^3 \to S^2$ (mappa di Hopf) è defnita nel modo seguente
$\vec p = (p_1,p_2,p_3,p_4) \mapsto \vec x = (x,y,z)$
\begin{equation}
   \begin{cases}
      x = 2(p_1 p_3 + p_2 p_4)          = \sin\theta \cos\phi \\
      y = 2(p_1 p_4 - p_2 p_3)          = \sin\theta \sin\phi \\
      z = p_1^2 + p_2^2 - p_3^2 - p_4^2 = \cos\theta          \\
   \end{cases}
\end{equation}
Si osservi che si perde completamente la dipendenza dalla variabile $\alpha$.
L'effetto della mappa di Hopf è di mappare un cerchio $S^1$ in un punto
della sfera $S^2$.\\

La sezione del fibrato può essere presa fissando un particolare valore di $\alpha$.
Si richiede che la metrica sulla sfera $S^3$ si riduca alla tradizionale metrica
$ds^2 = d\theta^2 + \sin^2\theta d\phi$ sulla sfera $S^2$ sia nell'emisfero
nord $U^+$($ \theta/2 \mapsto \theta $) che nell'emisfero sud $U^-$
($\pi/2 - \theta/2 \mapsto \theta $). Si vede immediatamente da \ref{} che la
condizione è verificata per $\alpha = 0$ e $\alpha = -\phi$, rispettivamente.
Con questa scelta, la forma di connessione globale $\omega$ su $S^3$
viene mappata dal pullback della sezione appena descritta nella forma locale $A$
come prima \ref{}\\

Le funzioni di transizione $\Phi$ sono definite nella regione di transizione
$U^0 = U^+ \cap U^-$, l'equatore, come mappe $S^1 \to S^1$
\begin{equation}
   \begin{aligned}
  \Phi_\pm & = \Phi_+ \circ \Phi_- ^{-1}
             = \frac{p_3 +  i \, p_4}{p_1 + i \, p_2}
                \sqrt{ \frac{p_1^2 + p_2^2}{p_3^2 + p_4^2} }
             = e^{i\phi} \\
  \Phi_\mp & = \Phi_- \circ \Phi_+ ^{-1}
             = \frac{p_1 + i \, p_2}{p_3 +  i \, p_4}
                  \sqrt{ \frac{p_3^2 + p_4^2}{p_1^2 + p_2^2}}
             = e^{-i\phi}
   \end{aligned}
\end{equation}


Si definisce allora la 1-forma di connessione $\omega$
\begin{equation}
   \begin{aligned}
      \omega & = p_1 d p_2 - p_2 d p_1 + p_3 d p_4 - p_4 d p_3 \\
             & = d\alpha + \frac{1}{2}(1 - cos\theta) d \phi
   \end{aligned}
\end{equation}

e la curvatura $\Omega$
$$
   \Omega = d \omega = 2 ( d p_1 \wedge d p_2 + d p_3 \wedge d p_4)
          = \frac{1}{2} \sin\theta \, d\theta \wedge d\phi
$$
Questa forma è chiusa ed esatta su $S^3$ e si nota subito che corrisponde a $1/2$
della forma di volume della sfera $S^2$\footnote{
che è appunto $dV = \sin\theta d\theta \wedge d\phi$}. Si ha allora, integrando
su $S^2$ come sottospazio di $S^3$\footnote{abuso di notazione}
$$
   \int_{S^2} \Omega = 2\pi
$$

\chapter{Monopoli in teorie di gauge non abeliane}
\textcolor{red}{(da scrivere)}\\
\textcolor{red}{Si veda Shnir, pag 106 per connessione tra non abeliano e dirac}\\
%We can look at this map from a different point of view. Note that the
%sphere S 3 coincides with the group manifold of the group SU(2). Thus we
%reformulate the Hopf map (3.84) in terms of elements of this group, the
%matrices U (theta, phi, alpha) in SU (2) (we use a slightly different parameterization
%in Appendix A, where general properties of the SU (2) group matrices are
%described)
%------------------------------------------------------------------------%
\section{Monopolo di ’t Hooft–Polyakov}
%------------------------------------------------------------------------%


\appendix
\chapter{Cenni preliminari di matematica}
\section{Topologia Generale e Algebrica}
Per una trattazione completa degli argomenti qua accennati si vedano le referenze.

%------------------------------------------------------------------------------%
%------------------------------------------------------------------------------%
\subsection{Definizioni base}

\begin{definition}{(Topologia)}
  Sia $X$ un insieme, è detta \emph{topologia} una collezione di sottoinsiemi
  di $X$ che rispettino i seguenti assiomi ($T = \{A_\alpha\} \subset X$):
     \begin{itemize}
          \item $\varnothing \in T$ e $X \in T$
          \item $ \bigcup\limits_\alpha A_\alpha \in T, \;
             \forall \alpha $ t.c. $A_\alpha \in T$
          \item $ A_\alpha \cap A_\beta \in T \;
             \forall A_\alpha, A_\beta \in T$
     \end{itemize}
\end{definition}
Si sottolinea che la seconda condizione richiede che l'unione \textbf {qualsiasi}
(finita, infinita, numerabile, non numerabile, etc) di elementi della topologia
appartienga alla topologia, mentre la terza richiede solamente che l'intersezione
di due elementi della topologia appartienga ad essa (è immediato generalizzare
a una qualsiasi intersezione \textbf{finita} di elementi della topologia)\\

Lo spazio $X$, dotato della topologia $T$ viene detto \textbf{spazio topologico}.
Gli elementi della topologia $A \in T$ vengono detti \textit{aperti} di $X$.\\

Piccoli esempi più commenti. Si veda \cite{sernesi} per ulteriori esempi di spazi
topologici \footnote{Capitoli 1 e 2}.\\
%------------------------------------------------------------------------------%
\begin{definition}{(Intorno)}
   Sia $(X,T)$ uno spazio topologico e $x \in X$. Un insieme $U \subset X$ è detto
   \emph{intorno di x} se esiste un aperto contenuto in $U$, contenente $x$.
   $$ \exists A \in T, \, A \subset U \:\mathrm{t.c.}\: x \in A \rightarrow U \:
   \mathrm{ \emph{intorno} \: di } \: x.$$
\end{definition}
%------------------------------------------------------------------------------%
\begin{definition}{(Base della topologia)}
   Sia $(X,T)$ spazio topologico, $x \in X$. Una \emph{Base} per la topologia $T$
   è una famiglia $\mathfrak{B}$ di aperti tale che ogni aperto $A \in T$
   è unione di insiemi di $\mathfrak{B}$
   $$ \forall A \in T \: A = \bigcup \limits_i B_i \: , \: \{B_i\} \in \mathfrak{B}$$
\end{definition}
%------------------------------------------------------------------------------%
%\begin{definition}{(Base di intorni)}
%   Sia $(X,T)$ spazio topologico, $x \in X$. Sia $N_x$ l'insieme di tutti gli
%   intorni di $x$. Una \emph{Base di intorni} è una famiglia $B_x$ di $x$
%   tale che per ogni intorno $U$ di $x$ esista un $B \in B_x$ contenuto in $U$
%   $$ \forall U \in T \: \exists B \in (B_x \subset N_x) \mathrm{\: t.c. \:} B \subset U $$
%\end{definition}
%------------------------------------------------------------------------------%
\begin{definition}{(Secondo assioma di Numerabilità)}
\label{def:IInumerabile}
   Lo spazio $X$ ha una base con cardinalità numerabile.
\end{definition}
Esempio di $\mathbb{R}^n$
%------------------------------------------------------------------------------%
 \begin{definition}{(Continuità)}
    Sia $f:X \to Y$ una funzione tra gli spazi topologici $X,Y$ dotati rispettivamente
    delle topologie $T_X, T_Y$ tale che la \emph{controimmagine} di ogni aperto in $Y$
    è un aperto in $X$
    $$\forall A_Y \in T_Y \rightarrow f^{-1}(A_Y) \in T_X$$
    allora $f$ è detta una funzione \emph{continua}
 \end{definition}
Per funzioni $\mathbb{R}^n \to \mathbb{R}^m$ questa definizione coincide con
la usuale definizione di continuità di Analisi Matematica (si veda \cite{soardi} per una
dimostrazione per funzioni $\mathbb{R} \to \mathbb{R}$).\footnote{Soardi, capitolo 7, sezione 7.3}\\

Se una funzione $f$ continua è invertibile e la sua inversa $f^{-1}$ è continua allora
$f$ è detta un \textbf{omeomorfismo}.
%------------------------------------------------------------------------------%
%------------------------------------------------------------------------------%
\subsubsection{Invarianti Topologici}
\textcolor{red}{(da scrivere)}
\subsubsection{Compattezza, Connessione, Connessione per archi}
\textcolor{red}{(da scrivere)}
%------------------------------------------------------------------------------%
\begin{definition}{(Varietà topologica)}
\label{def:var_topologica}
   Sia $(X ,T)$ uno spazio topologico con le seguenti proprietà:
  \begin{itemize}
     \item (Proprietà di Hausdorff) Punti distinti di $X$ hanno intorni disgiunti
     $$\forall x_\alpha,x_\beta \in X, \: x_\alpha \neq x_\beta , \:
       \exists A_\alpha , A_\beta \in T \: (x_\alpha \in A_\alpha , \:
       x_\beta \in A_\beta ,\: \mathrm{intorni}), \: \mathrm{t.c.} \: A_\alpha \cap A_\beta = \varnothing $$
     \item (Localmente n-Euclideo) Ciascun punto di $X$ ha un intorno che è
     omeomorfo a un aperto di $\mathbb{R}^n$.
     $$
       \forall x \in X \: \exists U \in T \:,\: \exists\phi : U \to \mathbb{R}^n
          \mathrm{\: t.c.\:} \phi \mathrm{\: omeomorfismo}
     $$
     \item (Secondo assioma di Numerabilità) Lo spazio $X$ rispetta il secondo assioma di Numerabilità (\ref{def:IInumerabile}).
  \end{itemize}
  $X$ è allora detto una \emph{varietà topologica}
\end{definition}
Il numero $n$ è detto la \emph{dimensione} della varietà. Si può dimostrare che è unico.\\
La coppia $(U,\phi)$ è detta \emph{intorno coordinato} o \emph{carta}.
Una famiglia di carte $\mathcal{A} = \{ (U_\alpha , \phi_\alpha) \}$ è detta
\emph{atlante} se rispetta le seguenti proprietà:
\begin{itemize}
    \item L'insieme degli intorni $\{U_\alpha\}$ ricopre $X$ :
        $X \subset \bigcup \limits_\alpha U_\alpha$
    \item Gli intorni coordinati sono a due a due \emph{compatibili}: per ogni coppia
    di intorni coordinati
       $(U_\alpha , \phi_\alpha), (U_\beta , \phi_\beta) \mathrm{\:t.c.\:} \:
       U_\alpha \cap U_\beta \neq \varnothing $
    le funzioni di transizione (cambio di coordinate)
    \begin{equation*}
        \begin{split}
           \psi = \phi_\beta \circ \phi_\alpha &: \phi_\alpha(U_\alpha \cap U_\beta)
              \to \phi_\beta (U_\alpha \cap U_\beta) \\
           \psi^{-1} = \phi_\alpha \circ \phi_\beta &: \phi_\beta(U_\alpha \cap U_\beta)
              \to \phi_\alpha (U_\alpha \cap U_\beta) \\
        \end{split}
    \end{equation*}
    sono funzioni continue
\end{itemize}
Si veda \cite{sernesi} per esempi di varietà topologiche\\

Si sottolinea che una varietà topologica è uno spazio che può essere descritto
localmente con un sistema di coordinate euclidee e applicare quindi tutti i metodi
noti di Analisi.
Ad esempio si può richiedere che le funzioni coordinate siano funzioni differenziabili
di classe $C^{k}(U)$, per qualche $k$ (senza perdita di generalità, si richiedere che le funzioni siano $C^\infty$). Questo porta alla definizione di
\textbf{varietà differenziale} (si veda \ref{sec:vardiff}).\\
\textcolor{red}{Cosa significa "in coordinate locali"}

\subsection{Omotopia e Classi di omotopia}

\section{Varietà differenziali}
\label{sec:vardiff}
In relazione alla definizione \ref{def:var_topologica}.
\begin{definition}{(Varietà differenziale)}
   Uno spazio topologico $M$ si dice \emph{varietà differenziale} se è una
   varietà topologica in cui le funzioni coordinate e le funzioni di transizione
   sono funzioni differenziabili $C^\infty$
\end{definition}

Struttura differenziale e dipendenza da atlante\\

Una funzione differenziabile, invertibile e con inversa differenziabile si dice
\textbf{diffeomorfismo} (quando serve, viene indicata la classe $C^k$ di differenziabilità).
%------------------------------------------------------------------------------%
%------------------------------------------------------------------------------%
\subsection{Spazio tangente e cotangente}
Si vuole generalizzare la nozione di vettore tangente ad uno spazio
$M \subset \mathbb{R}^n$ (si pensi pure a una superficie in $\mathbb{R}^3$
o una curva in $\mathbb{R}^2$). Un vettore identifica univocamente una direzione
in $\mathbb{R}^n$. Si vuole associare ad ogni vettore l'operazione derivata direzionale
lungo la direzione individuata dal vettore, valutata nel punto $p \in \mathbb{R}^n$.\\
Scelta una base $\{\vec e _i\}$ di $\mathbb{R}^n$, ogni vettore $\vec v \in \mathbb{R}^n$
si può esprimere in maniera unica come combinazione lineare degli elementi della base
$$ \vec v = v_1\vec e _1 + \dots + v_n\vec e _n $$
Sia $f$ una funzione differenziabile definita in un opportuno intorno $U$ del punto $p$
(per semplicità nel seguito si indica con $C^\infty(p)$ l'insieme delle funzioni lisce definite su
opportuni intorni del punto $p \in \mathbb{R}^n$).\\
La derivata direzionale di $f$ lungo $\vec v$ è quindi data da
   $$ \left. \frac{\partial f}{\partial \vec v} \right |_p =
         \left. v_1\frac{\partial f}{\partial x_1} \right |_p + \dots +
         \left. v_n\frac{\partial f}{\partial x_n} \right |_p $$
Inoltre, per le proprietà delle derivate $\forall \alpha,\beta \in \mathbb{R}$
e $\forall f,g \in C^\infty(p)$:
$$
      \left. \frac{\partial}{\partial \vec v}(\alpha f + \beta g) \right |_p
         = \alpha \left. \frac{\partial f}{\partial \vec v} \right |_p +
           \beta  \left. \frac{\partial g}{\partial \vec v} \right |_p
      \quad \mathrm{e} \quad
      \left. \frac{\partial}{\partial \vec v}(fg) \right |_p
         = \left. \frac{\partial f}{\partial \vec v} \right |_p g(p) +
           \left. f(p)\frac{\partial g}{\partial \vec v} \right |_p
$$
Il vettore $\vec v$ è allora individuato in maniera univoca dal modo in cui agisce
la derivata direzionale su tutte le funzioni differenziabili in un intorno di $p$.
Si definisce allora il \emph{vettore tangente} alla varietà $M$ nel punto $p$:
\begin{definition}{(Vettore tangente)}
      Sia $M$ una varietà differenziale e $p \in M$. Si dice \emph{vettore tangente}
      a $X$ nel punto $p$ un'applicazione $\vec V _p : C^\infty(p) \to \mathbb{R}$ tale che:
      \begin{itemize}
         \item $\vec V _p[\alpha f + \beta g] = \alpha \vec V _p[f] + \beta \vec V_p [g]
            \quad \forall f,g \in C^\infty(p) \mathrm{\:e\:} \forall \alpha,\beta \in \mathbb{R}$
         \item $\vec V _p [fg] = \vec V _p[f] g(p) + f(g)\vec V _p [g]
            \quad \forall f,g \in C^\infty(p)$
      \end{itemize}
\end{definition}
L'insieme dei vettori tangenti a una varietà $M$ nel punto $p$ è detto \textbf{spazio tangente}
alla varietà nel punto $p$ e si indica con $T_p(M)$.
Si può mostrare facilmente che $T_p(M)$ è uno spazio vettoriale.\\

Si definisce \textbf{spazio cotangente} $T_p^*(M)$ a una varietà $M$ nel punto
$p \in M$ il duale\footnote{Si ricorda che il duale $V^*$ di uno spazio vettoriale $V$ è l'insieme
delle applicazioni lineari $\phi : V \to \mathbb{R}$. Data una base $\{e_i\}$ di $V$,
la base canonica per $V^*$ è definita dalle proiezioni nella i-esima coordinata $dx^i$,
quindi $dx^i(e_j)=\delta^i_j$. Data una qualsiasi $\phi\in V^*$ applicazione lineare
su $V$, si ha $\phi = \sum_i a_i dx^i$ per $\{a_i\}\in\mathbb{R}$. }
dello spazio tangente $T_p(M)$.\\

Data una base di $\mathbb{R}^n$ si definiscono:
\begin{itemize}
   \item $ e_i = \left. \frac{\partial}{\partial x_i}\right |_p $
      le derivate parziali lungo la coordinata i-esima sono base di $T_p(M)$.
   \item $ e^i = \mathrm{d}x^i $ gli elementi di linea differenziali lungo la
      coordinata i-esima sono base di $T_p^*(M)$.
\end{itemize}

Si definisce \textbf{campo vettoriale} un'applicazione che a un punto $p$
della varietà $M$ associa un vettore tangente al punto. Si vuole richiedere anche
una dipendenza continua o liscia dal punto base $p$\footnote{Si veda \ref{sec:fibratotangente}
per chiarificare questa affermazione.}.
Lo spazio dei campi vettoriali sulla varietà $M$ si indica con $\mathcal{X}(M)$
$$ \vec V : M \to T_p(M) \: ,\: p \mapsto \vec V _p $$

Si definisce \textbf{campo covettoriale} un'applicazione che a un punto $p$
della varietà $M$ associa un vettore cotangente al punto. Si vuole richiedere anche
una dipendenza continua o liscia dal punto base $p$.
$$ \vec U : M \to T_p^*(M) \: ,\: p \mapsto \vec U _p $$

Siano $V = \sum_i v^i\frac{\partial}{\partial x^i}$ e $U = u_i dx^i$ campi
vettoriali/covettoriali e si consideri una generica trasformazione di coordinate
$x \mapsto y(x)$ $V$ e $U$ sono invarianti (indipendenti dalla scelta della base).
Si ha che:
$$
   dy^i = \sum_j \frac{\partial y^i}{\partial x^j} dx^j \mathrm{\quad e \quad}
   \frac{\partial}{\partial y^i} = \sum_j \frac{\partial x^j}{\partial y^i}
      \frac{\partial }{\partial x^j}
$$

In base a queste leggi di trasformazione in seguito a cambio di coordinate,
i vettori tangenti si dicono \textbf{covarianti} e i vettori cotangenti
si dicono \textbf{controvarianti}.\\
(o il contrario?)\\
%------------------------------------------------------------------------------%
due parole su funzioni tra varietà e Differenziali $F_*$\\

Per le coordinate si deve quindi avere:
$$
   v'^i = \sum_j \frac{\partial y^i}{\partial x^j} v^j \mathrm{\quad e \quad}
   u'_i = \sum_j \frac{\partial x^j}{\partial y^i} u_j
$$
Di conseguenza il prodotto interno $<U,V> $ è invariante:
$ <U,V> = \sum_i u_i v^i = \sum_j u'_j v^j $\\
(lo metto?)\\
%------------------------------------------------------------------------------%
%------------------------------------------------------------------------------%
\subsection{Varietà con bordo}
\textbf{Orientazione}\\

\section{Tensori e Forme differenziali}
Sia $V$ uno spazio vettoriale su su $\mathbb{R}$.\\
Si definisce \textbf{tensore} misto con $r$ indici covarianti e $s$ indici
controvarianti un'\emph{applicazione multilineare} $F : V^r \times V^{*s} \to
\mathbb{R}$ , dove $V^r = V \times \dots \times V$ e
$V^{*s} =  V^* \times \dots \times V^*$.\\

Si indica con $\mathcal{T}^{(r,s)}(V)$ lo spazio dei (r,s)-tensori su $V$\\

Si definisce il \textbf{prodotto tensoriale} (indicato col simbolo $\otimes$) tra
$F \in \mathcal{T}^r$(V) e $G \in \mathcal{T}^s(V)$ il $r+s$ tensore definito da:
$$ F \otimes G (v_1,\dots,v_{r+s}) = F(v_1,\dots,v_r)G(v_{r+1},\dots,v_{r+s})$$
dove a destra dell'uguale si ha il prodotto tra i due numeri reali
$F(v_1,\dots,v_r)$ e $G(v_{r+1},\dots,v_{r+s})$.\\
Analogamente, si definisce il prodotto tensoriale tra due tensori controvarianti.\\

Si può facilmente dimostrare la seguente
\begin{proposition}\label{prop:base1}
   Sia $M$ varietà differenziale e $p \in M$. Siano $V = T_p(M)$ e $V^* = T_p^*(M)$
   Sia $F \in \mathcal{T}^{(r,s)}(V)$, allora esistono $a_{i_1,\dots i_r}^{j_1,\dots,j_s}$
   tali che in coordinate locali\\
   $$
      F = \sum_{\substack{(i_1,\dots, i_r)\\(j_1,\dots,j_s)}}
         a_{i_1,\dots, i_r}^{j_1,\dots,j_s}
         \frac{\partial}{\partial x^{i_1}} \otimes \dots
         \otimes \frac{\partial}{\partial x^{i_r}}
         \otimes dx^{j_1} \otimes \dots \otimes dx^{j_s}
   $$
\end{proposition}
Ovvero le basi di $T_p(M)$ e $T_p^*(M)$ inducono una base per $\mathcal{T}^r(T_p(M))$
e $\mathcal{T}^s(T_p^*(M))$.\\

Si definisce \textbf{campo tensoriale} sulla varietà $M$ con $r$ indici covarianti
e $s$ indici controvarianti un'applicazione $ M \to \mathcal{T}^{(r,s)}(T_p^*(M)) $
che a ogni $p$ associa un tensore con punto base $p$. Si vuole richiedere anche
una dipendenza continua o liscia da $p$.\\

Si definice \textbf{r-forma} un r-tensore covariante \emph{totalmente antisimmetrico}.\\

Lo spazio delle r-forme sulla varietà $M$ nel punto $p$ si indica con $\Lambda^r(p)$.
La dimensione di $\Lambda^r(p)$ è ${n}\choose{k}$ se $n$ è la dimensione di $M$
\footnote{ Si veda \cite{sernesi},\cite{boothby},\cite{nakahara} }.\\
%------------------------------------------------------------------------------%
\subsection{Forme differenziali}
\begin{definition}{(Forma differenziale)}
   Si definice \emph{r-forma differenziale} un campo tensoriale covariante
  totalmente antisimmetrico, ossia un'applicazione multilineare
  $\omega : M \to \Lambda^r(p)$ che a $p \in M$ associa la $r$-forma $\omega_p$
\end{definition}
Lo spazio delle r-forme differenziali su $M$ si indica con $\Omega^r(M)$.\\

Analogamente a quanto fatto per il prodotto tensoriale, si vuole definire un prodotto
tra $r$ e $s$ forme differenziali che dia una $r+s$ forma differenziale (cioè un
prodotto tensoriale che mantenga l'antisimmetria del tensore).\\

Si definisce \textbf{prodotto esterno} o \textbf{prodotto wedge} tra due forme
differenziali $\alpha \in \Omega^r(M)$ e $\omega \in \Omega^s(M)$ la $r+s$ forma
differenziale definita da:
$$ \alpha \wedge \omega (V_1,\dots,V_{r+s})(p) =
   \frac{1}{r!s!}\sum_{\sigma \in S^{r+s} } (-1)^\sigma
   \alpha_p\otimes\omega_p (V_{\sigma(1)}, \dots , V_{\sigma(r),
      V_{\sigma(r+1)}, \dots , V_,\sigma(r+s)})$$
Dove $S^{r+s}$ è il gruppo delle permutazioni di $r+s$ elementi e $(-1)^\sigma$
è il segno della permutazione $\sigma$.\\
E gode delle seguenti proprietà, di immediata dimostrazione:
\begin{enumerate}
    \item $(\alpha + \beta) \wedge \omega = \alpha\wedge\omega + \beta\wedge\omega$ e
          $\omega \wedge(\alpha + \beta)  = \omega\wedge\alpha + \omega\wedge\beta$
    \item $(c\alpha)\wedge\omega = c(\alpha\wedge\omega) = \alpha\wedge(c\omega)$
    \item $\omega\wedge\alpha = (-1)^{r+s} \alpha\wedge\omega$
    \item $(\alpha\wedge\omega)\wedge\tau = \alpha\wedge(\omega\wedge\tau)$
       (valida grazie alla normalizzazione scelta nella definizione di $\wedge$)
\end{enumerate}
$\forall c\in \mathbb{R} \: , \: \forall \alpha,\beta \in \Omega^r(M) \: , \:
   \forall \omega \in \Omega^s(M) \: , \: \forall \tau \in \Omega^k(M)$
\\

Analogamente alla \ref{prop:base1}, vale anche
\begin{proposition}\label{prop:base2}
   Sia $M$ varietà differenziale e $p \in M$. Sia $\omega \in \Omega^r(T_p(M))$,
   allora esistono le funzioni $a_{i_1,\dots i_r}$ tali che in coordinate locali\\
   $$
      \omega = \sum_{(i_1,\dots, i_r)} a_{i_1,\dots, i_r}
         dx^{i_1} \wedge \dots \wedge dx^{i_r}
   $$
\end{proposition}

Si osservi che per la proprietà di antisimmetria se $u = v$ si ha
$$\omega(v,\dots,u,\dots) = - \omega(u,\dots,v,\dots) = - \omega(v,\dots,u,\dots) = 0$$
Per una varietà di dimensione $n$ si ha al massimo $n$ vettori linearmente indipendenti.
Se si prende in considerazione un vettore aggiuntivo, esso è combinazione lineare dei precedenti.\\
Di conseguenza \textbf{tutte le (r$>$n)-forme} su una varietà di dimensione $n$ sono \textbf{nulle}.\\
Se $u= a_1v^1 + \dots + a_nv^n$
$$
   \omega(v^1,\dots,v^n,u) = \dots = a_1\omega(v^1,\dots,v^n,v^1) + \dots
      + a_n\omega(v^1,\dots,v^n,v^n) = 0
$$
%------------------------------------------------------------------------------%
%------------------------------------------------------------------------------%
\subsection{Differenziale esterno}
Si definisce l'operatore che a una r-forma
$\omega \in \Lambda^r(p)$ associa la (r+1)-forma $d\omega \in \Lambda^{r+1}(p)$,
definita in coordinate locali da\\
$$
   \omega = \sum_{(i_1,\dots, i_r)} a_{i_1,\dots, i_r}
      dx^{i_1} \wedge \dots \wedge dx^{i_r} \to
   d\omega = \sum_{(i_1,\dots, i_r)} da_{i_1,\dots, i_r}\wedge
      dx^{i_1} \wedge \dots \wedge dx^{i_r}
$$
ossia
$$
   d : \Lambda^r(p) \to \Lambda^{r+1}(p) \quad \omega \mapsto
   d\omega = \sum_{(i_1,\dots, i_r,k)} \frac {\partial a_{i_1,\dots, i_r}}{\partial x^k} dx^k
   \wedge dx^{i_1} \wedge \dots \wedge dx^{i_r}
$$
Dall'antisimmetria del prodotto wedge discende immediatamente che $d^2\omega = d(d\omega) = 0$
\begin{equation*}
   \begin{split}
      d^2\omega & = d \left( \sum_{(i_1,\dots, i_r,k)}
            \frac {\partial a_{i_1,\dots, i_r}}{\partial x^k} dx^k
               \wedge dx^{i_1} \wedge \dots \wedge dx^{i_r} \right)
          = \sum_{(i_1,\dots, i_r,k)}
            d\left( \frac {\partial a_{i_1,\dots, i_r}}{\partial x^k} \right)
               \wedge dx^k \wedge dx^{i_1} \wedge \dots \wedge dx^{i_r} \\
       & = \sum_{(i_1,\dots, i_r,k,j)}
            \frac {\partial^2 a_{i_1,\dots, i_r}}{\partial x^k \partial x^j}
               \wedge dx^j \wedge dx^k \wedge dx^{i_1} \wedge \dots \wedge dx^{i_r}
         = 0
   \end{split}
\end{equation*}
in quanto contrazione del termine simmetrico $\frac {\partial^2 a_{i_1,\dots, i_r}}
{\partial x^k \partial x^j}$ e del termine antisimmetrico $dx^j \wedge dx^k $\\

Si definisce forma \textbf{esatta} una r-forma $\omega$ se esiste una
(r-1)-forma $\alpha$ che verifica $\omega = d\alpha$\\
Si definisce forma \textbf{chiusa} una r-forma $\omega$ tale che $d\omega = 0$.\\
Segue immediatamente che ogni forma esatta è chiusa. L'inverso è vero solo localmente.
\begin{lemma}{(di Poincarè)}
    Sia $M \subset \mathbb{R}^n$ una palla aperta. Una r-forma $\omega$ chiusa
    definita su $M$ è esatta.
\end{lemma}
%------------------------------------------------------------------------------%
\subsection{Coomologia di de Rham}
\textcolor{red}{(da scrivere)}
%------------------------------------------------------------------------------%
\subsection{Integrazione}
L'integrazione di una forma su una varietà può essere definita
tramite coordinate locali e ricondotta a integrazione su aperti di $\mathbb{R}^n$.
Una trattazione rigorosa richiederebbe l'introduzione del concetto di
\emph{partizione dell'unità}, che esula dallo scopo di questo elaborato.
Si vedano \cite{boothby},\cite{nakahara} per una trattazione rigorosa. \\
In maniera intuitiva, sia $(U,\phi)$ una carta della
varietà $M$, dove $U$ intorno del punto $p \in M$, e $R = \phi(U)$ e
$\omega = dx^1 \wedge \dots \wedge dx^n$ una n-forma.
Sia $f : U \to \mathbb{R}$ una funzione integranda.
Si definisce automaticamente la misura di integrazione $d\mu = dx^1 \dots dx^n$.
e si può dare senso all'espressione in coordinate locali:
$$
   \int_U f\omega = \int_R f(x_1,\dots,x_n) dx^1 \dots dx ^n
$$
Si vuole poi ripetere questa operazione per tutte le carte di un atlante dato e
"incollare" assieme i risultati in maniera che l'integrale sulla varietà sia uguale
alla somma degli integrali sulle singole carte.\\

Si osserva che è possibile integrare solamente n-forme su varietà di dimensione $n$
perchè le forme con $r>n$ sono tutte nulle, e gli spazi di dimensione $r<n$ hanno
misura nulla in $\mathbb{R}^n$.\\

\textbf{Hodge star}:\\
\textbf{Teorema di Stokes}:\\
%------------------------------------------------------------------------------%
%------------------------------------------------------------------------------%
\subsection{Varietà Riemanniane}
\begin{definition}
   Una \emph{metrica Riemanniana} $g$ su una varietà $M$ è un (2,0)-campo tensoriale
   su $M$ che per ogni punto $p \in M$ soddisfa:
   \begin{enumerate}
      \item $ g_p(U,V) = g_p(V,U) $
      \item $ g_p(U,U) \geq 0 $ dove $ g(U,U) = 0 \iff U = 0$
   \end{enumerate}
   dove $U,V \in T_p(M)$.\\

   Una \emph{metrica pseudo-Riemanniana} $g$ su una varietà $M$ è un
   (2,0)-campo tensoriale su $M$ che per ogni punto $p \in M$ soddisfa:
   \begin{enumerate}
      \item $ g_p(U,V) = g_p(V,U) $
      \item $ g_p(U,V) = 0 \: \forall U \in T_p(M) \Rightarrow V = 0$
   \end{enumerate}
\end{definition}

Data una carta $(U,\phi)$ di $M$ con coordinate $\{x^\mu\}$ il tensore $g$ può
essere scritto come
$$ g_p = g_{\mu\nu}(p) dx^\mu dx^\nu =: ds^2$$
Dove $g_{\mu\nu}(p)$ può essere considerato come la $\mu\nu$-esima entrata di una matrice.\\
Il numero $(p,n)$ di autovalori positivi $p$ e negativi $n$ è detto \emph{indice} della
metrica. Se $n=1$ la metrica è detta Lorentziana.\\
Si può diagonalizzare la metrica e riscalare gli autovettori in modo da ottenere
solamente $\pm 1$ sulla diagonale. Ad esempio si ha la metrica Euclidea
$\delta = diag(1,1,\dots,1)$ o Minkowskiana $\eta = diag(-1,1,...,1)$.\\

Se una varietà differenziale $M$ è dotata di una metrica Riemanniana $g$, la coppia $(M,g)$
è detta \textbf{varietà Riemanianna} (analogamente se $g$ è pseudo-Riemann).\\

%------------------------------------------------------------------------------%
%------------------------------------------------------------------------------%
\subsection{Derivata covariante}
\textcolor{red}{Cosa è, Cosa è in questo contesto. (piccola intro)}\\
Si vuole estendere il concetto di derivata direzionale agli (r,s)-tensori definiti
su una varietà differenziale $M$\\

Sia $X$ un campo vettoriale sulla varietà $M$ (supponiamo $M = \mathbb{R}^n$,
per semplicità), $p \in M$ e $h \in M$ pensato come piccolo spostamento da $p$, in $M$.
Volendo definire la derivata direzionale di un campo vettoriale nel modo usuale
$$
   \lim_{|h| \to 0} \frac{X_{p+h}-X_p}{|h|}(f)
$$
si riscontra subito un problema. I due vettori $X_{p+h} \in T_{p+h}(M)$ e
$X_p \in T_p(M)$ appartengono a due spazi differenti e non possono essere confontati.\\
Occorre un modo di trasportare il vettore $X_{p+h}$ da $T_{p+h}(M)$ a $T_p(M)$
lasciandolo inalterato. Questo processo è chiamato \textbf{trasporto parallelo}.
Purtroppo non esiste una maniera univoca per trasportare un vettore tangente in
una varietà, quindi è necessario specificare come viene effettuato il trasporto
parallelo.\\

\textcolor{red}{(qua si può tagliare o riassumere ulteriormente)}
\begin{definition}{(Connessione affine):}\label{def:affineconnection}
   Una connessione affine $\nabla$ è una mappa $\nabla : \mathcal{X}(M) \times
   \mathcal{X}(M)\to \mathcal{X}(M)$ t.c $(X,Y) \mapsto \nabla_X Y$ che verifica
   \begin{enumerate}
      \item $ \nabla_X (Y+Z) = \nabla_X Y + \nabla_X Z $
      \item $ \nabla_{X+Y} Z = \nabla_X Z + \nabla_Y Z $
      \item $ \nabla_{fX} Y = f\nabla_X Y $
      \item $ \nabla_X (fY) = X[f] Y + f\nabla_X Y $
   \end{enumerate}
   per $f \in C^\infty(M)$ e $X,Y,Z \in \mathcal{X}(M)$
\end{definition}

Sia $(U,\phi)$ una carta di $M$ con coordinate $x = \phi(p) \:,\: p\in M$.
L'azione di $\nabla$ sugli elementi $\{ e_\mu = \partial/ \partial x^\mu \}$
della base di $T_p(M)$ ne determina univocamente l'azione su qualsiasi vettore $X_p$.\\

Si definiscono i \textbf{coefficienti di connessione} $\Gamma^\lambda_{\nu\mu}$ da
(per semplicità di notazione si indica $\nabla_{e_\nu} = \nabla_\nu$):
$$
   \nabla (e_\nu,e_\mu) = \nabla_\nu e_\mu =
      e_\lambda \Gamma^\lambda_{\nu\mu}
$$
Presi allora due campi $X = X^\mu e_\mu \: , \: Y = Y^\nu e_\nu \:
\in \mathcal{X}(M)$ si ha:
$$
   \nabla_X Y
      = X^\mu \nabla_\mu(Y^\nu e_\nu)
      = X^\mu(e_\mu[Y^\nu]e_\nu + Y^\nu \nabla_\mu e_\nu)
      = X^\mu \left(\frac{\partial Y^\lambda }{\partial x^\mu}
         + Y^\nu \Gamma^\lambda_{\nu\mu} \right)
$$
il risultato dipende solo dalle $(\mathrm{dim} M)^3$ funzioni $\Gamma^\lambda_{\nu\mu}$.\\

Una connessione $\nabla$ è dette simmetrica se in coordinate vale
$\Gamma^\lambda_{\nu\mu} = \Gamma^\lambda_{\mu\nu}$

\begin{theorem}{(Teorema fondamentale della geometria (pseudo-)Riemanniana):}
  Sia $(M,g)$ una varietà (pseudo-)Riemanniana. Esiste un unica connessione
  \emph{simmetrica} che è \emph{compatibile} con la metrica $g$. Questa connessione
  è chiamata \textbf{connessione di Levi-Civita} definita da
  $$
       \Gamma^\kappa_{\mu\nu} = \frac{1}{2} g^{\kappa\lambda}
           (\partial_\mu g_{\lambda\nu}  + \partial_\nu g_{\lambda\mu}
              - \partial_\lambda g_{\mu\nu}) = : \chrsym{\kappa}{\mu\nu}
  $$
\end{theorem}
$\chrsym{\kappa}{\mu\nu}$ è detto simbolo di Christoffel.
Si veda \cite{nakahara} per la dimostrazione del teorema.\\

\section{Gruppo e azione di gruppo}
\begin{definition}{(Gruppo)}:
   Sia $G$ un insieme e $*$ un'operazione binaria su $G$. $G$ si definisce
   \emph{gruppo} con l'operazione $*$ se valgono le seguenti proprietà:
   \begin{enumerate}
      \item $ \forall f,g,h \in G \to f*(g*h) = (f*g)*h \quad$ associativa
      \item $ \exists e \in G $ t.c. $ e*g = g*e = g ,\quad \forall g \in G \quad$
       esistenza elemento neutro
      \item $ \forall a \in G \exists a^{-1} \in G $ t.c. $a*a^{-1} = a^{-1}*a = e \quad$
         esistenza dell'inverso
   \end{enumerate}
   Se vale anche la proprietà commutativa, il gruppo è detto \emph{abeliano}
   \begin{enumerate}\setcounter{enumi}{3}
      \item $ \forall g,h \in G \to g*h = h*g \quad$ commutativa
   \end{enumerate}
\end{definition}
Sovente si usa omettere il simbolo dell'operazione: $ g*h = gh $
%------------------------------------------------------------------------------%
\begin{definition}{(Azione di gruppo):}\label{def:groupaction}
   Sia $G$ un gruppo e $X$ un insieme non vuoto. \\
   Si definisce l'azione destra di $G$ su $X$
   una funzione $\phi : X \times G \to X$ che $(x,g) \mapsto \phi(x,g) = x \cdot g$
   con le seguenti proprietà:
   \begin{enumerate}
      \item $x \cdot e = x \quad$ identità
          ($e$ denota l'elemento neutro di $G$)
      \item $\forall g,h \in G \: , \: \forall x \in X \to
         x\cdot (gh) = (x \cdot g) \cdot h \quad $ compatibilità
   \end{enumerate}
   Si definisce l'azione sinistra di $G$ su $X$
   una funzione $\phi : G \times X \to X$ che $(g,x) \mapsto \phi(g,x) =  g \cdot x $
   con le seguenti proprietà:
   \begin{enumerate}
      \item $e \cdot x = x \quad$ identità
          ($e$ denota l'elemento neutro di $G$)
      \item $\forall g,h \in G \: , \: \forall x \in X \to
         (gh) \cdot x  = g \cdot (h \cdot x) \quad $ compatibilità
   \end{enumerate}
\end{definition}
La differenza tra azione destra e azione sinistra sta nell'ordine in cui $gh$
agisce sull'insieme, evidente per gruppi non abeliani.\\

L'azione di $G$ su $X$ si dice:
\begin{itemize}
   \item \emph{transitiva}: \quad $\forall x,y \in X \: \exists g \in G$ t.c. $g\cdot x = y$
   \item \emph{libera}: \quad Sia $g \in G$. Se $\exists x \in X$ t.c.$ g\cdot x = x
      \Rightarrow g$ è l'identità
\end{itemize}
%------------------------------------------------------------------------------%
\subsection{Gruppi e Algebre di Lie}
Si veda la sezione \ref{sec:vardiff} per una breve trattazione su varietà differenziali,
spazi tangenti e campi vettoriali.

\begin{definition}{(Gruppo di Lie):}\label{def:liegroup}
   Un gruppo di Lie $G$ è una varietà differenziale\footnote{Si veda la sezione
   \ref{sec:vardiff}}, dotata di di una struttura di gruppo in cui la
   moltiplicazione e l'inverso sono funzioni lisce. In altre parole è liscia la mappa
      $$ (x.y) \mapsto x^{-1}y \quad \forall x,y \in G$$
\end{definition}
La dimensione del gruppo equivale alla dimensione della varietà.
Esempi di gruppi di Lie sono $GL(n,\mathbb{R})$ e $GL(n,\C)$ i gruppi
delle matrici quadrate $n \times n$ non singolari a coefficenti reali/complessi.\\
Vale il seguente Teorema che non verrà qui dimostrato.
\begin{theorem}
   Ogni sottogruppo chiuso $H$ di un gruppo di Lie $G$ è un sottogruppo di Lie
\end{theorem}
Che garantisce che $O(n), SO(n), SL(n,\mathbb{R})$ sono sottogruppi di Lie di
$GL(n,\mathbb{R})$.\\

Essendo un gruppo di Lie $G$ contemporaneamente gruppo e varietà differenziale, si
può definire l'azione di $G$ (come gruppo) su se stesso (come varietà differenziale).

\begin{definition}\label{def:adjrep}
   Sia $h \in G$ si definisce la \emph{rappresentazione aggiunta} di G come
   l'omomorfismo $ ad_a : G \to G $
   $$
      ad_h : g \mapsto h g h^{-1}
   $$
\end{definition}
E induce la mappa tra gli spazi tangenti $(ad_h)_* : T_g(G) \to T_{h^{-1}gh}$.
Poichè $ad_h e = e$ si può valutare la restrizione di $(ad_h)_*$ al solo $g=e$,
ottenendo una mappa di $T_e(G)$ in se stesso
$$
   Ad_h : T_e(G) \to T_e(G) \quad Ad_h := (ad_h)_* |_{T_e(G)}
$$

%------------------------------------------------------------------------------%
\begin{definition}{(Algebra di Lie):}\label{def:LieAlgebra}
   Un'algebra di Lie $\mathfrak{g}$ è uno spazio vettoriale (su un opportuno campo)
   dotato di un'operazione binaria $[,] : \mathfrak{g} \times \mathfrak{g} \to \mathfrak{g}$
   che rispetta le seguenti proprietà:
   \begin{enumerate}
       \item \emph{Bilinearità} \quad $[ax+by,z] = a[x,z]+b[y,z]$ \quad e \quad
          $[x,ay+bz] = a[x,y] + b[x,z]$ \\
          per tutti gli scalari $a,b$ e $\forall x,y,z \in \mathfrak{g}$
       \item \emph{Identità di Jacobi} \quad $\forall x,y,z \in \mathfrak{g} \quad
          [x,[y,z]] + [z,[x,y]] + [y,[z,x]] = 0$
       \item \emph{Anticommutatività} \quad $[x,y] = -[y,x] \quad
          \forall x,y \in \mathfrak{g} $
       \item \emph{Alternanza} \quad $[x,x]=0 \quad \forall x \in \mathfrak{g}$ \quad
          (discende dalla precedente)
   \end{enumerate}
\end{definition}
La dimensione dell'algebra di Lie è la sua dimensione come spazio vettoriale.\\
Un set di elementi dell'algebra $\mathfrak{g}$ si dice set di \emph{generatori}
se la sottoalgebra più piccola che lo contiene è $\mathfrak{g}$ stessa.\\

Ad ogni gruppo di Lie si può associare un'algebra di Lie.\\

Siano $a,g \in G$. Si definiscono la \textbf{traslazione destra} $R_a : G \to G$
e la \textbf{traslazione sinistra} $L_a : G \to G$
$$
   R_a(g) = ga \quad,\quad L_a(g) = ag
$$
$R_a$ e $L_a$ sono diffeomorfismi di $G$ in se stesso per costruzione e inducono
quindi le mappe sugli spazi tangenti $L_{a*} : T_g(G) \to T_{ag}(G)$ e
$R_{a*} : T_g(G) \to T_{ga}(G)$ (si veda \ref{}).\\

Dato un gruppo di Lie $G$ esiste una speciale classe di campi vettoriali
(si veda \ref{}) che sono invarianti sotto l'azione di gruppo\footnote{
Ciò non accade, ad esempio, con le varietà differenziali usuali, in cui non vi è
modo di evidenziare una classe privilegiata di campi vettoriali
}.\\

Sia X un campo vettoriale sul gruppo di Lie $G$. X si dice campo vettoriale
\textbf{invariante a sinistra} se vale $L_{a*}X|_g = X|_{ag}$ (analogamente,
invariante a destra).\\

Un vettore tangente all'identità $e$ del gruppo di Lie $V \in T_e(G)$
definisce un unico campo vettoriale $X_V$ su $G$ invariante a sinistra tramite
l'azione sinistra (analogamente a destra)
$$
   X_V|_g : = L_{g*} V \quad , \quad g \in G
$$
Poichè
$$
   X_V|_{ag} = (L_{ag})_* V = (L_a L_g)_* V = (L_a)_*(L_g)_* V = (L_a)_* X_V|_g
$$

Viceversa, un campo vettoriale $X_V$ invariante a sinistra (analogamente a destra)
definisce un unico vettore $V$ tangente all'identità in $G$
$$
   V := X_V|_e \quad \in T_e(G)
$$
\begin{definition}\label{def:GroupLieLlgebra}
  Si indica con $\mathfrak{g}$ l'insieme dei campi vettoriali su $G$ invarianti
  a sinistra (denotando con $\mathcal{X}(G)$ l'insieme dei campi vettoriali su G)
  $$
    \mathfrak{g} := \{ X_V \in \mathcal{X}(G) \mathrm{\: t.c. \:}
       L_{a*}X_V|_g = X_V|_{ag} \forall a \in G \}
  $$
\end{definition}
La mappa che a un vettore tangente all'identità in $G$ associa un campo vettoriale
invariante a sinistra
$$
   T_e(G) \to \mathfrak{g}  \quad V \mapsto X_V
$$
è un isomorfismo. \\
Si dimostra che $\mathfrak{g}$ è uno spazio vettoriale con l'operazione di
traslazione a sinistra (?), e quindi $\mathfrak{g} \cong T_e(G)$.
In particolare dim $\mathfrak{g}$ = dim $G$.\\

Resta da definire un'operazione di parentesi di Lie affinchè lo spazio dei campi
invarianti a sinistra sia un'algebra di Lie.\\
\begin{definition}\label{def:parentesiLie}
   Si definisce parentesi di Lie tra due campi $X,Y \in \mathcal{X}(G)$ l'operazione
   $[,] : \mathcal{X}(G) \times \mathcal{X}(G) \to \mathcal{X}(G)$ definita da
   $$
      [X,Y]f = X(Yf) - Y(Xf) \quad \forall f \in C^\infty(G)
   $$
\end{definition}
Si sottolinea che $Xf$ così scritto è una funzione liscia definita su $G$ da
$Xf : G \in \mathbb{R}$ che agisce $p \mapsto X_p[f]\in \mathbb{R}$\\

Si verifica immediatamente che l'operazione così definita rispetta le proprietà
definite in \ref{def:LieAlgebra}\footnote{Intuitivamente, $[X,Y]$ è il commutatore
dei campi $X$ e $Y$, ed è noto che un commutatore rispetti suddette proprietà.}\\

Siano $X,Y \in \mathcal{X}(G)$ e si fissino due punti $g, ag \in G$ dove
$ag = L_a g$. Applicando $L_{a*}$ a $[X,Y]$ si ha
$$
   L_{a*}[X,Y]|_g = [L_{a*}X|_g,L_{a*}Y|_g] = [X,Y]|_{ag} \in \mathfrak{g}
$$
quindi $\mathfrak{g}$ è chiuso rispetto all'operazione $[,]$ così definita.\\

\begin{definition}\label{}
   L'insieme $\mathfrak{g}$ dei campi vettoriali su $G$ invarianti a sinistra
   ($\mathcal{X}(G)$)
   dotato delle parentesi di lie \ref{def:parentesiLie} si definisce
   \textbf{Algebra di Lie} associata al gruppo di Lie $G$.
\end{definition}
L'algebra di Lie associata a un gruppo viene indicata con lo stesso nome del gruppo,
in carattere gotico minuscolo, ad esempio $SO(n)\to \mathfrak{so(n)}$.\\

\begin{definition}
   Identificando $\mathfrak{g}$ con $T_e(G)$ tramite l'isomorfismo dell'azione sinistra
   (indicato qua con $\lambda : T_e(G) \to \mathfrak{g}$) e richiamando la definizione
   \ref{def:adjrep}, si definisce la \textbf{mappa aggiunta} del gruppo di Lie $G$.
   $$
      Ad : G \times \mathfrak{g} \to \mathfrak{g} \quad , \quad
      (h,X) \mapsto \lambda \circ Ad_h \circ \lambda^{-1}(X)
   $$
   dove $h \in G$, $X \in \mathfrak{g}$.
   Applicando la definizione si trova che $Ad_a Ad_b = Ad_ab$ e $Ad_{a^{-1}} = Ad_a^{-1}$ .\\
\end{definition}

\begin{definition}\label{def:inducedvectfield}
   Sia $G$ un gruppo di Lie che agisce sulla varietà $M$ a sinistra. Sia $V \in T_e(G)$
   un vettore tangente all'identità di $G$.
   Si definisce il \textbf{campo vettoriale indotto} da $V$ il campo $V^\#$ su $M$
   $$
      V^\# |_p = \left. \frac{\dd}{\dd t} (\exp(tV)p) \right |_{t=0} \quad p \in M
   $$
   E si definisce quindi una mappa $\# : T_e(G) \to \mathcal{X}(M)$
   con $V \mapsto V^\#$
\end{definition}

Nel caso di un gruppo di Lie $G$ di matrici reali, l'algebra di Lie può essere
formulata in termini della funzione esponenziali di matrici, ovvero
$$
\mathfrak{g} = \{ A \in \mathrm{Mat}(n,\C) \: | \:
\forall t \in \mathbb{R} \:\to \exp (tA) \in G \}
$$
Gli elementi dell'algebra di Lie vengono detti \emph{generatori} del gruppo di Lie
e, come visto sopra, la struttura del gruppo $G$ è completamente determinata
dalle costanti di struttura. In questo caso in cui $G$ è un gruppo di matrici,
la parentesi di Lie è il commutatore di matrici e quindi le costanti di struttura
si dice sono determinate dalle regole di commutazione dei generatori $\{T_\mu\}$:
$$
   [T_\mu, T_\nu] = T_\mu T_\nu - T_\nu T_\mu
      = c_{\mu\nu}^{\hphantom{\mu\nu}\lambda} T_\lambda
$$

\textcolor{red}{(non ancora usato)}\\
Poichè si ha che $\mathfrak{g} \cong T_e(G)$ sia $\{ V_1,\dots,V_n \}$ una base
per $T_e(G)$. Questa definisce, tramite l'azione sinistra, un set di campi vettoriali
$\{ X_1, \dots, X_n \} \in \mathfrak{g}$ linearmente indipendenti in ogni punto $g \in G$
$$
   X_\mu |_g := L_{g*}V_\mu \quad \forall \mu = 1,\dots,n
$$
che per ogni $T_g(G)$ è una base. Poichè anche $[X_\mu,X_\nu]|_g \in \mathfrak{g}$
nel punto $g$, può essere sviluppato $ \forall g \in G$ in termini dei vettori
$\{X_\mu|_g\}$, e quindi si ha:
$$
   [X_\mu,X_\nu] = \sum_\lambda c_{\mu\nu}^{\hphantom{\mu\nu}\lambda} X_\lambda
$$
i coefficienti $c_{\mu\nu}^{\hphantom{\mu\nu}\lambda}$ si chiamano
\textbf{costanti di struttura}.
Si dimostrano essere indipendenti dal particolare $g \in G$ preso in considerazione,
e determinano quindi completamente la struttura del gruppo di Lie $G$ (Teorema di Lie).\\


%------------------------------------------------------------------------------%

\section{Fibrati}\label{}
Si vuole iniziare con un esempio per rendere più chiaro l'argomento.
%------------------------------------------------------------------------------%
%------------------------------------------------------------------------------%
\subsection{Fibrato Tangente}\label{sec:fibratotangente}
Sia $M$ una varietà differenziale di dimensione $n$. Si definisce il \emph{fibrato tangente}
su $M$ (detta \emph{spazio base}) l'unione di tutti gli spazi tangenti alla varietà, indicato con $TM$.
   $$ TM := \bigcup\limits_{p \in M} T_p(M) $$
Si consideri una carta $(U,\phi)$, $U$ intorno di $p \in M$ e $x^i = \phi^i(p)$
coordinate. Gli elementi dello spazio $TU = \cup_{p \in U} T_p(M)$ sono
individuati da un punto $p \in U$ e un vettore tangente $V = \sum_i V^i
\left. \frac{\partial}{\partial x^i} \right|_p \in T_p(M)$. Per costruzione
$U$ è omeomorfo all'aperto $\phi(U) \subset \mathbb{R}^n$ e $T_p(M)$ è omeomorfo
a $\mathbb{R}^n$ stesso tramite l'identificazione tra derivata direzionale e
vettore (si veda la sezione \ref{sec:vardiff}).\\
Allora ogni punto $P \in TU$ può essere identificato con il punto
$(x^1,\dots,x^n,V^1,\dots,V^) \in \mathbb{R}^n \times \mathbb{R}^n$ in maniera
univoca. $TU$ è quindi una varietà differenziale di dimensione $2n$.\\
Inoltre $TU$ è identificato con $\mathbb{R}^n \times \mathbb{R}^n$ ed è
esattamente decomposto nel prodotto diretto $U \times \mathbb{R}^n$, cioè
ogni $P \in TU$ può essere scritto come $(p,V),\: p\in U V \in T_p(M)$.\\
Si può quindi definire la \emph{proiezione} $\pi : TU \to U \:,\: P=(p,V) \mapsto p$.
Lo spazio $T_p(M) = \pi^{-1}(p)$ viene detto \emph{fibra} in $p$.\\
Se $M = \mathbb{R}^n$ si ha che $TM = \mathbb{R}^n \times \mathbb{R}^n$ e si dice
che il fibrato ha struttura \emph{banale}. In generale però si ha una struttura
non banale ed occorre tener conto di tutte le carte possibili.\\
Siano $(U,\phi)$ e $(V,\psi)$ due carte tali che $U \cap V \neq \varnothing
\:,\: p\in U \cap V$. Siano $x^i = \phi^i(p)$ e $y^j = \psi^j(p)$, e sia $V \in T_p(M)$.
$V$ in coordinate è espresso come
$$
   V = \sum_i V^i  \left. \frac{\partial}{\partial x^i} \right |_p
     = \sum_j V'^j \left. \frac{\partial}{\partial y^j} \right |_p
     = \sum_{j,k}  V^k \left. \frac{\partial y^j}{\partial x^k} \right |_p
        \left. \frac{\partial}{\partial y^j} \right |_p
$$
dove $V'^j = \sum_k \left. \frac{\partial y^j}{\partial x^k}\right |_pV^k $.
L'applicazione $\psi \circ \phi$ deve essere invertibile, quindi la matrice Jacobiana
deve essere non singolare, cioè $J^j_i = \frac{\partial y^j}{\partial x^i} \in GL(n,\mathbb{R})$.
Il gruppo $GL(n,\mathbb{R})$ viene chiamato \emph{gruppo di struttura} di $TM$.
Le coordinate delle fibre, in seguito a un cambio di coordinate, risultano ruotate
per un elemento del gruppo di struttura.\\
Infine si definisce \emph{sezione} di $TM$ una mappa liscia $s : M \to TM$ tale
che $\pi \circ s = id_M$, ossia che a $p \in M$ associa un elemento di $TM$ $(p,V)
\: ,\: V \in T_p(M)$. Se $s$ è definita solo in un intorno $U$ viene detta sezione locale.\\

Avendo definito una struttura differenziale su $TM$, un campo vettoriale su $M$
può essere visto come una mappa liscia $M \to TM$ che a $p \in M$ associa $V_p \in T_p(M)$\\

%------------------------------------------------------------------------------%
%------------------------------------------------------------------------------%
\subsection{Fibrato}
Siano $M$ (detta \emph{spazio base}) e $F$ (detta \emph{fibra}) varietà diferenziali
(si pensi all'analogia con il fibrato tangente in cui $F=T_p(M)$) e sia
$\mathcal{U} = \{U_i\}$ un ricoprimento aperto di $M$.\\

Intuitivamente, un \emph{fibrato} $E$ su $M$ con fibra $F$ è una varietà diferenziale
(detta anche \emph{spazio totale}) che è localmente un prodotto diretto di $M$ e $F$,
ossia il fibrato $E$ è descritto topologicamente in ogni intorno $U_i$
dalla varietà prodotto $U_i \times F$.\\

Si definisce una funzione $\pi : E \to M$ continua e suriettiva (la
\emph{proiezione di fibra}) che mappa ogni fibra $F_p = \{ (p,f) | f \in F\}
\subset E$ nel punto $ p\in M$, e che rispetti la seguente condizione.\\

Si definisce un'operazione duale alla proiezione, la \emph{sezione} del fibrato
come una mappa tra lo spazio base $M$ ed il fibrato $E$
$$
s : M \to E \quad \mathrm{t.c.}\quad \pi \circ s (p) = p \quad \forall p \in M
$$
Se la sezione è definita solo in un intorno $U$ di un punto $p$ è detta \emph{sezione locale.}


\begin{axiom}\textbf{Condizione di non trivialità:} \label{nontriv}
      Per ogni punto $p \in M$ esiste un intorno $U_i \in \mathcal{U}$ di $p$ e un isomorfismo
      $\phi_i : U_i \times F \to \pi^{-1}(U_i) \subset E$ tale che per ogni
      $(p,f) \in U_i \times F$ si ha $\pi \circ \phi_i(p,f) = p$
\end{axiom}

Occorre specificare inoltre un insieme di \emph{funzioni di transizione} $\{\Phi_{ij}\}$ che
descrivono come si trasformano le coordinate delle fibre nella regione di
sovrapposizione tra due intorni $U_i \cap U_j$. Per ogni $x \in U_i \cap U_j$ fissato si
cosidera $\phi_{i,x}$ come una mappa di $F$ in $F_x$.
Allora si definiscono le mappe
\begin{equation}\label{eq:transfunctions}
   \Phi_{ij} : F \to F \quad , \quad \Phi_{ij} = \phi_i^{-1} \circ\phi_j
\end{equation}
che rispettano le condizioni:
\begin{equation}
   \Phi_{ii} = id \quad , \quad \Phi_{ij} \circ \Phi_{jk} = \Phi_{ik}
   \quad \forall x \in U_i \cap U_j \cap U_k
\end{equation}
Grazie a queste proprietà, le funzioni di transizione formano un gruppo $G$ detto
\emph{gruppo di struttura} del fibrato, che agisce su $F$ a sinistra.\\
Gli elementi di $G$, le funzioni di transizione, sono anche detti
\emph{local trivialization}.\\

Sebbene la topologia locale del fibrato sia banale, le funzioni di transizione
possono essere fortemente influenzate dalla topologia globale a causa di torsioni
relative tra fibre adiacenti (si veda l'esempio del nastro di Möbius). Un fibrato
è completamente determinato dalle sue funzioni di transizione.\\

\begin{definition}
   Un \emph{fibrato} $E$ con fibra $F$ sullo spazio base $M$ è uno spazio topologico
   $E$ dotato di una proiezione $\pi : E \to M$ che soddisfa la condizione di
   non trivialità \ref{nontriv}.
\end{definition}

\begin{example}\emph{(Cilindro): }\label{ex:cilindro}
   Il cilindro è il primo esempio di fibrato banale, ossia la cui topologia globale
   è quella prodotto diretto $E = M\times F$. Sia lo spazio base $M=S^1$ il cerchio
   unitario, parametrizzato dall'angolo $\theta \in [0,2\pi]$ e sia $F$ il segmento
   parametrizzato da $t \in [-1,1]$. Sia $\mathcal{U} = U_+ \cup U_-$ un ricoprimento
   formato dai due intorni semicircolari:
   \begin{equation*}
      \begin{aligned}
         U_+ &= \{\theta : \epsilon < \theta < \pi + \epsilon \} \quad ,&
         U_- &= \{\theta : \pi - \epsilon < \theta < 2\pi + \epsilon = \epsilon \}
      \end{aligned}
   \end{equation*}
   Il fibrato consiste in
      $$ U_+ \times F \: , \: (\theta_+,t_+) \mathrm{\quad e \quad}
         U_- \times F \: , \: (\theta_-,t_-) $$
  e le funzioni di transizione che legano $t_+$ e $t_-$ sono definite in
  $U_+\cap U_- = A \cup B$
  $$ A = \{ -\epsilon < \theta < \epsilon \} \quad , \quad
     B = \{ \pi-\epsilon < \theta < \pi + \epsilon \} $$
  Le funzione di transizione sono:
  $$ \begin{cases}
     t_+= t_- \mathrm{\: in \:} A \\
     t_+= t_- \mathrm{\: in \:} B
  \end{cases}$$
  Si ha quindi un fibrato banale uguale al cilindro $E = S^1 \times [-1,1]$.
\end{example}

\begin{example}\emph{(Nastro di Möbius): }\label{ex:mobius}
   Con la stessa notazione dell'esempio precedente, si scelgano le funzioni di
   transizione:
   $$ \begin{cases}
      t_+= t_- \mathrm{\: in \:} A \\
      t_+= -t_- \mathrm{\: in \:} B
   \end{cases}$$
   L'identificazione di $t$ con $-t$ nella regione $B$ torce il fibrato e gli dà
   la topologia non banale del Nastro di Möbius.
\end{example}

(questa parte serve solo per far capire la notazione nella definizione \ref{def:connection1form},
si può anche togliere).\\

Un \textbf{fibrato vettore} è un fibrato la cui fibra è uno spazio vettoriale.\\

Il \textbf{fibrato cotangente} è definito analogamente al fibrato tangente.
$ TM^* := \bigcup\limits_{p \in M} T^*_p(M) $\\

Siano $E,E'$ fibrati con spazio base $M,M'$ e fibra $F,F'$ rispettivamente\\

Un \textbf{fibrato prodotto} $E \times E'$ è un fibrato con spazio base
$M \times M'$ e fibra $F \oplus F'$. Gli elementi della fibra $F \oplus F'$ sono
vettori del tipo
${f}\choose{f'}$, dove  $f \in F$ e $f' \in F'$.\\

Il \textbf{fibrato prodotto tensore} $E \otimes E'$ è un fibrato vettore in cui
a ogni punto $p \in M$ è assegnato il prodotto di fibre $F_p \otimes F'_p$.
Date due basi $\{e_\alpha\}$ ed $\{f_\beta\}$ di $F$ e $F'$, la fibra prodotto
è lo spazio generato da $\{ e_\alpha \otimes f\beta \}$.

\begin{definition}\label{def:liealgebraform} \textbf{(Forma a valori in un'algebra di Lie):}
   Una k-forma differenziale su una varietà $M$ a valori in un'algebra di Lie è
   una \emph{sezione} del fibrato $(\mathfrak{g} \times M)\otimes \Lambda^k(T^*M)$
\end{definition}
%------------------------------------------------------------------------------%

%------------------------------------------------------------------------------%s
\subsection{Fibrato Principale e Connessione}
\begin{definition}{(Fibrato principale)}:
   Un \emph{fibrato principale} è un fibrato $P$ sullo spazio base $M$ la cui
   fibra $F$ è un gruppo di Lie e coincide con il gruppo di struttura
   $G$\footnote{Si veda \cite{nakahara}, \cite{shnir}, \cite{eguchi}
   per una definizione più completa}.
\end{definition}
Si indica anche con $P(M,G)$ o solamente $P$ dando per scontato la fibra
e lo spazio base, per brevità.\\

Si vuole ora generalizzare il concetto di connessione\footnote{Il concetto di
connessione generalizza quello di derivata direzionale ai tensori}
(e di trasporto parallelo) estendendolo alla struttura del fibrato principale.
Tale generalizzazione ha il vantaggio di definire un trasporto parallelo
indipenente dalla metrica (a differenza della connessione di Levi-Civita). \\

Si sceglie di utilizzare due approcci per definire la connessione su un fibrato
principale. Il primo è di suddividere lo spazio tangente al fibrato in componente
"verticale" e "orizzontale" (l'affermazione verrà chiarita in seguito).
Il secondo è di definirla come una 1-forma a valori nell'algebra di Lie che
rispetta determinate proprietà. Le due definizioni si dimostrano essere
equivalenti\footnote{Si veda \cite{nakahara}}.\\
Il primo approccio, geometrico, ha il vantaggio di essere globale e non dipendere
dalle coordinate scelte, mentre il secondo è più efficace a livello pratico e
computazionale.\\

Sia $M$ lo spazio base di dimensione $n$, $G$ il gruppo di struttura e fibra e
$P$ il fibrato principale.
Sia $(U,\phi)$ una carta con coordinate $x_\mu = \phi_\mu(p) \:,p \in M$. \\

\subsubsection{Definizione geometrica }
Sia $u$ un punto del fibrato principale $P(M,G)$, sia $p = \pi(u) \in M$ e
$G_p$ la fibra in $p$\\

Si definisce \textbf{sottospazio verticale} $V_u(P)$ un sottospazio di $T_u(P)$
tangente alla fibra $G_p$ nel punto $u$. Gli elementi di $V_u(P)$ si costruiscono
nel modo seguente.\\
Sia $A \in \mathfrak{g}$
Richiamando la definizione \ref{def:inducedvectfield}, si definisce il vettore
$A^\# \in T_u(P)$ dall'azione su una generica funzione $f \in C\infty(P)$
$$
   (A^\#)_u [f] := \left. \frac{d}{dt} f(u \exp(tA)) \right|_{t=0}
$$

Analogamente a quanto visto per i gruppi di matrici, se $A$ è nell'algebra di Lie
$g = \exp(t_0 A)$ per $t_0$ fissato è un elemento del gruppo di Lie $G$.
Al variare di $t$, $g_t = \exp(tA)$ è una curva in $G$, quindi $u = \exp(tA) = ug_t$
è una curva lungo la fibra in $p = \pi(u)$ in quanto trasla $u$ per azione
destra di $G$. La funzione $f$ è quindi ristretta alla curva $ug_t$. \\
Allora  l'espressione a destra indica il vettore ($\frac{d}{dt}$) tangente alla curva
(si pensi al vettore velocità istantanea) $ug_t$ all'istante $t=0$, cioè nel punto
$u\exp(0A) = ue = u$.\\
Il significato dell'espressione a destra è esattamente definire
un vettore tangente alla fibra passante in $u$, nel punto $u$, a partire da
un elemento dell'algebra $A$, appunto $A^\#_u \in V_u(P)$.\\

Definendo il vettore $A^\#_u$ per ogni punto $u \in P$ si costruisce il campo
vettoriale $A^\#$, detto il \textbf{campo vettoriale fondamentale} generato
dall'algebra Lie ($A \in \mathfrak{g}$, infatti).\\
È immediatamente definito l'isomorfismo $\# : \mathfrak{g} \to V_u(P)$ da $A \mapsto A^\#$\\

Si definisce \textbf{sottospazio orizzontale} $H_u(P)$ il complemento di $V_u(P)$
in $T_u(P)$ ed è univocamente determinato.

\begin{definition}\label{def:prbundleconnection}
   Sia $P(M,G)$ un fibrato principale, $u \in P$. Una \textbf{connessione} su $P$ è una
   suddivisione univoca dello spazio tangente $T_u(P)$ nei sottospazi orizzontale $H_u(P)$
   e verticale $V_u(P)$ tale che:
   \begin{enumerate}
      \item $T_u(P) = H_u(P) \oplus V_u(P)$
      \item Ogni campo vettoriale $X$ su $P$ è separato nelle componenti $X = X_h + H_v$ \\
      dove $X_h \in H_u(P)$ e $X_v \in V_u(P)$
      \item spazi orizzontali sulla stessa fibra $H_u(P)$ e $H_{ug}(P)$ sono legati
      dalla mappa $(R_g)_*$ indotta dall'azione destra, ovvero
      $ H_{ug}(P) = (R_g)_* H_u(P) \quad \forall u \in P \: , \: g\in G$
   \end{enumerate}
\end{definition}

L'idea alla base di questo approccio è la seguente.
Per valutare la derivata direzionale di un campo su una varietà $M$ si vuole
confrontare i valori del campo in due punti $p$ e $p'$ di $M$, connessi da un cammino.
Si vuole "sollevare" i punti dallo spazio base $M$ in due punti $u$ e $u'$
nel fibrato $P$ e confrontare i due spazi tangenti $T_u(P)$ e $T_{u'}(P)$.\\
Il fibrato è definito localmente come prodotto $U \times G$
(dove U è un intorno di $p$ in $M$), quindi un vettore tangente $X_u \in T_u$
può essere scomposto in due componenti (proprietà 2 della defininizione):
una lungo la base (componente orizzontale) e una lungo la fibra (componente verticale).\\
Il trasporto verticale è univocamente determinato dall'azione del gruppo $G$
(proprietà 3 della definizione), mentre il trasporto orizzontale non è unico
(come non è unico il cammino che unisce i due punti $p$ e $p'$ nello spazio base).\\
È solamente quest'ultimo, però, di interesse in quanto mappa un cammino tra
i due punti $u$ e $u'$ in un cammino attraverso le fibre.
Il concetto alla base è quindi quello di scaricare il problema sul trasporto
attraverso le fibre del campo vettoriale di cui si vuole calcolare la derivata.\\

Manca però un metodo pratico di calcolo del trasporto orizzontale, che sarà dato
dall'approccio successivo.
%------------------------------------------------------------------------------%
\subsubsection{Definizione tramite 1-forma}
\begin{definition}\label{def:connection1form}
   Una 1-forma di connessione $\omega \in \mathfrak{g} \otimes T^*P$\footnote{
   Cfr. definizione \ref{def:liealgebraform}}
   è una proiezione dello spazio $T_u(P)$ su $V_u(P)$, che è isomorfo all'algebra
   di Lie $\mathfrak{g}$, tale che
   \begin{enumerate}
      \item $\omega(A^\#) = A   \quad A^\# \in T_u(P) \: ,  \: A \in \mathfrak{g}$
      \item $R_g^* \omega = Ad_{g^-1} \omega\footnote{Definizione \ref{def:adjrep}} $ \\
   \end{enumerate}
\end{definition}
La proprietà 2 corrisponde a
 $$ R_g^* \omega_u(X) = \omega_{ug}((R_g)_*X) = g^{-1} \omega_u(X) g .$$

Lo spazio orizzontale è definito come il kernel di $\omega$
$$
   H_u(P) := \{ X \in T_u(P) \: | \: \omega (X) = 0 \}
$$
La prima proprietà garantisce che ogni vettore verticale venga mandato nell'opportuno
elemento dell'algebra, mentre la seconda garantisce il corretto trasporto dello
spazio orizzontale lungo una fibra e fa si che la definizione appena data combaci
con la precedente, come dimostrato da:
\begin{proposition}
   Lo spazio orizzontale verifica
   $$
        (R_g)_* H_u P = H_{ug} P
   $$
\end{proposition}
\begin{proof}
   Sia $u \in P$ e si consideri lo spazio orizzontale $H_u(P) = ker\omega$.
   Sia $X \in H_u(P)$ e gli si applichi la mappa indotta dall'azione sinistra
   $(R_g)_*$, costruendo $(R_g)_*X \in T_{ug}(P)$. Si ha che:
   $$
      \omega( (R_g)_* X) = R_g^* \omega(X) = g^{-1}\omega(X)g \footnote{Per la
      proprietà 2.} = 0 \footnote{Poichè $\omega(X) = 0$}
   $$
   Quindi anche $(R_g)_*X$ sta nel kernel di $\omega$ e quindi $(R_g)_*X \in H_{ug}(P)$.\\
   Poichè $(R_g)_*$ è una mappa invertibile, qualsiasi vettore $Y \in H_{ug}(P)$ è
   esprimibile come $Y = (R_g)^* X$ per qualche $X \in H_u(P)$.\\
   Di conseguenza lo spazio $T_u(P)$ è separato in $H_u(P) \oplus V_u(P)$ da $\omega$,
   rendendo la definizione di 1-forma di connessione analoga a quella di connessione
   \ref{def:prbundleconnection}.
\end{proof}
La 1-forma di connessione $\omega$ così definita prende il nome di \emph{connessione
di Ehresmann}.\\

%\begin{definition}
%   Una 1-forma canonica (o forma di Maurer-Cartan) è una $\theta : T_g(G) \to T_e(G)$
%   $$ \theta : X \mapsto (L_{g^{-1}})_* X \quad X \in T_g(G)$$
%   $L$ è l'azione sinistra.
%\end{definition}
%\footnote{Vedi \cite{nakahara}, pag 234}
%
%Quello che fa è prendere un vettore in $T_g(G)$ e lo manda in $T_e(G)$ facendo il
%trasporto parallelo. $T_e(G)$ è isomorfo a $\mathfrak{g}$\\
%L'algebra sono i campi vettoriali left invarianti ricorda \\
%
%Può essere espressa come $\theta = V_\mu \otimes \theta^\mu$.\\
%Scegliamone una particolare $\theta = g^{-1}dg$\\
%
%La connessione è definita come proiezione sul verticale e mappata nel tangente a e
%che è isomorfo all'algebra
%$\omega = \theta \circ \pi$

Si osserva che la 1-forma di connessione è definita \emph{globalmente} su P, quindi
assegnare una forma di connessione fornisce un metodo per il trasporto parallelo,
unico per tutto il fibrato $P$.\\

Si vuole ora trasportare la forma di connessione $\omega$, globale, dagli spazi
tangenti a $P$ agli spazi tangenti alla varietà $M$.\\

Sia $\{U_i\}$ un ricoprimento aperto di $M$ e siano $s_i$ sezioni locali definite
su ciascun $U_i$. Si può allora trasportare $\omega$ definendo su $U_i$ la forma
$$
   A_i := s_i^* \omega \in \mathfrak{g} \otimes \Omega^1(U_i)
$$
Viceversa, sia $A_i$ una 1-forma a valori in una algebra di Lie assegnata su su $U_i$.
Si può sempre costruire una 1-forma di connessione $\omega$ che trasportata tramite
$s_i^*$ sia esattamente $A_i$.
\begin{theorem}
   Data una 1-forma $A_i$ a valori in $\mathfrak{g}$ su $U_i$ e una sezione locale
   $s_i : U_i \to \pi^{-1}(U_i)$, esiste una 1-forma di connessione $\omega$
   tale che $A_i = s_i \omega$, che ristretta a $\pi^{-1}(U_i)$ assume la forma
   $$
      \omega|_{\pi^{-1}(U_i)} = \omega_i := g_i^{-1} \pi^* A_i g_i + g_i^{-1} d_P g_i
   $$
   dove $d_P$ è la derivata esterna su $P$ e $g_i$ la canonica local trivialization
   data da $\phi_i^{-1}(u) = (p,g_i)$ per $u = s_i(p)g_i$
\end{theorem}
Si veda \cite{nakahara} per la dimostrazione.\\

Si sottolinea l'importanza di quest'ultimo teorema in quanto la conoscenza di un'unica
forma $A_i$ su un aperto $U_i$ di $M$, unitamente alla condizione di unicità della
1-forma di connessione $\omega$, vincola tutte le altre forme defnite sugli aperti
$U_{j\neq i}$ del ricoprimento. La condizione da rispettare è la seguente\footnote{
Si veda \cite{nakahara} per la derivazione completa.
}.

Siano $U_i$ e $U_j$ due aperti del ricoprimento aperto di $M$ a intersezione non vuota.
Affinchè la forma di connessione $\omega$ sia definita \textbf{univocamente} su tutto $P$,
si deve avere sull'intersezione $U_i \cap U_j$ che $\omega_i = \omega_j$.\\
Affinchè ciò sia verificato, le forme locali $\omega_i,\omega_j$ devono verificare
la seguente condizione di compatibilità. Siano $\Phi_{ij}$ le funzioni di transizione
definite in \ref{eq:transfunctions}
$$
   s^*_j \omega (X) = \Phi_{ij}^{-1} \omega( (s_i)_*X ) \Phi_{ij}
                         + \Phi_{ij}^{-1}d\Phi_{ij}(X)
$$
Che deve essere valida per ogni campo $X \in T_p(M)$. Quindi si traduce nella
condizione per le forme $A_i$:
\begin{equation}\label{eq:condcompatibility}
   A_j = \Phi_{ij}^{-1} A_i \Phi_{ij} + \Phi_{ij}^{-1}d\Phi_{ij}
\end{equation}

Se, al contrario, sono dati $\{U_i\}$ ricoprimento aperto di $M$, $\{s_i\}$
sezioni, e $\{A_i\}$ forme che rispettano \ref{eq:condcompatibility}, si può
ricostruire la forma di connessione $\omega$\\
Si sottolinea che se il fibrato $P$ non è banale, non è possibile definire in
maniera univoca una sezione globale, quindi la forma $A_i$ esiste solo localmente.
La forza di questa costruzione è che invece la forma di connessione $\omega$
è definita globalmente su $P$.
%------------------------------------------------------------------------------%
Si vuole ora definire la derivata covariante di una
1-forma di connessione $\omega$.
In maniera intuitiva, la si vuole definire in maniera tale che trasformi allo
allo stesso modo di $\omega$ sotto l'azione del gruppo $G$. Così facendo si
può identificare la derivata covariante $D$ con la derivata esterna $d$.\\
Se la 1-forma $\omega$ trasforma come un vettore $\omega \mapsto g\omega$ per
$g \in G$, allora anche la derivata covariante trasforma $ D\omega \mapsto
g D\omega $. Si definisce allora:
   $$ D\omega := d\omega + A \wedge \omega $$
dove $A$ è la 1-forma definita in precedenza, che ripetta la legge di
trasformazione \ref{eq:condcompatibility}. In maniera analoga si può definire la
derivata covariante di 1-forme di connessione che trasformano come tensori o sono
invarianti.
\begin{definition}
   Sia $\omega$ la 1-forma di connessione di un fibrato principale $P$. Si definisce
   la 2-forma di textbf{curvatura} $\Omega$ come la derivata covariante di $\omega$
   $$ \Omega := D\omega $$
\end{definition}
Poichè $A_i$ è il pullback di $\omega$ su un aperto di $U_i \subset M$
tramite una sezione $s_i$, si definisce una 2-forma su $U_i$ tramite
\begin{equation} F_i := s_i^* \Omega \end{equation}
che rispetta la condizione di compatibilità sull'intersezione di due aperti
$U_i \cap U_j$
\begin{equation} F_j = \Phi_{ij}^{-1}F_i\Phi_{ij} \end{equation}

La 2-forma di curvatura fornisce n metodo di classificazione topopologica
dei fibrati.

\chapter{Notazione}
%
\subsubsection{Coordinate polari:}
 $\theta \in[0,\pi]$ Angolo polare\\
 $\varphi \in [0,2\pi]$ Angolo azimutale\\
 $r \in (0,\infty)$ Raggio\\

\subsubsection{Versori:}
$\vec u_x = (1,0,0)$ Versore asse x \\
$\vec u_y = (0,1,0)$ Versore asse y \\
$\vec u_z = (0,0,1)$ Versore asse z \\

$\vec u _\theta = \frac{1}{r\sqrt{r^2-z^2}}(xz,yz,-z^2-r^2)$ Versore angolo polare \\
$\vec u _\varphi = \frac{1}{\sqrt{r^2-z^2}}(-x,y,0)$ Versore angolo azimutale \\
$\vec u _r = \frac{1}{r}(x,y,z) = \frac{\vec r}{r}$ Versore radiale \\

\begin{equation}\label{eq:versors}
   \begin{cases}
       \vec u_r = \sin\theta \cos\varphi \vec u _x
                + \sin\theta \sin\varphi \vec u _y + \cos\theta \vec u _z\\
       \vec u_\theta = \cos\theta \cos\varphi \vec u _x +
                       \cos\theta \sin\varphi \vec u _y - \sin\theta \vec u _z\\
       \vec u_\varphi = -\sin\varphi \vec u _x + \cos\varphi \vec u _y
   \end{cases}
\end{equation}

\subsubsection{Funzioni:}
$\Theta$ funzione di Heaviside
$$
    \Theta(x) = \begin{cases}
       0 & x < 0\\
       1 & x > 0
    \end{cases}
$$

\subsubsection{Spazi topologici}
$S^n = \{ x \in \R^{n+1} \tc |x| = 1 \}$: Sfera n-dimensionale di raggio unitario.

\subsubsection{Gruppi di matrici}
\begin{itemize}
   \item $GL(n,\mathbb{F}) = \{M \in Mat(n,\mathbb{F}) \tc \mathrm{det} M \neq 0)\}$\\
      \ttab Gruppo Lineare Generale: Matrici $n \times n$ sul campo $\mathbb{F}$,
      invertibili.

   \item $SL(n,\mathbb{F}) = \{ M \in GL(n,\mathbb{F}) \tc \mathrm{det} M = 1)\}$\\
      \ttab Gruppo Lineare Speciale: Matrici a determinante 1.

   \item $O(n) = \{ M \in GL(n,\R) \tc  M^T M = M M^T = 1 \}$\\
      \ttab Gruppo Ortogonale: Matrici ortogonali.

   \item $SO(n) = \{ M \in O(n)\tc \mathrm{det} M  = 1  \}$\\
      \ttab Gruppo Ortogonale Speciale: Matrici ortogonali a determinante 1.

   \item $U(n) = \{ M \in GL(n,\C)\tc M^\dagger M = M M^\dagger = 1\}$\\
      \ttab Gruppo Unitario: Matrici unitarie.

   \item $SU(n) = \{ M \in U(n)\tc \mathrm{det}M = 1  \}$\\
      \ttab Gruppo Unitario Speciale: Matrici unitarie a determinante 1.
\end{itemize}

Le matrici di $SU(2)$ hanno la forma
\begin{equation}\label{eq:matSU2}
   U =
        \begin{pmatrix}
          z^* & w^* \\
          -w  & z   \\
        \end{pmatrix}
     =
       \begin{pmatrix}
          \cos \frac{\theta}{2} e^{-i\alpha} & \sin \frac{\theta}{2} e^{-i(\varphi + \alpha)} \\
         -\sin \frac{\theta}{2} e^{i(\varphi + \alpha)} & \cos \frac{\theta}{2} e^{i\alpha}  \\
       \end{pmatrix}
\end{equation}
Dove $z,w \in \C$ tali che $zz^* + ww^* = 1$
(affinchè $U^\dagger U = UU^\dagger = 1$), $\theta,\varphi$ sono gli angoli delle
coordinate polari e $\alpha$ è il parametro ciclico del gruppo $U(1)$, cioè
$g \in U(1) \to g = e^{i\alpha}$

\subsubsection{Matrici di Pauli}
\begin{equation}\label{eq:paulimatrix}
   \begin{aligned}
      \sigma_1 &= \begin{pmatrix} 0 & 1  \\ 1 & 0  \end{pmatrix}&
      \sigma_2 &= \begin{pmatrix} 0 & -i \\ i & 0  \end{pmatrix}&
      \sigma_3 &= \begin{pmatrix} 1 & 0  \\ 0 & -1 \end{pmatrix}&
   \end{aligned}
\end{equation}
\subsubsection{(pseudo)Tensore di Levi-Civita}
$$
   \varepsilon_{ijk} := \begin{cases}
      1  & \mathrm{se \:} i,j,k \mathrm{\: permutazione \:   pari \: di\:} 1,2,3 \\
      -1 & \mathrm{se \:} i,j,k \mathrm{\: permutazione \: dispari\: di\:} 1,2,3 \\
      0  & \mathrm{altrimenti}
   \end{cases}
$$

\subsubsection{Equazioni di Maxwell nel vuoto}
\begin{equation}
   \begin{aligned}
      \partial _\nu F^{\mu\nu} & = 0 ,&
      \partial _\nu \tilde{F}^{\mu\nu}
         & =  \varepsilon^{\mu\nu\alpha\beta} \partial _\nu F_{\alpha\beta}
         = 0
   \end{aligned}
\end{equation}

\begin{thebibliography}{9}
\bibitem{soardi}
   Paolo Maurizio Soardi.
   \textit{Analisi Matematica}.
   Città Studi Edizioni, 2010.

\bibitem{sernesi}
   Edoardo Sernesi.
   \textit{Geometria 2}.
   Bollati Boringhieri, 1994.

\bibitem{fulton}
  William Fulton.
  \textit{Algebraic Topology, a first course}.
  Springer, 1995.

\bibitem{boothby}
   William M. Boothby.
   \textit{An introduction to Differentiable Manifolds and Riemannian Geometry}.
   Academic Press, 1975.

\bibitem{nakahara}
   Mikio Nakahara.
   \textit{Titolo}.
   Institute of Physics publishing, Bristol and Philadelphia, 2003.

\bibitem{eguchi}
  T. Eguchi, P.B Gilkey, A.J. Hanson.
  \textit{Gravitation, Gauge Theories and Differential Geometry}.
  North Holland Publishing Company, 1980.

\bibitem{shnir}
  Yakov M. Shnir.
  \textit{Magnetic Monopoles}.
  Springer, 2005.

\bibitem{label}
   Autore.
   \textit{Titolo}.
   Edizione, Anno.

\bibitem{label}
   Autore.
   \textit{Titolo}.
   Edizione, Anno.

\bibitem{label}
  Autore.
  \textit{Titolo}.
  Edizione, Anno.

\bibitem{label}
  Autore.
  \textit{Titolo}.
  Edizione, Anno.

\end{thebibliography}

%------------------------------------------------------------------------%

\end{document}
