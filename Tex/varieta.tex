\section{Varietà differenziali}
\label{sec:vardiff}
In relazione alla definizione \ref{def:var_topologica}.
\begin{definition}{(Varietà differenziale)}
   Uno spazio topologico $M$ si dice \emph{varietà differenziale} se è una
   varietà topologica in cui le funzioni coordinate e le funzioni di transizione
   sono funzioni differenziabili $C^\infty$
\end{definition}

Struttura differenziale e dipendenza da atlante\\

Una funzione differenziabile, invertibile e con inversa differenziabile si dice
\textbf{diffeomorfismo} (quando serve, viene indicata la classe $C^k$ di differenziabilità).
%------------------------------------------------------------------------------%
%------------------------------------------------------------------------------%
\subsection{Spazio tangente e cotangente}
Si vuole generalizzare la nozione di vettore tangente ad uno spazio
$M \subset \mathbb{R}^n$ (si pensi pure a una superficie in $\mathbb{R}^3$
o una curva in $\mathbb{R}^2$). Un vettore identifica univocamente una direzione
in $\mathbb{R}^n$. Si vuole associare ad ogni vettore l'operazione derivata direzionale
lungo la direzione individuata dal vettore, valutata nel punto $p \in \mathbb{R}^n$.\\
Scelta una base $\{\vec e _i\}$ di $\mathbb{R}^n$, ogni vettore $\vec v \in \mathbb{R}^n$
si può esprimere in maniera unica come combinazione lineare degli elementi della base
$$ \vec v = v_1\vec e _1 + \dots + v_n\vec e _n $$
Sia $f$ una funzione differenziabile definita in un opportuno intorno $U$ del punto $p$
(per semplicità nel seguito si indica con $C^\infty(p)$ l'insieme delle funzioni lisce definite su
opportuni intorni del punto $p \in \mathbb{R}^n$).\\
La derivata direzionale di $f$ lungo $\vec v$ è quindi data da
   $$ \left. \frac{\partial f}{\partial \vec v} \right |_p =
         \left. v_1\frac{\partial f}{\partial x_1} \right |_p + \dots +
         \left. v_n\frac{\partial f}{\partial x_n} \right |_p $$
Inoltre, per le proprietà delle derivate $\forall \alpha,\beta \in \mathbb{R}$
e $\forall f,g \in C^\infty(p)$:
$$
      \left. \frac{\partial}{\partial \vec v}(\alpha f + \beta g) \right |_p
         = \alpha \left. \frac{\partial f}{\partial \vec v} \right |_p +
           \beta  \left. \frac{\partial g}{\partial \vec v} \right |_p
      \quad \mathrm{e} \quad
      \left. \frac{\partial}{\partial \vec v}(fg) \right |_p
         = \left. \frac{\partial f}{\partial \vec v} \right |_p g(p) +
           \left. f(p)\frac{\partial g}{\partial \vec v} \right |_p
$$
Il vettore $\vec v$ è allora individuato in maniera univoca dal modo in cui agisce
la derivata direzionale su tutte le funzioni differenziabili in un intorno di $p$.
Si definisce allora il \emph{vettore tangente} alla varietà $M$ nel punto $p$:
\begin{definition}{(Vettore tangente)}
      Sia $M$ una varietà differenziale e $p \in M$. Si dice \emph{vettore tangente}
      a $X$ nel punto $p$ un'applicazione $\vec V _p : C^\infty(p) \to \mathbb{R}$ tale che:
      \begin{itemize}
         \item $\vec V _p[\alpha f + \beta g] = \alpha \vec V _p[f] + \beta \vec V_p [g]
            \quad \forall f,g \in C^\infty(p) \mathrm{\:e\:} \forall \alpha,\beta \in \mathbb{R}$
         \item $\vec V _p [fg] = \vec V _p[f] g(p) + f(g)\vec V _p [g]
            \quad \forall f,g \in C^\infty(p)$
      \end{itemize}
\end{definition}
L'insieme dei vettori tangenti a una varietà $M$ nel punto $p$ è detto \textbf{spazio tangente}
alla varietà nel punto $p$ e si indica con $T_p(M)$.
Si può mostrare facilmente che $T_p(M)$ è uno spazio vettoriale.\\

Si definisce \textbf{spazio cotangente} $T_p^*(M)$ a una varietà $M$ nel punto
$p \in M$ il duale\footnote{Si ricorda che il duale $V^*$ di uno spazio vettoriale $V$ è l'insieme
delle applicazioni lineari $\phi : V \to \mathbb{R}$. Data una base $\{e_i\}$ di $V$,
la base canonica per $V^*$ è definita dalle proiezioni nella i-esima coordinata $dx^i$,
quindi $dx^i(e_j)=\delta^i_j$. Data una qualsiasi $\phi\in V^*$ applicazione lineare
su $V$, si ha $\phi = \sum_i a_i dx^i$ per $\{a_i\}\in\mathbb{R}$. }
dello spazio tangente $T_p(M)$.\\

Data una base di $\mathbb{R}^n$ si definiscono:
\begin{itemize}
   \item $ e_i = \left. \frac{\partial}{\partial x_i}\right |_p $
      le derivate parziali lungo la coordinata i-esima sono base di $T_p(M)$.
   \item $ e^i = \mathrm{d}x^i $ gli elementi di linea differenziali lungo la
      coordinata i-esima sono base di $T_p^*(M)$.
\end{itemize}

Si definisce \textbf{campo vettoriale} un'applicazione che a un punto $p$
della varietà $M$ associa un vettore tangente al punto. Si vuole richiedere anche
una dipendenza continua o liscia dal punto base $p$\footnote{Si veda \ref{sec:fibratotangente}
per chiarificare questa affermazione.}.
Lo spazio dei campi vettoriali sulla varietà $M$ si indica con $\mathcal{X}(M)$
$$ \vec V : M \to T_p(M) \: ,\: p \mapsto \vec V _p $$

Si definisce \textbf{campo covettoriale} un'applicazione che a un punto $p$
della varietà $M$ associa un vettore cotangente al punto. Si vuole richiedere anche
una dipendenza continua o liscia dal punto base $p$.
$$ \vec U : M \to T_p^*(M) \: ,\: p \mapsto \vec U _p $$

Siano $V = \sum_i v^i\frac{\partial}{\partial x^i}$ e $U = u_i dx^i$ campi
vettoriali/covettoriali e si consideri una generica trasformazione di coordinate
$x \mapsto y(x)$ $V$ e $U$ sono invarianti (indipendenti dalla scelta della base).
Si ha che:
$$
   dy^i = \sum_j \frac{\partial y^i}{\partial x^j} dx^j \mathrm{\quad e \quad}
   \frac{\partial}{\partial y^i} = \sum_j \frac{\partial x^j}{\partial y^i}
      \frac{\partial }{\partial x^j}
$$

In base a queste leggi di trasformazione in seguito a cambio di coordinate,
i vettori tangenti si dicono \textbf{covarianti} e i vettori cotangenti
si dicono \textbf{controvarianti}.\\
(o il contrario?)\\
%------------------------------------------------------------------------------%
due parole su funzioni tra varietà e Differenziali $F_*$\\

Per le coordinate si deve quindi avere:
$$
   v'^i = \sum_j \frac{\partial y^i}{\partial x^j} v^j \mathrm{\quad e \quad}
   u'_i = \sum_j \frac{\partial x^j}{\partial y^i} u_j
$$
Di conseguenza il prodotto interno $<U,V> $ è invariante:
$ <U,V> = \sum_i u_i v^i = \sum_j u'_j v^j $\\
(lo metto?)\\
%------------------------------------------------------------------------------%
%------------------------------------------------------------------------------%
\subsection{Varietà con bordo}
\textbf{Orientazione}\\
