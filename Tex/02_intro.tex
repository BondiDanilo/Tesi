\chapter*{Introduzione}

La ricerca della simmetria in Natura è un aspetto focale della Fisica. Nulla
affascina un fisico quanto una teoria ricca di simmetrie: queste sono sinonimo
di una più profonda struttura dell'universo e indagarle porta nuove consapelovezze.
In un mondo reale e non idealizzato alcune simmetrie vengono rotte. Quando ciò
accade, riuscire a comprenderne il meccanismo permette di svelare gli ingranaggi
della Natura.\\

Campi elettrici e campi magnetici presentano un elevato grado di simmetria: come
spiegano le equazioni di Maxwell e la Relatività Ristretta, il campo magnetico
altro non è che un campo elettrico in movimento o in evoluzione, e viceversa.
Una carica ferma nel nostro sistema di riferimento produce un campo elettrostatico:
se iniziassimo a correre vedremmo la carica allontanarsi da noi. Poichè
una corrente non è altro che un moto di carica elettrica, osserveremmo quindi un campo
magnetico prodotto dalla corrente stessa.
Al contrario, un filo percorso da corrente produce un campo magnetico. Se ora ci muovessimo
alla stessa velocità della carica che corre nel filo, vedremmo solamente il campo
elettrostatico prodotto dalla stessa carica, adesso ferma nel nostro sistema di riferimento.

I monopoli magnetici sono una rottura di questa simmetria: nonostante in natura
esistano singole cariche elettriche, la loro controparte magnetica non è mai
stata osservata.\\

Paul Dirac, con il suo celebre articolo del 1931\footnote{si veda \cite{dirac}}, pose le basi per
la moderna teoria del monopolo magnetico e accese l'interesse per la ricerca a
livello teorico sull'argomento.\\
Un primo modello elementare per descrivere il moto di un elettrone
nel campo prodotto da una ipotetica carica magnetica può essere formulato,
senza grossi problemi, con la meccanica newtoniana e l'elettromagnetismo classico.
Tuttavia, per costruire una formulazione hamiltoniana o lagrangiana del sistema,
occorre introdurre un potenziale elettromagnetico. Questo, però, non può essere
definito per cariche magnetiche isolate senza incorrere in singolarità e contraddizioni.
Poichè la Meccanica Quantistica è sviluppata sulla base di un formalismo
hamiltoniano, l'ipotesi del monopolo magnetico risulta scoraggiata.

Nonostante ciò, non si è mai spento l'interesse per la costruzione di una
teoria consistente dei monopoli. Come detto in precedenza, riuscire a spiegare
la rottura di simmetria tra elettricità e magnetismo può
portare a più profonde consapevolezze sul funzionamento della Natura.\\

A livello più applicativo, la soluzione del problema del monopolo magnetico si
lega con molti dei problemi irrisolti in Fisica, quali il problema del confinamento
in Cromodinamica Quantistica, il problema del decadimento del protone e la spiegazione
della quantizzazione della carica elettrica.\\

Lo scopo di questo elaborato è fornire una breve panoramica della teoria dei monopoli
magnetici, secondo i formalismi della Relatività Ristretta, della Meccanica Quantistica
Non Relativistica e delle teorie di campo classiche. Non verranno trattati aspetti
relativi a una teoria quantistica relativistica o a teorie quantistiche di campo.\\

Si inizierà presentando un modello elementare di monopolo magnetico utilizzando
la meccanica newtoniana di base, con il solo scopo di evidenziare le principali
problematiche della questione. Successivamente si adotterà il formalismo delle
teorie di gauge, partendo dal più semplice caso abeliano per arrivare alla
più efficace descrizione di una teoria non abeliana. Qui si vedranno il monopolo
di Wu-Yang e un caso particolare del modello di Georgi-Glashow.
Si concluderà accennando alla proposta di soluzione di 't Hooft e Polyakov, lasciando
aperta la questione riguardo la soluzione numerica del modello.
