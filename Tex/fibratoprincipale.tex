%------------------------------------------------------------------------------%s
\subsection{Fibrato Principale e Connessione}
\begin{definition}{(Fibrato principale)}:
   Un \emph{fibrato principale} è un fibrato $P$ sullo spazio base $M$ la cui
   fibra $F$ è un gruppo di Lie e coincide con il gruppo di struttura
   $G$\footnote{Si veda \cite{nakahara}, \cite{shnir}, \cite{eguchi}
   per una definizione più completa}.
\end{definition}
Si indica anche con $P(M,G)$ o solamente $P$ dando per scontato la fibra
e lo spazio base, per brevità.\\

Si vuole ora generalizzare il concetto di connessione\footnote{Il concetto di
connessione generalizza quello di derivata direzionale ai tensori}
(e di trasporto parallelo) estendendolo alla struttura del fibrato principale.
Tale generalizzazione ha il vantaggio di definire un trasporto parallelo
indipenente dalla metrica (a differenza della connessione di Levi-Civita). \\

Si sceglie di utilizzare due approcci per definire la connessione su un fibrato
principale. Il primo è di suddividere lo spazio tangente al fibrato in componente
"verticale" e "orizzontale" (l'affermazione verrà chiarita in seguito).
Il secondo è di definirla come una 1-forma a valori nell'algebra di Lie che
rispetta determinate proprietà. Le due definizioni si dimostrano essere
equivalenti\footnote{Si veda \cite{nakahara}}.\\
Il primo approccio, geometrico, ha il vantaggio di essere globale e non dipendere
dalle coordinate scelte, mentre il secondo è più efficace a livello pratico e
computazionale.\\

Sia $M$ lo spazio base di dimensione $n$, $G$ il gruppo di struttura e fibra e
$P$ il fibrato principale.
Sia $(U,\phi)$ una carta con coordinate $x_\mu = \phi_\mu(p) \:,p \in M$. \\

\subsubsection{Definizione geometrica }
Sia $u$ un punto del fibrato principale $P(M,G)$, sia $p = \pi(u) \in M$ e
$G_p$ la fibra in $p$\\

Si definisce \textbf{sottospazio verticale} $V_u(P)$ un sottospazio di $T_u(P)$
tangente alla fibra $G_p$ nel punto $u$. Gli elementi di $V_u(P)$ si costruiscono
nel modo seguente.\\
Sia $A \in \mathfrak{g}$
Richiamando la definizione \ref{def:inducedvectfield}, si definisce il vettore
$A^\# \in T_u(P)$ dall'azione su una generica funzione $f \in C\infty(P)$
$$
   (A^\#)_u [f] := \left. \frac{d}{dt} f(u \exp(tA)) \right|_{t=0}
$$

Analogamente a quanto visto per i gruppi di matrici, se $A$ è nell'algebra di Lie
$g = \exp(t_0 A)$ per $t_0$ fissato è un elemento del gruppo di Lie $G$.
Al variare di $t$, $g_t = \exp(tA)$ è una curva in $G$, quindi $u = \exp(tA) = ug_t$
è una curva lungo la fibra in $p = \pi(u)$ in quanto trasla $u$ per azione
destra di $G$. La funzione $f$ è quindi ristretta alla curva $ug_t$. \\
Allora  l'espressione a destra indica il vettore ($\frac{d}{dt}$) tangente alla curva
(si pensi al vettore velocità istantanea) $ug_t$ all'istante $t=0$, cioè nel punto
$u\exp(0A) = ue = u$.\\
Il significato dell'espressione a destra è esattamente definire
un vettore tangente alla fibra passante in $u$, nel punto $u$, a partire da
un elemento dell'algebra $A$, appunto $A^\#_u \in V_u(P)$.\\

Definendo il vettore $A^\#_u$ per ogni punto $u \in P$ si costruisce il campo
vettoriale $A^\#$, detto il \textbf{campo vettoriale fondamentale} generato
dall'algebra Lie ($A \in \mathfrak{g}$, infatti).\\
È immediatamente definito l'isomorfismo $\# : \mathfrak{g} \to V_u(P)$ da $A \mapsto A^\#$\\

Si definisce \textbf{sottospazio orizzontale} $H_u(P)$ il complemento di $V_u(P)$
in $T_u(P)$ ed è univocamente determinato.

\begin{definition}\label{def:prbundleconnection}
   Sia $P(M,G)$ un fibrato principale, $u \in P$. Una \textbf{connessione} su $P$ è una
   suddivisione univoca dello spazio tangente $T_u(P)$ nei sottospazi orizzontale $H_u(P)$
   e verticale $V_u(P)$ tale che:
   \begin{enumerate}
      \item $T_u(P) = H_u(P) \oplus V_u(P)$
      \item Ogni campo vettoriale $X$ su $P$ è separato nelle componenti $X = X_h + H_v$ \\
      dove $X_h \in H_u(P)$ e $X_v \in V_u(P)$
      \item spazi orizzontali sulla stessa fibra $H_u(P)$ e $H_{ug}(P)$ sono legati
      dalla mappa $(R_g)_*$ indotta dall'azione destra, ovvero
      $ H_{ug}(P) = (R_g)_* H_u(P) \quad \forall u \in P \: , \: g\in G$
   \end{enumerate}
\end{definition}

L'idea alla base di questo approccio è la seguente.
Per valutare la derivata direzionale di un campo su una varietà $M$ si vuole
confrontare i valori del campo in due punti $p$ e $p'$ di $M$, connessi da un cammino.
Si vuole "sollevare" i punti dallo spazio base $M$ in due punti $u$ e $u'$
nel fibrato $P$ e confrontare i due spazi tangenti $T_u(P)$ e $T_{u'}(P)$.\\
Il fibrato è definito localmente come prodotto $U \times G$
(dove U è un intorno di $p$ in $M$), quindi un vettore tangente $X_u \in T_u$
può essere scomposto in due componenti (proprietà 2 della defininizione):
una lungo la base (componente orizzontale) e una lungo la fibra (componente verticale).\\
Il trasporto verticale è univocamente determinato dall'azione del gruppo $G$
(proprietà 3 della definizione), mentre il trasporto orizzontale non è unico
(come non è unico il cammino che unisce i due punti $p$ e $p'$ nello spazio base).\\
È solamente quest'ultimo, però, di interesse in quanto mappa un cammino tra
i due punti $u$ e $u'$ in un cammino attraverso le fibre.
Il concetto alla base è quindi quello di scaricare il problema sul trasporto
attraverso le fibre del campo vettoriale di cui si vuole calcolare la derivata.\\

Manca però un metodo pratico di calcolo del trasporto orizzontale, che sarà dato
dall'approccio successivo.
%------------------------------------------------------------------------------%
\subsubsection{Definizione tramite 1-forma}
\begin{definition}\label{def:connection1form}
   Una 1-forma di connessione $\omega \in \mathfrak{g} \otimes T^*P$\footnote{
   Cfr. definizione \ref{def:liealgebraform}}
   è una proiezione dello spazio $T_u(P)$ su $V_u(P)$, che è isomorfo all'algebra
   di Lie $\mathfrak{g}$, tale che
   \begin{enumerate}
      \item $\omega(A^\#) = A   \quad A^\# \in T_u(P) \: ,  \: A \in \mathfrak{g}$
      \item $R_g^* \omega = Ad_{g^-1} \omega\footnote{Definizione \ref{def:adjrep}} $ \\
   \end{enumerate}
\end{definition}
La proprietà 2 corrisponde a
 $$ R_g^* \omega_u(X) = \omega_{ug}((R_g)_*X) = g^{-1} \omega_u(X) g .$$

Lo spazio orizzontale è definito come il kernel di $\omega$
$$
   H_u(P) := \{ X \in T_u(P) \: | \: \omega (X) = 0 \}
$$
La prima proprietà garantisce che ogni vettore verticale venga mandato nell'opportuno
elemento dell'algebra, mentre la seconda garantisce il corretto trasporto dello
spazio orizzontale lungo una fibra e fa si che la definizione appena data combaci
con la precedente, come dimostrato da:
\begin{proposition}
   Lo spazio orizzontale verifica
   $$
        (R_g)_* H_u P = H_{ug} P
   $$
\end{proposition}
\begin{proof}
   Sia $u \in P$ e si consideri lo spazio orizzontale $H_u(P) = ker\omega$.
   Sia $X \in H_u(P)$ e gli si applichi la mappa indotta dall'azione sinistra
   $(R_g)_*$, costruendo $(R_g)_*X \in T_{ug}(P)$. Si ha che:
   $$
      \omega( (R_g)_* X) = R_g^* \omega(X) = g^{-1}\omega(X)g \footnote{Per la
      proprietà 2.} = 0 \footnote{Poichè $\omega(X) = 0$}
   $$
   Quindi anche $(R_g)_*X$ sta nel kernel di $\omega$ e quindi $(R_g)_*X \in H_{ug}(P)$.\\
   Poichè $(R_g)_*$ è una mappa invertibile, qualsiasi vettore $Y \in H_{ug}(P)$ è
   esprimibile come $Y = (R_g)^* X$ per qualche $X \in H_u(P)$.\\
   Di conseguenza lo spazio $T_u(P)$ è separato in $H_u(P) \oplus V_u(P)$ da $\omega$,
   rendendo la definizione di 1-forma di connessione analoga a quella di connessione
   \ref{def:prbundleconnection}.
\end{proof}
La 1-forma di connessione $\omega$ così definita prende il nome di \emph{connessione
di Ehresmann}.\\

%\begin{definition}
%   Una 1-forma canonica (o forma di Maurer-Cartan) è una $\theta : T_g(G) \to T_e(G)$
%   $$ \theta : X \mapsto (L_{g^{-1}})_* X \quad X \in T_g(G)$$
%   $L$ è l'azione sinistra.
%\end{definition}
%\footnote{Vedi \cite{nakahara}, pag 234}
%
%Quello che fa è prendere un vettore in $T_g(G)$ e lo manda in $T_e(G)$ facendo il
%trasporto parallelo. $T_e(G)$ è isomorfo a $\mathfrak{g}$\\
%L'algebra sono i campi vettoriali left invarianti ricorda \\
%
%Può essere espressa come $\theta = V_\mu \otimes \theta^\mu$.\\
%Scegliamone una particolare $\theta = g^{-1}dg$\\
%
%La connessione è definita come proiezione sul verticale e mappata nel tangente a e
%che è isomorfo all'algebra
%$\omega = \theta \circ \pi$

Si osserva che la 1-forma di connessione è definita \emph{globalmente} su P, quindi
assegnare una forma di connessione fornisce un metodo per il trasporto parallelo,
unico per tutto il fibrato $P$.\\

Si vuole ora trasportare la forma di connessione $\omega$, globale, dagli spazi
tangenti a $P$ agli spazi tangenti alla varietà $M$.\\

Sia $\{U_i\}$ un ricoprimento aperto di $M$ e siano $s_i$ sezioni locali definite
su ciascun $U_i$. Si può allora trasportare $\omega$ definendo su $U_i$ la forma
$$
   A_i := s_i^* \omega \in \mathfrak{g} \otimes \Omega^1(U_i)
$$
Viceversa, sia $A_i$ una 1-forma a valori in una algebra di Lie assegnata su su $U_i$.
Si può sempre costruire una 1-forma di connessione $\omega$ che trasportata tramite
$s_i^*$ sia esattamente $A_i$.
\begin{theorem}
   Data una 1-forma $A_i$ a valori in $\mathfrak{g}$ su $U_i$ e una sezione locale
   $s_i : U_i \to \pi^{-1}(U_i)$, esiste una 1-forma di connessione $\omega$
   tale che $A_i = s_i \omega$, che ristretta a $\pi^{-1}(U_i)$ assume la forma
   $$
      \omega|_{\pi^{-1}(U_i)} = \omega_i := g_i^{-1} \pi^* A_i g_i + g_i^{-1} d_P g_i
   $$
   dove $d_P$ è la derivata esterna su $P$ e $g_i$ la canonica local trivialization
   data da $\phi_i^{-1}(u) = (p,g_i)$ per $u = s_i(p)g_i$
\end{theorem}
Si veda \cite{nakahara} per la dimostrazione.\\

Si sottolinea l'importanza di quest'ultimo teorema in quanto la conoscenza di un'unica
forma $A_i$ su un aperto $U_i$ di $M$, unitamente alla condizione di unicità della
1-forma di connessione $\omega$, vincola tutte le altre forme defnite sugli aperti
$U_{j\neq i}$ del ricoprimento. La condizione da rispettare è la seguente\footnote{
Si veda \cite{nakahara} per la derivazione completa.
}.

Siano $U_i$ e $U_j$ due aperti del ricoprimento aperto di $M$ a intersezione non vuota.
Affinchè la forma di connessione $\omega$ sia definita \textbf{univocamente} su tutto $P$,
si deve avere sull'intersezione $U_i \cap U_j$ che $\omega_i = \omega_j$.\\
Affinchè ciò sia verificato, le forme locali $\omega_i,\omega_j$ devono verificare
la seguente condizione di compatibilità. Siano $\Phi_{ij}$ le funzioni di transizione
definite in \ref{eq:transfunctions}
$$
   s^*_j \omega (X) = \Phi_{ij}^{-1} \omega( (s_i)_*X ) \Phi_{ij}
                         + \Phi_{ij}^{-1}d\Phi_{ij}(X)
$$
Che deve essere valida per ogni campo $X \in T_p(M)$. Quindi si traduce nella
condizione per le forme $A_i$:
\begin{equation}\label{eq:condcompatibility}
   A_j = \Phi_{ij}^{-1} A_i \Phi_{ij} + \Phi_{ij}^{-1}d\Phi_{ij}
\end{equation}

Se, al contrario, sono dati $\{U_i\}$ ricoprimento aperto di $M$, $\{s_i\}$
sezioni, e $\{A_i\}$ forme che rispettano \ref{eq:condcompatibility}, si può
ricostruire la forma di connessione $\omega$\\
Si sottolinea che se il fibrato $P$ non è banale, non è possibile definire in
maniera univoca una sezione globale, quindi la forma $A_i$ esiste solo localmente.
La forza di questa costruzione è che invece la forma di connessione $\omega$
è definita globalmente su $P$.
%------------------------------------------------------------------------------%
Si vuole ora definire la derivata covariante di una
1-forma di connessione $\omega$.
In maniera intuitiva, la si vuole definire in maniera tale che trasformi allo
allo stesso modo di $\omega$ sotto l'azione del gruppo $G$. Così facendo si
può identificare la derivata covariante $D$ con la derivata esterna $d$.\\
Se la 1-forma $\omega$ trasforma come un vettore $\omega \mapsto g\omega$ per
$g \in G$, allora anche la derivata covariante trasforma $ D\omega \mapsto
g D\omega $. Si definisce allora:
   $$ D\omega := d\omega + A \wedge \omega $$
dove $A$ è la 1-forma definita in precedenza, che ripetta la legge di
trasformazione \ref{eq:condcompatibility}. In maniera analoga si può definire la
derivata covariante di 1-forme di connessione che trasformano come tensori o sono
invarianti.
\begin{definition}
   Sia $\omega$ la 1-forma di connessione di un fibrato principale $P$. Si definisce
   la 2-forma di textbf{curvatura} $\Omega$ come la derivata covariante di $\omega$
   $$ \Omega := D\omega $$
\end{definition}
Poichè $A_i$ è il pullback di $\omega$ su un aperto di $U_i \subset M$
tramite una sezione $s_i$, si definisce una 2-forma su $U_i$ tramite
\begin{equation} F_i := s_i^* \Omega \end{equation}
che rispetta la condizione di compatibilità sull'intersezione di due aperti
$U_i \cap U_j$
\begin{equation} F_j = \Phi_{ij}^{-1}F_i\Phi_{ij} \end{equation}

La 2-forma di curvatura fornisce n metodo di classificazione topopologica
dei fibrati.
