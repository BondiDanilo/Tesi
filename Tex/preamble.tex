
\documentclass[10pt,a4paper]{report}
\usepackage{graphicx}
\usepackage[utf8]{inputenc}
\usepackage[italian]{babel}
\usepackage[T1]{fontenc}

\usepackage{amsthm}             %stili teoremi
\usepackage{amssymb,amsmath}
\usepackage[colorlinks = true,
          linkcolor = blue,
          urlcolor  = blue,
          citecolor = blue,
          anchorcolor = blue]{hyperref}
\usepackage{bookmark}
\usepackage{wrapfig}

\usepackage{geometry}
\geometry{
	top=25mm,
	bottom=25mm,
	left=30mm,
	right=25mm
}
\usepackage[ampersand]{easylist}
\usepackage[nottoc]{tocbibind} %bibliografia
%------------------------------------------------------------------------------%

% Stile+numerazione teoremi, definizinoni, etc...
\theoremstyle{plain}
\newtheorem{theorem}{Teorema}[equation]

\theoremstyle{definition}
\newtheorem{definition}{Definizione}[equation]

\theoremstyle{plain}
\newtheorem{proposition}{Proposizione}[equation]

\theoremstyle{plain}
\newtheorem{lemma}{Lemma}[equation]

\theoremstyle{remark}
\newtheorem{example}{Esempio}[equation]

\theoremstyle{definition}
\newtheorem{axiom}{}[section]

\renewcommand\qedsymbol{$\blacksquare$} % quadrato della dimostrazione
%------------------------------------------------------------------------------%
\renewcommand{\vec}[1]{\mathbf{#1}} % simbolo di vettore = grassetto
\newcommand{\chrsym}{\genfrac{\{}{\}}{0pt}{}} % Simboli Christoffel
%------------------------------------------------------------------------------%
\graphicspath{{/home/dan/Desktop/UNI/TESI/Images/}}	%Default path for graphics
%------------------------------------------------------------------------------%
% Remove default parindent
\newlength\tindent
\setlength{\tindent}{\parindent}
\setlength{\parindent}{0pt}
\renewcommand{\indent}{\hspace*{\tindent}}

\newcommand{\tab}[1][1cm]{\hspace*{#1}} % ridefinisce la tabulazione

%------------------------------------------------------------------------------%
% Metadati
\hypersetup{
	pdftitle={Tesi},%
	pdfauthor={Danilo Bondì},%
	pdfsubject={Aspetti classici e quantistici dei monopoli magnetici in teorie di gauge},%
	pdfkeywords={},%
	colorlinks=true,%
	linkcolor=blue,%
	linktocpage=true,%
	pageanchor=true
}
