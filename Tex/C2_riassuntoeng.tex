
\documentclass[10pt,a4paper]{report}
\usepackage{graphicx}
\usepackage[utf8]{inputenc}
\usepackage[italian]{babel}
\usepackage[T1]{fontenc}

\usepackage{amsthm}             %stili teoremi
\usepackage{mathtools}

\usepackage{amssymb,amsmath}
\usepackage[colorlinks = true,
          linkcolor = blue,
          urlcolor  = blue,
          citecolor = blue,
          anchorcolor = blue]{hyperref}
\usepackage{bookmark}
\usepackage{wrapfig}

\usepackage{geometry}
\geometry{
	top=25mm,
	bottom=25mm,
	left=30mm,
	right=25mm
}
\usepackage[ampersand]{easylist}
\usepackage[nottoc]{tocbibind} %bibliografia
%------------------------------------------------------------------------------%

% Stile + numerazione teoremi, definizinoni, etc...
\theoremstyle{plain}
\newtheorem{theorem}{Teorema}[section]

\theoremstyle{definition}
\newtheorem{definition}{Definizione}[section]

\theoremstyle{plain}
\newtheorem{proposition}{Proposizione}[section]

\theoremstyle{plain}
\newtheorem{lemma}{Lemma}[section]

\theoremstyle{remark}
\newtheorem{example}{Esempio}[section]

\theoremstyle{definition}
\newtheorem{axiom}{}[section]

\renewcommand\qedsymbol{$\blacksquare$} % quadrato della dimostrazione
%------------------------------------------------------------------------------%
\renewcommand{\vec}[1]{\mathbf{#1}} % simbolo di vettore = grassetto
\newcommand{\chrsym}{\genfrac{\{}{\}}{0pt}{}} % Simboli Christoffel
%------------------------------------------------------------------------------%
\graphicspath{{/home/dan/Desktop/UNI/TESI/Images/}}	%Default path for graphics
%------------------------------------------------------------------------------%
% Remove default parindent
\newlength\tindent
\setlength{\tindent}{\parindent}
\setlength{\parindent}{0pt}
\renewcommand{\indent}{\hspace*{\tindent}}

\newcommand{\tab}[1][1cm]{\hspace*{#1}} % ridefinisce la tabulazione
\newcommand{\ttab}[1][2cm]{\hspace*{#1}} % ridefinisce la tabulazione
\newcommand{\dd}{\mathrm{d}} % dx negli integrali
\newcommand{\tc}{\mathrm{\: t.c. \:}} % tale che
\newcommand{\R}{\mathbb{R}} % insieme dei reali
\newcommand{\C}{\mathbb{C}} % insieme dei complessi
\newcommand{\Z}{\mathbb{Z}} % insieme degli interi
\newcommand{\Ll}{\mathcal{L}} % lagrangiana
\newcommand\Chapter[2]{
  \chapter*[#1: {\itshape#2}]{#1\\[2ex]\Large\itshape#2}
} %sottotitolo al capitolo

%------------------------------------------------------------------------------%
% Metadati
\hypersetup{
	pdftitle={Tesi},%
	pdfauthor={Danilo Bondì},%
	pdfsubject={Aspetti classici e quantistici dei monopoli magnetici in teorie di gauge},%
	pdfkeywords={},%
	colorlinks=true,%
	linkcolor=blue,%
	linktocpage=true,%
	pageanchor=true
}

\begin{document}

{\huge \textbf{Classical and quantum analysis of magnetic monopoles in gauge theories}}\\
{\Large \emph{Bondì Danilo}}\\


No magnetic monopole has ever been observed in nature. Classical electrodynamics,
however, does not provide a reason for their non-existance. Since Maxwell equations
merely formalize the experimental observations on electric and magnetic phenomena,
no \emph{a priori} rejection of the monopole is made.
Despite the lack of experimental results, the interest in a consistent
theory of magnetic monopoles has not vanished throughout the past century, since
Dirac first published his original paper in 1931.\\

Our first step in introducing a theory of the magnetic monopole will be a naive
construction of an elementary classical model, aimed at writing the equations of motion
of an electron in a Coulomb-like magnetic field. This is only meant to stress the
central issues of the subject.
We begin by assuming the existence of a magnetic charge, analogous to the electric
charge. Consequently modify Maxwell's equations, including the non-zero
divergence of the magnetic field $\vec B$, which has to be equal to the local
magnetic charge density, namely $\rho_g$.\\
Here arises the first contraddiction. If one defines the electromagnetic potential
$\vec A$, $\vec B$ being the curl of $\vec A$, it is impossible to have both
$\vec B = \nabla \times \vec A$ and $\nabla \cdot \vec B = 4\pi \rho_g$ at the same
time. This is due to the fact that $\vec A$ is not defined everywhere in space, but
it will always have a string of singularities.\\

A non-contradictory monopole theory is incompatible with a global vector potential.
We are required to describe our model using only local potentials, which must
agree on their overlap region via an appropriate trasformation.
The frame which fits this picture the most is that of gauge theories.\\

Since classical electrodynamics is a $U(1)$ gauge theory, we define local potentials
on two open sets, requiring that a $U(1)$ gauge transformation connects them within the
overlapping region. This way, it is possibile to define the correct monopole field
and to remove the previous contradiction. We also assing a topological meaning
to the magnetic charge, via the Chern characteristic classes of the considered manifold.
Due the condition of the fields transformation to be single-valued,
a relationship between magnetic and electric charges is obtained.
One of the main issues with this Abelian model arises when we try to
extend our theory to a quantum field theory.\\

The next step is to generalize our theory to a non-Abelian gauge group, having
$U(1)$ as a subgroup, and which will reduce to our previous abelian theory
under normal conditions.
This spontaneous symmetry-breaking process preserves all the previous predictions,
giving a solution to the problems of the afore-mentioned model.\\
Theories of this kind are named \emph{Yang-Mills theories}.\\

The first and simplest case is to consider $SU(2)$ as the gauge group. We will
analyze two models of this type.

Onw proposed by Wu and Yang in 1969 is capable of solving all the problems with
the abelian model, but gives badly-defined energy configurations, leading
to divergencies.\\
The second one is a special case of the more general model proposed by Georgi and
Glashow in 1974. Here, the gauge potential is coupled with another complex field,
the Higgs field. This immediatly breaks the symmetry of $SU(2)$ down to $U(1)$,
and solves the infinite-energies problem. The last achievement is a definition
of the magnetic charge directly derived from the conservation of the
field-strength tensor, therefore from Maxwell equations themselves.\\

Unfortunately, the model has no analytic solution in general: numerical
solutions to the problem need to be found.
We conclude our dissertation by briefly mentioning a solution proposed
by 't Hooft and Polyakov, a starting point for numerical solutions of the Georgi-Glashow
model.
\end{document}
