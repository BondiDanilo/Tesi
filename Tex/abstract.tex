\chapter*{Introduzione}
\begin{equation*}
   \begin{aligned}
      \nabla \times \vec E & = -\frac{1}{c}\frac{\partial \vec B}{\partial t}&
      \nabla \times \vec B & = \frac{1}{c} \left(
         4\pi \, \vec J + \frac{\partial \vec E}{\partial t} \right) \\
      \nabla \cdot \vec E & = 4\pi \, \rho &
      &\boxed{\nabla \cdot \vec B = 0} \\
   \end{aligned}
\end{equation*}\\

La quarta delle equazioni di Maxwell asserisce che "non esistono sorgenti del
campo magnetico", ossia non esistono monopoli magnetici. Essendo le equazioni di
Maxwell di natura fenomenologica, non esite tutt'ora una teoria che ne escluda
totalmente l'esistenza. \\
La relatività speciale mostra che vi è una quasi totale simmetria tra campo elettrico
e campo magnetico. Il campo magnetico prodotto da un filo percorso da corrente
diventa un campo puramente elettrostatico se osservato nel sistema di riferimento
solidale con la carica in moto, e allo stesso modo il campo elettrostatico prodotto
da una carica ferma diventa campo magnetico se osservato in un sistema in moto rispetto
alla carica.\\
Questa simmetria è però rotta dal fatto che non sono mai stati tuttora osservati
monopoli magnetici, mentre i monopoli elettrici sono ben noti fin dalle prime
formulazioni dell'elettromagnetismo.\\
L'interesse nello studio dei monopoli magnetici è innanzitutto ripristinare
questa rottura di simmetria, processo molto comune in Fisica (ad esempio per le
Grandi Teorie Unificative, \emph{GUT}), ma presenta relazioni anche con altri
grandi problemi tutt'ora irrisolti della Fisica Teorica, ad esempio il problema
del Confinamento in Cromodinamica Quantistica, il problema del decadimento del
protone e la quantizzazione della carica elettrica.\\

Lo scopo di questo elaborato è dare una breve panoramica del monopolo magnetico,
secondo i formalismi della relatività ristretta, la meccanica quantistica non
relativistica e le teorie di campo classiche. Non verranno trattati in alcun modo
aspetti di teorie di campo quantistiche. \\
Saranno affrontati sostanzialmente tre tipi di monopoli: il monopolo di Dirac,
per introdurre il problema a livello classico, il monopolo di Wu-Yang in una
teoria di gauge abeliana, e il monopolo di 't-Hooft-Polyakov in teoria di gauge
non abeliana.
