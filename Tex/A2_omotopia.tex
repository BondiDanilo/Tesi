\section{Omotopia e Classi di omotopia}
Nel seguito, se non diversamente specificato, siano $X,Y$ spazi topologici con
topologia assegnata $T_X , T_Y$ rispettivamente.
Le dimostrazioni dei teoremi vengono omesse per brevità di trattazione. Si veda
\cite{sernesi}, \cite{fulton}, \cite{nakahara}.

\begin{definition}(\emph{Omotopia tra funzioni continue}):
   Siano $f,g : X \to Y$ funzioni continue. Si definisce \textbf{omotopia} tra
   $f$ e $g$ un'applicazione continua $H : X \times [0,1] \to Y$ tale che
   $$ H(x,0) = f(x) \quad \mathrm{e} \quad H(x,1) = g(x) \quad \forall x \in X$$
   Se esiste un'omotopia tra le funzioni $f$ e $g$, si dicono \emph{omotope} $f \sim g$.
   \label{def:omotopia}
\end{definition}

Intuitivamente, due funzioni sono omotope se le loro immagini $f(X),g(X)$ possono essere
deformate con continuità (senza "strappare" l'insieme) una nell'altra.

\begin{definition}(\emph{Equivalenza omotopica}):
   Due spazi topologici $X,Y$ si dicono \emph{omotopicamente equivalenti} o che hanno
   \emph{lo stesso tipo di omotopia} se esistono due applicazioni $f : X \to Y$ e
   $g : Y \to X$ tali che le composizioni $f \circ g : Y \to Y$ e
   $g \circ f : X \to X$:
      $$  f \circ g \sim \mathrm{id}_Y \quad , \quad g \circ f \sim \mathrm{id}_X $$
   Le due applicazioni così definite si dicono \emph{equivalenze omotopiche}.
\end{definition}

\begin{lemma}
   La relazione di omotopia è una relazione di equivalenza tra funzioni.
   L'equivalenza omotopica è una relazione di equivalenza tra spazi topologici.
\end{lemma}

\subsubsection{Esempio:}
\begin{example}
   Siano $X = \R^n\setminus\{0\}$ e $Y = S^{n-1}$. I due spazi sono
   omotopicamente equivalenti. Le funzioni di equivalenza omotopica sono
   $f : x \in \R^n\setminus\{0\} \mapsto \frac{x}{|x|}$ e
   $g : S^{n-1} \to \R^n\setminus\{0\}$ l'inclusione
   $y \in S^{n-1} \mapsto y \in \R^n\setminus\{0\}$. \\
   Un'omotopia è ad esempio $H(x,t) = x e^{(t-1)log|x|}$. È di immediata verifica
   che si tratta dell'omotopia cercata.
\end{example}

\subsection{Primo gruppo di omotopia}
\begin{definition}
   Si definisce \textbf{cammino} o \emph{arco} in $X$ una funzione continua
   $\sigma : [0,1] \to X$. Un cammino si dice \textbf{laccio} o \emph{circuito}
   se è chiuso, ossia $\sigma(0) = \sigma(1)$.\\
   Alternativamente, un \textbf{laccio} è una funzione continua del cerchio in $X$
   $\sigma : S^1 \to X$.
\end{definition}
L'idea è quella di avvolgere, deformandolo, il cerchio attorno allo spazio $X$,
senza però strapparlo (continuità della funzione).
Sia $x_0 \in X$. Un laccio $\sigma$ tale che $\sigma(0) = x_0$ si dice
\emph{laccio di base $x_0$}.\\

Nel definire l'omotopia tra due cammini, si vuole anche richiedere che il
punto base venga lasciato inalterato dalla funzione dell'omotopia.

\begin{definition}(\emph{Omotopia tra cammini}):
   Siano $\sigma,\tau : [0,1] \to X$ cammini con lo \emph{stesso} punto iniziale
   $\sigma(0) = \tau(0)$ e punto finale $\sigma(1) = \tau(1)$.
   Si definisce \textbf{omotopia} tra $\sigma$ e $\tau$ un'applicazione continua
   $H : [0,1] \times [0,1] \to X$ tale che
      $$ H(t,0) = \sigma(t) \quad \mathrm{e} \quad H(t,1) = \tau(t)
         \quad \forall t \in [0,1]$$
      $$ H(0,s) = \sigma(0) = \tau(0) \quad \mathrm{e} \quad
         H(1,s) = \sigma(1) = \tau(1) \quad \forall s \in [0,1] $$
\end{definition}

\begin{definition}(\emph{Gruppo fondamentale}):
   Si definisce il \textbf{gruppo fondamentale} o \emph{primo gruppo di omotopia}
   dello spazio $X$, denotato con $\pi_1(X,x_0)$, l'insieme delle classi di
   equivalenza dei lacci di base $x_0$, rispetto alla relazione di omotopia.
\end{definition}

Lacci che possono essere deformati con continuità l'uno nell'altro appartengono
alla stessa classe. Questa proprietà dipende strettamente dalla geometria dello
spazio (ad esempio numero di buchi) e il gruppo fondamentale è un modo di classificare
queste proprietà topologiche.\\
Un gruppo fondamentale composto da una sola classe di equivalenza è detto \emph{banale},
in quanto ogni laccio è deformabile in qualsiasi altro e in particolare al laccio
costante nel punto base.\\
Uno spazio con gruppo fondamentale \emph{isomorfo} al gruppo banale si dice
\textbf{contraibile}. \\

\begin{theorem}
   Siano $X,Y$ spazi topologici con lo stesso tipo di omotopia, con equivalenza
   omotopica data da $\phi : X \to Y$, essa induce un isomorfismo tra i gruppi
   fondamentali.
   $$
      \pi_1(X,x_0) \cong \pi_1 (Y, \phi(x_0)) \quad \forall x_0 \in X
   $$
   Con isomorfismo dato da $\phi_* : \pi_1 (X,x_0) \to \pi_1 (Y,\phi(x_0))$
   $$
      \phi_*([\sigma]) = [\phi \circ \sigma]
   $$
\end{theorem}

Dal teorema appena enunciato è chiara l'importanza di classificare gli spazi in
base al gruppo fondamentale. Esso infatti è un \textbf{invariante topologico},
ossia spazi che sono omeomorfi hanno lo stesso gruppo fondamentale (isomorfi).\\

\subsubsection{Numero di avvolgimento}

Dato un punto nel piano $(x,y) \in R^2$ l'angolo $\theta$ che individua il punto
in coordinate polari è dato da $\theta = \arctan(y/x)$, per $x \neq 0$. Si consideri
un cammino $\sigma$ nel piano, che sia differenziabile a tratti, si definisce
la funzione angolo $\theta(t) = \arctan (y(t)/x(t))$. Differenziando ambo i lati si
definisce la forma differenziale:
$$
  \dd \theta = \frac{1}{x^2+y^2}(x \dd y - y \dd x)
$$
\begin{definition}(\emph{Numero di avvolgimento}):
  Sia $\sigma$ un laccio in $\R^2$. Si definisce \textbf{numero di avvolgimento} ($W$)
  del laccio il rapporto tra la variazione totale di angolo lungo $\sigma$ e
  l'angolo totale $2\pi$:
  $$
     W := \int_\sigma \frac{\dd \theta}{2\pi}
        = \frac{1}{2\pi} \int_\sigma \frac{x \dd y - y \dd x}{x^2 + y^2}
  $$
\end{definition}
Si può estendere la definizione per una generica varietà differenziale, si veda
\cite{fulton}.\\


%------------------------------------------------------------------------------%
\subsubsection{Esempi di gruppo fondamentale}
\begin{example}
   (\emph{Gruppo fondamentale di $S^1$}): Cammini con numero di avvolgimento differente
   non sono omotopi (si pensi di avvolgere un elastico attorno al dito una, due, tre
   volte e provare a passare da una configurazione all'altra senza sfilare l'elastico
   dal dito), quindi si può stabilire un isomorfismo tra il gruppo fondamentale
   del cerchio e il gruppo degli interi $\Z$ associando a ogni cammino il suo numero
   di avvolgimento. Quindi
   $$
      \pi_1 (S^1) \cong \Z
   $$
\end{example}

\begin{example}
   (\emph{Gruppo fondamentale di $S^2$}): Il gruppo fondamentale della sfera è banale.
   Intuitivamente si pensi di avvolgere un elastico intorno a una palla, inchiodando
   un punto: si può sempre sciogliere l'elastico.
   $$
      \pi_1 (S^2) \cong \{ 0\}
   $$
\end{example}
%------------------------------------------------------------------------------%
\subsection{Gruppi di omotopia superiori}
Si può pensare a un laccio in $X$ come una mappa tra $S^1$ e $X$\footnote{
Siccome l'intervallo $[0,1]$ con gli estremi identificati
(dal fatto che un laccio è un cammino chiuso) è identificato con il cerchio}.
Quindi la relazione di omotopia tra cammini può essere vista come una relazione
di equivalenza tra le funzioni continue $\sigma : S^1 \to X$, secondo la definizione
di omotopia tra funzioni continue \ref{def:omotopia} e ridefinire
il gruppo fondamentale come l'insieme di queste classi di omotopia\\

Si può allora generalizzare il procedimento alle mappe continue da $S^n \to X$ e
definire i gruppi superiori di omotopia.

\begin{definition}(\emph{$n$-esimo gruppo di omotopia}):
   Sia $x_0 \in X$. Si definisce l'\textbf{$n$-esimo gruppo di omotopia}
   dello spazio $X$ l'insieme delle classi di equivalenza delle funzioni continue
   $\sigma : S^n \to X$.
\end{definition}

\subsubsection{Gruppi di omotopia delle sfere}
\begin{equation}
   \begin{aligned}
      \pi_1(S^1) & \cong \Z  &
      \pi_2(S^1) & \cong \{ 0 \}   &
      \pi_3(S^1) & \cong \{ 0 \}   \\
      \pi_1(S^2) & \cong \{ 0 \}   &
      \pi_2(S^2) & \cong \Z  &
      \pi_3(S^2) & \cong \Z
   \end{aligned}
\end{equation}
