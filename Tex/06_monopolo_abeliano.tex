\chapter{Monopoli in teorie di gauge abeliane}
Nella sezione \ref{sec:gaugestring} si è visto come non è possibile definire globalmente
(su tutto lo spazio $\R^3 \setminus \{0\}$) un potenziale di monopolo regolare,
incorrendo nella stringa di singolarità. Il primo approccio è stato di rendere
equivalenti tutte le possibili configurazioni di direzione della stringa, tramite
trasformazione di gauge di tipo $U(1)$. Si è arrivato a definire due potenziali
(\ref{eq:localdiracpotential}) $\vec A ^\pm$ che hanno la stringa situata rispettivamente
lungo l'asse $z$ negativo/positivo, quindi regolari su asse $z$ positivo/negativo
ripettivamente.\\
L'approccio qui seguito, proposto da Wu e Yang (1968) \cite{wuyang}, è di
rinunciare a una definizione globale del potenziale di Dirac
in favore di una descrizione tramite due potenziali definiti localmente su due aperti
$U^\pm$, che concordano nella regione di intersezione tramite una trasformazione
di gauge.\\

Innanzitutto, Poichè $\R^3 \setminus \{0\}$ è equivalente omotopicamente
a $S^2$, si vuole studiare il problema sulla sfera. Siano allora $U^\pm$ l'emisfero
nord e sud della sfera
\begin{equation}\label{eq:U+U-}
   \begin{aligned}
      U^+ &= \{(x,y,z) \in S^2 \: | \: z > 0 \}
          = \{(r,\theta,\varphi ) \in S^2 \: | \: 0 \leq \theta \leq \pi/2 \} \\
      U^- &= \{(x,y,z) \in S^2 \: | \: z < 0 \}
          = \{(r,\theta,\varphi ) \in S^2 \: | \: \pi/2 \leq \theta \leq \pi \}
   \end{aligned}
\end{equation}
La regione di intersezione è l'equatore
\begin{equation*}
   U^0 = U^+ \cap U^- = \{(x,y,z) \in S^2 \: | \: z = 0 \}
       = \{(r,\theta,\varphi ) \in S^2 \: | \: \theta = \pi/2 \}
\end{equation*}
Si possono allora definire i potenziali $\vec A^\pm$ su $U^\pm$, che nella regione
$U^0$ sono legati, come visto nella sezione \ref{sec:gaugestring}, da
$$
   \vec A ^+ - \vec A^- = \nabla \lambda = \nabla (2g\varphi ) = \frac{2g}{r\sin\theta} \, \vec u _\varphi
$$
Si noti che $\nabla \lambda$ è singolare in $\theta = 0,\pi$, ma nella regione $U^0$
si ha $\theta = \pi/2$.\\
Si calcola allora il flusso totale come:
\begin{equation*}
   \begin{aligned}
      \int_{S^2} \dd \sigma \: \nabla \times \vec A & =
         \int_{U^+} \dd \sigma \: \nabla \times \vec A^+ +
         \int_{U^-} \dd \sigma \: \nabla \times \vec A^-
        = \oint_{U^0} \dd \vec l \: \vec A^+ -
          \oint_{U^0} \dd \vec l \: \vec A^- \\
      & = \oint_{U^0} \dd \vec l \: \nabla \, (2g\varphi )
        = \int_0^{2\pi} \, 2g \, \varphi  \, \dd \varphi
        = 4g\pi
   \end{aligned}
\end{equation*}
In accordo con \ref{eq:B4gpi}.\\

Il formalismo più naturale per descrivere il potenziale di monopolo in una teoria
di gauge è quello di un fibrato principale con spazio base $M = \R^3
\setminus \{0\}$ e gruppo di struttura $G$ il gruppo di gauge.\\
Si faccia riferimento all'appendice per la teoria sui fibrati \ref{sec:fibrati}.

Se il fibrato ammette una sezione globale, può essere coperto con un'unica carta.
Il fibrato è allora banale e ha struttura globale di prodotto diretto tra la
varietà $M$ e la fibra $G$, e la classe di Chern caratteristica è $c_0 = 1$.
La complessità della struttura del fibrato di solito è legata alla contraibilità
dello spazio base.\\

Si consideri una teoria di gauge con gruppo $G = U(1)$, come l'elettrodinamica classica.\\

Se si prende come base $M = \R^3$, che è uno spazio contraibile, può essere
ricoperto da un'unica carta e il potenziale di gauge $A$ è definito globalmente su $M$,
continuo e differenziabile, e il tensore elettromagnetico $F$ è una 2-forma
sia chiusa $DF = 0$ che esatta $F = \dd A$.\\

Se invece lo spazio base è $M = \R^3 \setminus \{0\}$, che non è uno spazio
contraibile, la situazione cambia drasticamente. $\R^3 \setminus \{0\}$ è
equivalente omotopicamente alla sfera $S^2$.
In riferimento all'esempio \ref{ex:monopolechern}, è possibile costruire due fibrati
principali: uno con classe di Chern $c_0 = 1$ e uno con $c_1 = F/2\pi$. Quest'ultimo,
detto appunto \emph{fibrato di monopolo} è quello che descrive correttamente il
monopolo di Wu-Yang.
%------------------------------------------------------------------------------%
%------------------------------------------------------------------------------%
\section{Fibrato di Monopolo di Wu–Yang}\label{sec:wuyangmonopole}
Si consideri un fibrato principale con $M = S^2$ e $G = U(1)$.
Il gruppo di gauge $U(1) = \{e^{i\alpha} \}$ è parametrizzato da un
parametro $\alpha$, che è una coordinata ciclica lungo la fibra, ed è identificato
quindi con la sfera $S^1$.\\

Siano $U^\pm$ emisferi nord e sud e $U^0$ l'equatore come sopra.
Si hanno allora le due carte locali per il fibrato
\begin{equation}
   \begin{aligned}
      U^+ \times S^1 ,&  \mathrm{\:con\: coordinate\:} \{ \theta, \varphi ,\alpha^+ \}\\
      U^- \times S^1 ,&  \mathrm{\:con\: coordinate\:} \{ \theta, \varphi ,\alpha^- \}
   \end{aligned}
\end{equation}

Una parametrizzazione per la sfera di raggio unitario è data dalle usuali
coordinate polari $\varphi  \in[0,2\pi)$ e $\theta \in [0,\pi)$,
quindi si può definire una base per lo spazio tangente $T_pS^2$ come
$\{ \partial / \partial \theta \: , \: \partial / \partial \varphi  \}$
e per lo spazio cotangente $T^*_pS^2$ come $\{ \dd \theta \: , \: \dd \varphi  \}$.
La derivata esterna è data da
$$
   \dd  = \dd \theta \frac{\partial}{\partial \theta}
        + \dd \varphi    \frac{\partial}{\partial \varphi  }
$$

La 1-forma di connesione $\omega$\footnote{Si veda \ref{}}
definita globalmente sul fibrato, portata su $U^\pm$ tramite pullback,
dà il potenziale $A$

\begin{equation}\label{eq:hopfpotential}
   A = a(\theta)\dd \varphi =  \begin{cases}
      A^+ =  \frac{n}{2}(1 - \cos\theta ) \dd \varphi  & \mathrm{su \:} U^+ \\
      A^- = -\frac{n}{2}(1 + \cos\theta ) \dd \varphi  & \mathrm{su \:} U^-
   \end{cases}
\end{equation}

e la 2-forma di curvatura $\Omega$\footnote{Si veda \ref{}} viene portata
sul tensore elettromagnetico $F$
$$
   F = \dd A = \dd a(\theta)\wedge \dd \varphi= \frac{n}{2} \: \sin\theta \: \dd \theta \wedge \dd \varphi
$$
%Osservando che i versori delle coordinate sferiche sono dati da
%$\vec u _\varphi  = r\sin\theta \dd \varphi $ e
%$ \vec u _\theta = r\dd \theta$\footnote{
%   Nel senso che l'azione del vettore $\vec u _\varphi \in \R^3$ su un vettore
%   $X \in \R^3$ è uguale all'azione della forma $r\sin\theta \dd \varphi$ su un
%   vettore tangente (nel senso di proiezione nella coordinata $\varphi$)
%},
%si riconosce in $F$ la componente radiale del campo magnetico di un monopolo, con
%$g = n/2$.
%$$
%   \vec B = \frac{g}{r^2} \vec u _r
%          = \frac{g}{r^2} \, (r\vec u _\theta) \times (r\sin\theta \vec u _\phi)
%          = g \sin \theta \, \vec u _\theta \times \vec u _\phi
%$$

La 2-forma $F$ è chiusa ($\dd F = \dd ^2A = 0$ su entrambi gli emisferi),
analogamente al caso del fibrato banale, ma questa volta non è esatta perchè i due potenziali
sono definiti su insiemi differenti.\\

Nella regione di intersezione $U^0$ i due potenziali sono vincolati dalla condizione
di compatibilità \ref{eq:condcompatibility}, la trasformazione di gauge \ref{eq:gaugetrasf}
$$
   A^+ = \Phi A^- \Phi ^{-1} + \frac{i}{e}\Phi ^{-1} \dd \Phi
       = A^- + 2g \dd  \varphi
$$
Si osserva che siccome il gruppo di gauge è abeliano si può far commutare $A^+$
con $\Phi$ e semplificarlo nel primo termine. Questo non sarà possibile in una
teoria di gauge non abeliana. \\
Si definiscono allora le funzioni di transizione \ref{eq:transfunctions}
che sono elementi di $\Phi = e^{i 2eg \varphi} \in U(1)$ che collegano le coordinate delle fibre
$$
    e^{i\alpha^+} = \Phi  e^{i\alpha^-} = e^{i2eg\varphi } e^{i\alpha^-}
$$

Si osserva inoltre che anche la funzione d'onda di una particella carica in un
campo di monopolo non può essere definita globalmente.
Si definiscono su $U^\pm$ le funzioni d'onda $\psi^\pm$ che su $U^0$ sono collegate
da $\psi^+ = \Phi  \psi^-$. L'effetto della trasformazione di gauge è di cambiare la fase
della funzione d'onda. Affinchè la fase sia ben definita e non multivalore, è necessario
che sia rispettato
$$
   e^{2ige \varphi} \psi ^- = \psi ^+ =  e^{2ige (\varphi + 2\pi)} \Rightarrow
   \Rightarrow
   \boxed{
   2ge = n \:,\: n \in \Z
   }
$$

Le funzioni di transizione $\Phi$ sono mappe dell'equatore $S^1 \subset S^2$ nel gruppo
di gauge $U(1)$: sono quindi cammini in $U(1)$ e possono essere
classificate in base alla classe di omotopia in $\pi_1(U(1))$ a cui appartengono.
L'intero $n$ assume il significato topologico di \emph{numero di avvolgimento}
della corrispondente classe di omotopia.\\

Come visto sopra, la topologia del monopolo è caratterizzata dalla classe di Chern
relativa a $c_1$
\begin{equation*}
%   \begin{aligned}
      c_1   = \frac{1}{2\pi} \int_{S^2} F
            = \frac{1}{2\pi} \left( \int_{U^+} \dd A^+ \int_{U^-} \dd A^- \right)
            = \frac{1}{2\pi} \int_{S^1} (A^+-A^-)
            = \frac{1}{2\pi} \int_0^{2\pi} n\dd \varphi   = n
%   \end{aligned}
\end{equation*}
La carica magnetica coincide quindi con la prima classe di Chern, in quanto
può essere interpretata come il flusso del campo magnetico $F$ attraverso una sfera.\\
Per $n=0$ il fibrato è banale e ha la forma $S^2 \times S^1$, mentre per $n=1$ si ha il
\emph{fibrato di Hopf}, di seguito descritto.

%------------------------------------------------------------------------------%
%------------------------------------------------------------------------------%
\section{Fibrato di Hopf}
Il fibrato di Hopf descrive la sfera $S^3$ come fibrato con spazio
base $S^2$ (parametrizzato dagli angoli $\theta,\varphi $) e come fibra $S^1$ (parametrizzato,
dal paramentro $\alpha$, come sopra).\\

La sfera $S^3$ è definita da:
$$ S^3 = \{ \vec p \in \R^4 \: : \: |\vec p|^2 = 1 \} $$
e può essere parametrizzata in coordinate cartesiane da $(p_1,p_2,p_3,p_4)$
tali che:
\begin{equation}
   \begin{cases}
      p_1 = \cos \frac{\theta}{2} \cos \alpha \\
      p_2 = \cos \frac{\theta}{2} \sin \alpha \\
      p_3 = \sin \frac{\theta}{2} \cos (\varphi  + \alpha) \\
      p_4 = \sin \frac{\theta}{2} \sin (\varphi  + \alpha) \\
   \end{cases}
\end{equation}

Assegnata una base $ \{ \partial/\partial p_\mu \} $ dello spazio tangente
$T_p S^3$, si può definire la metrica
\begin{equation}
   \dd  s^2 = g_{\mu\nu} \dd x^\mu \dd x^\nu
         = \frac{1}{4} \dd \theta^2 + \dd \alpha^2 + \sin^2\frac{\theta}{2}\dd \varphi ^2
         + 2\sin^2\frac{\theta}{2} \dd \varphi  \dd \alpha
\end{equation}
La proiezione $\pi : S^3 \to S^2$ (mappa di Hopf) è defnita nel modo seguente
$\vec p = (p_1,p_2,p_3,p_4) \mapsto \vec x = (x,y,z)$
\begin{equation}\label{eq:hopfmap}
   \begin{cases}
      x = 2(p_1 p_3 + p_2 p_4)          = \sin\theta \cos\varphi  \\
      y = 2(p_1 p_4 - p_2 p_3)          = \sin\theta \sin\varphi  \\
      z = p_1^2 + p_2^2 - p_3^2 - p_4^2 = \cos\theta          \\
   \end{cases}
\end{equation}
Si osservi che si perde completamente la dipendenza dalla variabile $\alpha$.
L'effetto della mappa di Hopf è di mappare un cerchio $S^1$ in un punto
della sfera $S^2$.\\

La sezione del fibrato può essere presa fissando un particolare valore di $ \alpha $.
Fissato un punto sulla sfera $ S^2 $ (parametrizzato da due coordinate angolari
$ \theta,\phi $), può essere immerso in $ S^3 $ completando arbitrariamente il grado
 di libertà mancante $\alpha$
$$
   (\theta,\varphi) \mapsto (\theta',\varphi',\alpha).
$$
Si costuiscono ($ \theta',\varphi' $) con opportune combinazioni lineari degli angoli
($ \theta, \varphi $), tali che applicando la proiezione si ottenga correttamente il
punto di partenza.\\

Per eliminare l'arbitrarietà nella scelta di $\alpha$ occorre fissare un altro
vincolo.
Si richiede che la metrica sulla sfera $S^3$ si riduca alla tradizionale metrica
$\dd s^2 = \dd \theta^2 + \sin^2\theta \dd \varphi $ sulla sfera $S^2$ sia nell'emisfero
nord $U^+$($ \theta/2 \mapsto \theta $) che nell'emisfero sud $U^-$
($\pi /2 - \theta /2 \mapsto \theta $). Si vede immediatamente da \ref{eq:hopfmap} che la
condizione è verificata per $\alpha = 0$ e $\alpha = -\varphi $, rispettivamente.\\

Si definisce allora la 1-forma di connessione $\omega$ su $S^3$
\begin{equation}
   \begin{aligned}
      \omega & = p_1 \dd  p_2 - p_2 \dd  p_1 + p_3 \dd  p_4 - p_4 \dd  p_3 \\
             & = \dd \alpha + \frac{1}{2}(1 - \cos\theta) \dd  \varphi
   \end{aligned}
\end{equation}
Si osserva che $\omega$ viene mappata,
dal pullback della sezione appena descritta, nella forma locale $A$
definita in \ref{eq:hopfpotential}, nel caso particolare di $n = 1$.\\

Le funzioni di transizione (\ref{eq:transfunctions}) $\Phi$ sono definite nella regione di transizione
$U^0 = U^+ \cap U^-$, l'equatore, come mappe $S^1 \to S^1$
\begin{equation}
   \begin{aligned}
  \Phi_{NS} & = \Phi _+ \circ \Phi _- ^{-1}
             = \frac{p_3 +  i \, p_4}{p_1 + i \, p_2}
                \sqrt{ \frac{p_1^2 + p_2^2}{p_3^2 + p_4^2} }
             = e^{i\varphi } \\
  \Phi _{SN} & = \Phi _- \circ \Phi _+ ^{-1}
             = \frac{p_1 + i \, p_2}{p_3 +  i \, p_4}
                  \sqrt{ \frac{p_3^2 + p_4^2}{p_1^2 + p_2^2}}
             = e^{-i\varphi }
   \end{aligned}
\end{equation}

Si definisce la curvatura $\Omega$
$$
   \Omega = \dd  \omega = 2 ( \dd  p_1 \wedge \dd  p_2 + \dd  p_3 \wedge \dd  p_4)
          = \frac{1}{2} \sin\theta \dd \theta \wedge \dd \varphi
$$
Questa forma è chiusa ed esatta su $S^3$ e si nota subito che corrisponde a $1/2$
della forma di volume della sfera $S^2$\footnote{
   che è appunto $\dd S = \sin\theta \dd \theta \wedge \dd \varphi $}.
Si ha allora, integrando su $S^2$ come sottospazio di $S^3$\footnote{
   Abuso di notazione.}
$$
   \int_{S^2} \Omega = 2\pi.
$$

In sintesi, un modello abeliano del monopolo magnetico si costruisce definendo
la struttura di fibrato di Hopf sulla sfera $S^2$. La definizione della 1-forma
di connessione globale $\omega$ porta alla definizione del potenziale $A$
e del tensore elettromagnetico $F$ che descrivono correttamente il campo
generato da un monopolo (si veda \ref{eq:diracpotential} e \ref{eq:monopolefield}).\\
Abbandonando l'idea del potenziale definito ovunque in favore di una descrizione
in carte locali, viene risolto il problema della stringa di singolarità, evidenziato
nel capitolo \ref{cap:diracmonopole}.\\
Si sottolinea che, nella costruzione del modello, non si è prestata attenzione
a una corretta formulazione relativistica o quantistica. Volendo applicare questi
due formalismi al modello abeliano si incontrano delle difficoltà che non verranno
qui trattate.\\

Per concludere il capitolo, si vuole dare un'interpretazione del modello
matematico appena discusso. L'esistenza di due classi di chern del tipo $c_1$,
legate a $n = 0,1$, suggerisce la seguente considerazione: un monopolo abeliano
può esistere solamente se la struttura topologica dell'universo è non banale ($n=1$).
Il fallimento nell'osservazione di monopoli magnetici sarebbe quindi
un argomento in favore di una struttura topologica banale ($n=0$).
