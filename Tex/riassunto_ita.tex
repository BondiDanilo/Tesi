
\documentclass[10pt,a4paper]{report}
\usepackage{graphicx}
\usepackage[utf8]{inputenc}
\usepackage{amsthm}             %stili teoremi
\usepackage{amssymb,amsmath}
\usepackage[colorlinks = true,
          linkcolor = blue,
          urlcolor  = blue,
          citecolor = blue,
          anchorcolor = blue]{hyperref}
\usepackage{bookmark}
\usepackage{wrapfig}

\usepackage{geometry}
\geometry{
	top=25mm,
	bottom=25mm,
	left=30mm,
	right=25mm
}
\usepackage[ampersand]{easylist}
\usepackage[nottoc]{tocbibind} %bibliografia
%------------------------------------------------------------------------------%

% Stile+numerazione teoremi, definizinoni, etc...
\theoremstyle{plain}
\newtheorem{theorem}{Teorema}[equation]

\theoremstyle{definition}
\newtheorem{definition}{Definizione}[equation]

\theoremstyle{plain}
\newtheorem{proposition}{Proposizione}[equation]

\theoremstyle{plain}
\newtheorem{lemma}{Lemma}[equation]

\theoremstyle{remark}
\newtheorem{example}{Esempio}[equation]

\theoremstyle{definition}
\newtheorem{axiom}{}[section]

\renewcommand\qedsymbol{$\blacksquare$} % quadrato della dimostrazione
%------------------------------------------------------------------------------%
\renewcommand{\vec}[1]{\mathbf{#1}} % simbolo di vettore = grassetto
\newcommand{\chrsym}{\genfrac{\{}{\}}{0pt}{}} % Simboli Christoffel
%------------------------------------------------------------------------------%
\graphicspath{{/home/dan/Desktop/UNI/TESI/Images/}}	%Default path for graphics
%------------------------------------------------------------------------------%
% Remove default parindent
\newlength\tindent
\setlength{\tindent}{\parindent}
\setlength{\parindent}{0pt}
\renewcommand{\indent}{\hspace*{\tindent}}

\newcommand{\tab}[1][1cm]{\hspace*{#1}} % ridefinisce la tabulazione

%------------------------------------------------------------------------------%
% Metadati
\hypersetup{
	pdftitle={Tesi},%
	pdfauthor={Danilo Bondì},%
	pdfsubject={Aspetti classici e quantistici dei monopoli magnetici in teorie di gauge},%
	pdfkeywords={},%
	colorlinks=true,%
	linkcolor=blue,%
	linktocpage=true,%
	pageanchor=true
}

\begin{document}


\chapter*{
   Aspetti classici e quantistici dei monopoli magnetici in teorie di gauge
}


In natura non sono mai stati osservati i monopoli magnetici. La ragione non è però
spiegata dall'elettrodinamica classica, in quanto le equazioni di Maxwell
si limitano a constatarne la non-esistenza. Le leggi dell'elettrodinamica
sono basate su osservazioni sperimentali, non teoremi derivati da assiomi primi.
Non si esclude quindi a priori l'esistenza dei monopoli magnetici, ma non se ne
è trovata, ad oggi, conferma sperimentale. \\
Nonostante lo scarso successo dell'ipotesi del monopolo magnetico a livello sperimentale,
nel corso del secolo scorso non è cessato l'interesse nel formularne una teoria
completa e consistente. Una delle ragioni principali per cui tale ricerca viene
portata avanti, oltre alle numerose implicazioni a livello
teorico e sperimentale, è la mancata comprensione della motivazione
per cui non siano mai stati osservati monopoli magnetici in natura.\\

Un primo approccio per introdurre una teoria del monopolo magnetico, volto solamente
a evidenziarne le problematiche, è la costruzione di un elementare modello classico:
assumendo l'esistenza di una carica magnetica $g$, si scrivono le equazioni del moto
di un elettrone immerso in un campo magnetico di monopolo, analogo al campo elettrico
coulombiano prodotto da una carica elettrica isolata
(monopolo elettrico). Occorre quindi modificare le equazioni di Maxwell,
eguagliando la divergenza del campo magnetico $\vec B$ alla densità di carica magnetica
locale $\rho_g$.
Emerge però subito una contraddizione: se si vuole definire
un potenziale elettromagnetico $\vec A$, il cui rotore è il campo magnetico $\vec B$,
si ha incompatibilità tra le due condizioni $\nabla \cdot \vec B = 4 \pi\rho_g$ e
$\vec B = \nabla \times \vec A$. Il motivo è l'impossibilità di definire ovunque un
potenziale vettore regolare: esiste sempre un semiasse in cui il potenziale è singolare.\\

Per costruire una teoria non contradditoria bisogna rinunciare alla definizione
di un potenziale globale, in favore di una descrizione con più potenziali definiti
localmente e che si raccordino in maniera "corretta" nelle regioni in cui si intersecano.\\
Il formalismo più adatto a questa descrizione è quello delle teorie di gauge, in
cui l'accento è posto sul comportamento locale dei campi.\\

Ricordando che l'elettrodinamica classica è una teoria di gauge con simmetria
di gruppo $U(1)$, si inizia tentando di definire, su due aperti, due potenziali
gauge legati nella regione di intersezione da una trasformazione di
gauge di tipo $U(1)$. Si arriva dunque a costruire il campo di monopolo
con l'andamento corretto (già definito nell'esempio "alla Coulomb"), risolvendo
la contraddizione evidenziata in precedenza. e a dare un significato topologico
alla carica magnetica che viene associata alle classi di Chern della varietà
in esame.
Osservazioni sulla non-polidromia della trasformazione dei campi permettono di
ottenere una relazione tra le cariche elettrica e magnetica - relazione che costiuisce
uno dei principali interessi nel costruire una teoria del monopolo magnetico.\\
Nell'estendere il modello a una teoria di campo quantistica,
il modello abeliano presenta tuttavia alcuni punti critici.\\

Si vuole allora generalizzare la teoria a un gruppo di gauge non abeliano di cui
$U(1)$ è sottogruppo, la quale venga ridotta alla precedente teoria abeliana in
condizioni "normali", ad esempio a basse energie.
Questo processo di rottura spontanea della simmetria comprende tutte le previsioni
del modello abeliano ed ha il vantaggio di risolverne le criticità.
Teorie di campo che generalizzano
l'elettrodinamica in questo modo prendono il nome di teorie di \emph{Yang-Mills}.\\
Il caso più semplice è quello in cui viene preso come gruppo di gauge il gruppo
$SU(2)$, che ha appunto $U(1)$ come sottogruppo.
Vengono presi in esame due modelli di monopolo magnetico.
Il primo, proposto da Wu e Yang nel 1969, risolve i problemi del modello abeliano,
ma deficita nella definizione delle energie di configurazione dei campi, che
risulta divergente.
Il secondo modello in esame è un caso
particolare del modello proposto da Georgi e Glashow nel 1974, in cui
si accoppia il potenziale di gauge a un campo, il campo di Higgs. Si manifesta
immediatamente il processo di rottura di simmetria da $SU(2)$ a $U(1)$ e viene
risolto il problema delle configurazioni a energia divergente. Si arriva inoltre
a dare un ulteriore definizione della carica magnetica, associandola alla
conservazione del tensore elettromagnetico, discendente quindi dalle equazioni di
Maxwell stesse.\\

Le difficoltà di un modello non abeliano sono certamente anche di natura
computazionale. Si danno allora alcuni cenni, in conclusione, al modello di
soluzione proposto da 't Hooft e Polyakov, che è un punto di partenza per le
soluzioni numeriche del modello di Georgi-Glashow.

\end{document}
