
\documentclass{beamer}

\usepackage{graphicx}

\usepackage[absolute,overlay]{textpos}   % posizione assoluta immaginin e testo
\setlength{\TPHorizModule}{1cm}  % mette unità misura in multipli di 1mm
\setlength{\TPVertModule}{1cm}   % per textpos
\usepackage{float}
\usepackage{wrapfig}

\usepackage{subcaption}
\graphicspath{{/home/dan/Desktop/UNI/TESI/Images/}}

\usepackage[utf8]{inputenc}
\usepackage[italian]{babel}
\usepackage{amssymb,amsmath}

\usetheme{dan}

\title{Aspetti classici e quantistici dei monopoli magnetici in teorie di gauge}
\author[Bondì,Zaffaroni]{
    Candidato: ~Bondì Danilo \and \\
    Relatore: ~Prof. Alberto Zaffaroni
    }
\date{24 settembre 2018}

%\institute{Università degli Studi Milano Bicocca}
%\logo{\includegraphics[height=1.5cm]{logo.jpg}}

\newcommand{\dd}{\: \mathrm{d}} % differenziale
\renewcommand{\vec}[1]{\mathbf{#1}} % simbolo di vettore = grassetto
\newcommand{\vers}[1]{\hat{\vec{#1}}} % simbolo di versore  = grassetto

\newcommand{\Z}{\mathbb{Z}}
\newcommand{\R}{\mathbb{R}}
