\documentclass{report}
\usepackage{amsmath}
\usepackage[utf8]{inputenc}
\usepackage[italian]{babel}

\begin{document}
Come tutti sapete, $\nabla \cdot B = 0$ implica che non esistono nell'universo
cariche magnetiche isolate. Questo è confermato sperimentalmente: non sono mai tutt'ora
stati osservati monopoli magnetici. Ma.\\
Facciamo finta per il prossimo quarto d'ora che ciò non sia vero.\\

Supponiamo l'esistenza di una carica magnetica $g$ isolata, posta nell'origine
del nostro sistema di riferimento, e consideriamo il campo magnetico che produce.
Data la perfetta simmetria con l'elettrostatica, il campo sarà di tipo coulombiano
(indicare slide). Vogliamo ora quindi modificare le equazioni di maxwell
cambiando $\nabla \cdot B = 0$ con $\nabla \cdot B = 4\pi\rho_g$(spiegarla con il flusso),
in analogia
con la legge di gauss per il campo elettrico. Arriviamo subito però a una contraddizione.\\
Se volessimo definire un potenziale vettore per la nostra teoria, si ha incompatibilità
tra queste due (indicare slide). Vediamo come mai.\\

Il potenziale vettore così definito (di immediata verifica che genera
il campo magnetico di monopolo) è singolare lungo l'asse z negativo. Si può
tentare di regolarizzarlo e definire un campo regolarizzato ritagliando
un cilindro di raggio $\varepsilon$ attorno all'asse e valutando il limite per
$\varepsilon \to 0$. Si ottiene questa regolarizzazione per il campo, che se integrata
da un flusso totale nullo perchè i due contributi si cancellano esattamente.
Il ruolo della stringa è quello di generare un campo costante lungo l'asse delle
singolarità che annulli il flusso totale.\\
Abbiamo, in maniera molto sporca, aggirato la contradizione. Osserviamo che il campo
così ottenuto è analogo al campo prodotto da un solenoide infinito e infinitamente
sottile, in prossimità di una delle estremità.
Poniamoci a distanza infinitesima dalla stringa, tale che ci sia solo la sua linea
di campo. Si ha una regione in cui c'è campo magnetico e una in cui no, ma il potenziale
è non nullo. La configurazione richiama il tipico esperimento alla Aharonov-Bohm.
(spiegare brevemente). Si ha quindi che un elettrone lungo cammini diversi acquisisce
fasi differenti. \\
La stringa però non è una cosa fisica, quindi occorre che il suo effetto non debba
essere misurabile. Questo vuol dire che per un percorso chiuso attorno all'asse z
la fase della funzione d'onda dell'elettrone non deve cambiare.\\
Questo porta alla condizione di quantizzazione di dirac. \\
Abbiamo allora che l'esistenza di una singola carica magnetica spiegherebbe perchè
la carica elettrica è quantizzata. Siccome questo è precisamente ciò che si osserva,
capiamo quale sia il tipo di interesse nella ricerca dei monopoli.\\

Un modello teorico più completo va inserito nel contesto delle teorie di gauge.
Poichè una teoria di gauge è una teoria che pone l'accento sul comportamento locale
dei campi, rinunciamo ad una definizione globale del potenziale vettore
in favore di più descrizioni in carte locali. Queste devono chiaramente
concordare nelle regioni di intersezione con una trasformazione di gauge.\\

L'elettrodinamica classica è una teoria di gauge del gruppo U(1). Si può costruire
una teoria abeliana del monopolo magnetico, ma presenta vari problemi. Consideriamo
allora una teoria che ha come gruppo di gauge un generico $G$, che contiene $U(1)$.
Teorie di questo tipo vengono dette teorie di Yang-Mills. L'esempio più semplice
è quello di $G=SU(2)$. Si ricorda, Innanzitutto, che l'algebra di Lie associata SU(2)
ha dimensione 3, quindi il gruppo ha 3 generatori. Ad ogni generatore è associato un
potenziale di gauge. Definiamo allora il potenziale generalizzato $A_\mu$, che generalizza
il potenziale di monopolo trovato all'inizio.\\

Vogliamo scrivere una teoria che descriva l'accoppiamento di $A_\mu$ con 3 campi scalari
$\phi^a$. La Lagrangiana del modello, descritto da G.G., è questa (slide, commentare?)
Dove le quantità sono così definite... Il tensore elettromagnetico però ha la componente
di commutatore dei potenziali, perchè siamo in una teoria non abeliana. Il tensore elettromagnetico
non abeliano generalizza correttamente il campo di monopolo magnetico. con l'andamento
corretto all'infinito e in una zona vicino al monopolo è a determinare risolvendo
numericamente le equazioni di moto dei campi (campo a riccio, non banale). esempio
't Hooft-Polyakov.\\

Prendiamo il duale di questo tensore per una soluzione non banale di campo a riccio.
si ha che la sua derivata è non nulla, ma uguale a un quadrivettore, che a sua volta
ha derivata nulla. Esiste allora una quantità conservata, che definiamo essere
la carica magnetica. Dopo conti si ottiene che ha questa espressione (slide).
Dove n questa volta ha un significato topologico molto più profondo. È il numero
di avvolgimento della soluzione di campo, all'infinito (spiegare numero di avvolgimento)
Le varie soluzioni di campo a diverso n sono fisicamente separate da barriere
infinite di energia, mentre topologicamente appartengono a classi di omotopia diverse.\\
La condizione è l'analogo della condizione di dirac.\\

Si vede allora come la carica magnetica assume un significato più profondo nel contesto
di una teoria non abeliana: è legata alla topologia dell'universo. Topologia banale
ha n = 0 e quindi no carica magnetica. Topologia non banale ha n >= 1 ed esistono i Monopoli.\\

In conclusione, è adesso più chiaro perchè nonostante i fallimenti sperimentali
sia vivo l'interesse per la ricerca dei monopoli magnetici.
Molti altri sono gli ambiti in cui il monopolo magnetico risolverebbe i problemi, tipo...
Ma più di tutti, ripristinerebbe la simmetria rotta tra campo elettrico e campo magnetico,
scrivendo le equazioni
$$ \nabla \cdot E = 4\pi \rho \quad \nabla \cdot B = 4\pi\rho_g $$
e a noi fisici piace la simmetria, anche più della figa.\\

Grazie del vostro tempo, e arrivederci. Ora vado a brindare con l'idromele.

\end{document}
